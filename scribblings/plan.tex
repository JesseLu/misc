\documentclass{article}
\usepackage{graphicx}
\usepackage{amsmath}
\usepackage{hyperref}
\input defs.tex
\newcommand{\inv}[1]{\frac{1}{#1}}
\newcommand{\tC}{\tilde{C}}
\newcommand{\BSE}{\begin{subequations}\begin{align}}
\newcommand{\EA}{\end{align}}
\newcommand{\ESE}{\end{subequations}\end{align}}
\newcommand{\env}[2]{\begin{#1}#2\end{#1}}
\newcommand{\E}[2]{\begin{subequations}\begin{align}#1\end{align}\label{eq:#2}\end{subequations}}
\DeclareMathOperator*{\minimize}{\text{minimize}\quad}
\newcommand{\subto}{\text{subject to}\quad}
\newcommand{\er}[1]{\eqref{eq:#1}}
\newcommand{\BI}{\begin{itemize}\item}
\newcommand{\EI}{\end{itemize}}
\newcommand{\I}{\item}
\title{Objective-First Nanophotonic Design Plan} 
\author{Jesse Lu}
\begin{document}
\maketitle
\tableofcontents

\section{Goal}
    Software to solve the following general inverse design problem, specfically for nanophotonics.
    \E{ \minimize&  f(x) + g(z) \\
        \subto&     A(z)x - b(z) = 0}{prob}
    where $f(x)$ and $g(z)$ are the \emph{design objectives} for the field ($x$) and structure ($z$) variables respectively,
    and $A(z)x - b(z)$ is the \emph{physics residual} of the problem.

\section{Strategy}
    The general strategy is to divide the problem 
        into field and structure sub-problems,
        which can be tackled separately and in a modular fashion
        by using various
        \emph{optimization strategies} and 
        \emph{structure parameterizations}
        interchangeably.

    Specifically, the available optimization strategies are
    \BI adjoint
    \I  ob-1    \EI
    and the available structure parameterizations include
    \BI point
    \I  boundary
    \I  shape
    \I  include/exclude \EI

\end{document}
