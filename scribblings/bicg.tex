\documentclass{article}
\usepackage{amsmath}
\usepackage{hyperref}
\title{Notes on the Bi-Conjugate Gradient Method}
\author{Jesse Lu}
\begin{document}
\maketitle
\tableofcontents
\section{The algorithm}
\subsection{Asymmetric case, $A \ne A^T$}
The algorithm attempts to solve $Ax = b$,
    where the matrix $A$ is indefinite.
The algorithm accepts as inputs
    \begin{enumerate}
    \item a method to multiply a vector by $A$,
    \item a method to multiply a vector by $A^T$ 
        (note that this is the transpose of $A$, 
        \emph{not} its conjugate-transpose), and
    \item the vector $b$.
    \end{enumerate}

The algorithm begins by initializing the following variables,
    \begin{subequations}\begin{align}
    r_0 &= b - Ax_0 \\
    \hat{r}_0 &= b - A^T \hat{x}_0 \\
    p_0 &= r_0 \\
    \hat{p}_0 &= \hat{r}_0.
    \end{align}\end{subequations}
    \begin{subequations}\begin{align}
    \end{align}\end{subequations}
Then loop over the following for $k = 0, 1, \ldots$
    \begin{subequations}\begin{align}
    \alpha_k &= \hat{r}_k^T r_k / \hat{p}_k^T A p_k \\
    x_{k+1} &= x_k + \alpha_k p_k \\
    \hat{x}_{k+1} &= \hat{x}_k + {\alpha_k} \hat{p}_k \\
    r_{k+1} &= r_k - \alpha_k A p_k \\
    \hat{r}_{k+1} &= \hat{r}_k - {\alpha_k} A^T \hat{p}_k \\
    \beta_k &= \hat{r}_{k+1}^T r_{k+1} / \hat{r}_k^T r_k \\
    p_{k+1} &= r_{k+1} + \beta_k p_k \\
    \hat{p}_{k+1} &= \hat{r}_{k+1} + {\beta_k} \hat{p}_k.
    \end{align}\end{subequations}

\subsection{Symmetric case, $A = A^T$}
\subsubsection{Basic version}
Initialize
    \begin{subequations}\begin{align}
    r_0 &= b - Ax_0 \\
    p_0 &= r_0.
    \end{align}\end{subequations}
Then loop over the following for $k = 0, 1, \ldots$
    \begin{subequations}\begin{align}
    \alpha_k &= r_k^T r_k / p_k^T A p_k \\
    x_{k+1} &= x_k + \alpha_k p_k \\
    r_{k+1} &= r_k - \alpha_k A p_k \\
    \beta_k &= r_{k+1}^T r_{k+1} / r_k^T r_k \\
    p_{k+1} &= r_{k+1} + \beta_k p_k. 
    \end{align}\end{subequations}

\subsubsection{Efficient version}
We now attempt to eliminate redundant calculations by introducing variables
    $v_k$ and $\rho_k$.
The algorithm then proceeds by initializing
    \begin{subequations}\begin{align}
    r_0 &= b - Ax_0 \\
    p_{-1} &= r_0 \text{ (arbitrary)} \\
    \rho_{0} &= r_0^T r_0 \\
    \beta_0 &= 0,
    \end{align}\end{subequations}
and looping over the following for $k = 0, 1, \ldots$
    \begin{subequations}\begin{align}
    p_{k} &= r_{k} + \beta_k p_{k-1} \\
    v_k &= A p_k \\
    \alpha_k &= \rho_k / p_k^T v_k \\
    x_{k+1} &= x_k + \alpha_k p_k \\
    r_{k+1} &= r_k - \alpha_k v_k \\
    \rho_{k+1} &= r_{k+1}^T r_{k+1} \\
    \beta_{k+1} &= \rho_{k+1} / \rho_k.
    \end{align}\end{subequations}

\subsubsection{Efficient version with overwrites}
If we eliminate all possible subscripts then 
    the algorithm then proceeds by initializing
    \begin{subequations}\begin{align}
    r &= b - Ax \\
    p &= r \text{ (arbitrary)} \\
    \rho_{0} &= r^T r \\
    \beta &= 0,
    \end{align}\end{subequations}
and looping over the following for $k = 0, 1, \ldots$
    \begin{subequations}\begin{align}
    p &= r + \beta p \\
    v &= A p \\
    \alpha &= \rho_k / p^T v \\
    x &= x + \alpha p \\
    r &= r - \alpha v \\
    \rho_{k+1} &= r^T r \\
    \beta &= \rho_{k+1} / \rho_k.
    \end{align}\end{subequations}

\subsubsection{Lumped version}
Additionally, the loop operations can be ``lumped'' into two steps.
By lumping, we mean that all the calculations in one step can proceed
    simultaneously, before any one calculation finishes,
    as long as intermediate values are available.
For this algorithm the loop is composed of the following two steps:
\begin{enumerate}
\item $\rho$-step: $\rho\_\text{step}(\alpha, p, v, r, x) 
                    \rightarrow \rho_k, \text{err} $.
    Requires four loads, $x$, $p$, $r$, and $v$, and two writes, $x$ and $r$.
    Note that the error, $\|r\|$, should also be computed on this step.
    \begin{subequations}\begin{align}
    x &= x + \alpha p \\
    r &= r - \alpha v \\
    \rho_k &= r^T r \\
    \text{err} &= \|r\| 
    \end{align}\end{subequations}
\item $\alpha$-step: $\alpha\_\text{step}(\rho_k, \rho_{k-1}, p, v, r) 
                        \rightarrow \alpha$.
    Requires two loads, $p$ and $r$, and two writes, $p$ and $v$. 
    Additional loads may also be required to perform the multiplication by $A$.
    \begin{subequations}\begin{align}
    p &= r + \rho_k / \rho_{k-1} p \\
    v &= A p \\
    \alpha &= \rho_k / p^T v,
    \end{align}\end{subequations}
\end{enumerate}

The algorithm is initiated by setting
    \begin{subequations}\begin{align}
    \alpha &= 1 \\
    p &= 0 \\
    v &= Ax \\
    r &= b.
    \end{align}\end{subequations}
Note that $v$ can be initiated through the use of 
    $r =b, \alpha\_\text{step}(0, 1, 0, v, r)$.
The algorithm then proceeds via:
    \begin{subequations}\begin{align}
    \rho_k, \text{err} &= \rho\_\text{step}(\alpha, p, v, r, x) \\
    & \text{check termination condition} \\
    \alpha &= \alpha\_\text{step}(\rho_k, \rho_{k-1}, p, v, r) 
    \end{align}\end{subequations}




\appendix
\section{Reference algorithm: From C. T. Kelley}
Found in section 3.6.1 of \cite{kelley}.
Initialize as follows:
\begin{equation}
r = b - Ax, \hat{r} = r, \rho_0 = 1, \hat{p} = p = 0, k = 0.
\end{equation}
Repeat the following until a termination condition 
    such as $\|r\|_2 < \epsilon \|b\|_2$ is satisfied,
    \begin{subequations}\begin{align}
    k &= k + 1 \\
    \rho_k &= \hat{r}^T r, \beta = \rho_k / \rho_{k-1} \\
    p &= r + \beta p, \hat{p} = \hat{r} + \beta \hat{p} \\
    v &= Ap \\
    \alpha &= \rho_k / (\hat{p}^T v) \\
    x &= x + \alpha p \\
    r &= r - \alpha v, \hat{r} = \hat{r} - \alpha A^T \hat{p}.
    \end{align}\end{subequations}

\section{Reference algorithm: From Wikipedia}
First, choose initial vectors $x_0$, $x_0^\ast$, and $b^\ast$.
Initialize using
    \begin{subequations}\begin{align}
    r_0 &= b - Ax_0 \\
    r_0^\ast &= b^\ast - x_0^\ast A \\
    p_0 &= r_0 \\
    p_0^\ast &= r_0^\ast.
    \end{align}\end{subequations}
Then loop over the following for $k = 0, 1, \ldots$
    \begin{subequations}\begin{align}
    \alpha_k &= r_k^\ast r_k / p_k^\ast A p_k \\
    x_{k+1} &= x_k + \alpha_k p_k \\
    x_{k+1}^\ast &= x_k^\ast + \overline{\alpha_k} p_k^\ast \\
    r_{k+1} &= r_k - \alpha_k A p_k \\
    r_{k+1}^\ast &= r_k^\ast - \overline{\alpha_k} p_k^\ast A \\
    \beta_k &= r_{k+1}^\ast r_{k+1} / r_k^\ast r_k \\
    p_{k+1} &= r_{k+1} + \beta_k p_k \\
    p_{k+1}^\ast &= r_{k+1}^\ast + \overline{\beta_k} p_k^\ast.
    \end{align}\end{subequations}

The residuals $r_k$ and $r_k^\ast$ satisfy
    \begin{subequations}\begin{align}
    r_k &= b - A x_k \\
    r_k^\ast &= b^\ast - x_k^\ast A,
    \end{align}\end{subequations}
    where $x_k$ and $x_k^\ast$ are the approximate solutions to
    \begin{subequations}\begin{align}
    Ax &= b \\
    x^\ast A &= b^\ast.
    \end{align}\end{subequations}



\begin{thebibliography}{99}
\bibitem{kelley} C. T. Kelley, 
    ``Iterative Methods for Linear and Nonlinear Equations'', 
    SIAM (1995).
\bibitem{pubdom} \url{http://en.wikipedia.org/wiki/Biconjugate_gradient_method}
\end{thebibliography}
\end{document}
