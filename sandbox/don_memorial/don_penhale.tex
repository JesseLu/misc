\documentclass{article}
\title{Don Penhale Memorial}
\begin{document}
\maketitle
\section{Outline for Don's Memorial}
(Above all present the service as an occasion for rejoicing.)
\begin{enumerate}
\item Music just before the service: 
        The Adagietto from Gustav Mahler's Symphony \#5 (CD)
\item Wally to sing: The Love of God
\item Read my brief history
\item Wally sing: He Took My Place
\item Read my testimony
\item Wally sing: Finally Home
\item Congregation sing: How Great Thou Art
\item Read Jude 24-25
\item Benediction de Deux by Franz Liszt played by Steven Hough (CD)
\end{enumerate}

\section{A brief history}
(A brief history of my life that you will give.)

Don was born in in Oakland, CA on July 25, 1922.
He lived on a chicken farm in Novato, Marin County, CA in an area
    known as Indian Valley.
It was rather primitive living initially, with an out-house,
    no electricity and a shallow well.
It was during the depression years.
He was given piano lessons, with the intent of making him a concert pianist.
When World Was II came along that put an end to that endeavor.
He enlisted in the Nacy and served with the amphibious forces
    in the Pacific theater.
Upon retirement in 1946 he resumed school at the University of California
    in Berkeley.
He roomed at a student rooming house owned by Stella whom he eventually married
    and adopted her daughter Mary.
He worked as an auditor for 
    the National Service Life Insurance office for 2 years,
    as a surveyor for the City of San Pablo for 2 years,
    as a laboratory assistant for Shell Development Company for 19 years,
    as a chemist fo Alza Corporation in Palo Alto for 5 years,
    and for SRI International in Menlo Park for 14 years.
He retired in 1989.
Stella passed away in 1993.

\section{A testimony}
(This is my testimony to be read by you)

Don left with me his testimony which he wanted me to read to you:\\

When I retired from the service I resumed school at UC Berkeley.
I married Stella, the owner of a rooming house for students.
She introduced me to Bible study and
    we attended First Prebyterian Church of Berkeley.

Now I would like to look back in retrospect and
    tell you how I discovered the absolute Sovereignty of God over my life.
I however will not bore you with the many trivial incidents 
    which constituted this discovery.

While in the Navy I was assigned as a deck officer on the crew 
    of an amphibious ship LST599.
We, with 150 Marines, were in on the invasion of Okinawa.
We were hit by a Japanese bomber who suicided into us by diving the bomber
    through our deck.
We had a raging fire but lost no lives.
When I was staring straight down the fuselage of that plane
    coming in at us I thought I was surely facing death.
When the commotion subsided I was alive.

After, I was troubled by the thoughts of what would have happened?
What was my destiny if I had any?
Those thoughts pervaded my thinking ever after until one evening,
    during an evangelistic meeting at First Prebyterian Church, Berkeley,
    where the speaker, Dr. Harold Okenga, 
    then president of the American Bible Society,
    spoke on the verse in the Bible in the book of John, verse 3:16,
    ``For God so loved the world that He gave His only begotten Son,
    that whosoever believeth in Him should not perish 
    but have everlasting life.''
The Holy Spirit led me to go forward and make a profession of faith,
    repenting of my sin,
    believing by God's gift of faith,
    that Jesus Christ was crucified in payment of my debt of sin,
    and was buried and rose again as the Victor over sin and death.

I knew I had been born again to a new life in Christ 
    and that I had received the answer to the question on my mind 
    as to my destiny.
Since then it has been a continuing process of learning 
    about the person of Jesus Christ.

Mind you, I have been everything but a perfect person
    but thanks be to God,
    I experience God's grace and mercy daily 
    for there is not a righteous bone in me.
I do not deserve it but He does it for me,
I know it and can testify to it with certainty.
As Paul wrote in Philippians 1:6,
    ``I can be confident of this very thing,
    that He who has begun a good work in you
    will complete it unto the day of Jesus Christ.''
I praise Him and thank Him for it.

I have found that the word of God is sufficient and practical 
    to be my guide to deal with the vicissitudes of life.
Paul knew whereof he spoke when he wrote these words 
    in his first letter to the Corinthians in chapter 10, verses 12 and 13,
    ``When ye think ye stand beware lest ye fall.
    There hath no temptation taken you but such as is common to man,
    but God is faithful who will not suffer you to be tempted
    above what ye are able,
    but with the temptation make a way of escape,
    that ye may be able to bear it.''
God's promises, admonitions, and doctrine from Scripture provide our means
    by which we meet and escape temptation.

I would like to share with you three favorite passages from Scripture.
\begin{itemize}
\item Philippians 3:7-15
\item Romans 8:28-39
\item Hebrews 13:5-6
\end{itemize}
I now rejoice in the incorruptible inheritance I possess.

I love all of you who are hearing these and sincerely admonish you:
    if you have not done so,
    to consider your position before God and repent of your sins,
    because Scripture does declare that,
    ``All have sinned and fall short of the glory of God.''
God gives you the faith to believe and receive Jesus Christ 
    as your personal Savior 
    so that we can all have a joyful reunion in Heaven for eternity.
I give thanks and praise to Jesus Christ. 
Amen.

\section{He Took My Place}

\begin{verse}
Sometimes I see through mists of bitter tears \\
A distant hill on which a cross appears; \\
And on that cross, where I myself should be, \\
I see the lowly man of Galilee.
\end{verse}

\begin{verse}
\emph{Chorus} \\
He took my place, His life He freely gave. \\
O boundless grace, my soul from sin to save. \\
He took my place upon the cruel tree. \\
He took a guilty sinner's place and I am free.
\end{verse}

\begin{verse}
O that my lips might speak His worthy praise; \\
And that my hands might serve Him all my days; \\
Until at last through His redeeming grace \\
I meet and greet the Man who took my place.
\end{verse}


\end{document}
