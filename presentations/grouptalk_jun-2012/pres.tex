\documentclass[landscape]{foils}
\usepackage{graphicx}
\usepackage{amsmath}
\usepackage[pdf]{pstricks}
\usepackage[off]{auto-pst-pdf}
\input defs.tex
\raggedright
\special{! TeXDict begin /landplus90{true}store end }
\renewcommand{\oursection}[1]{
\foilhead[-1.0cm]{#1}
}

% Custom figures.
\newcommand{\myfig}[1]{\begin{figure}[!h]\includegraphics[width=\textwidth]{fig/#1.jpg}\end{figure}}

\newcommand{\BE}{\begin{equation}}
\newcommand{\EE}{\end{equation}}
\newcommand{\BA}{\begin{eqnarray}}
\newcommand{\EA}{\end{eqnarray}}
\newcommand{\curl}{\nabla\times}
\newcommand{\minimize}[1]{\JLUminimize_{#1}\;&}
\newcommand{\subto}{\text{subject to}\;&}

\title{Jesse Lu, inverse design mini-conf, 2012-07-02}
\author{}
\MyLogo{}
\date{}

\begin{document}
\setlength{\parskip}{0cm}
\maketitle

% \BIT \itemsep -1pt
% \EIT

\vfill

\oursection{Ob-1 says...}
\BA \minimize{x,p} \| A(p) x - b(p) \|^2 \notag \\
    \subto f(x) = f_\text{ideal} \notag \\
        & p_0 \le p \le p_1. \notag \EA
\BIT
\I  $x \to H$
\I  $p \to \epsilon^{-1}$
\I  $A(p) x - b(p) \to \curl \epsilon^{-1} \curl H - \mu \omega^2 H$
\I  In general, $A(p) x - b(p) \ne 0$, the \emph{physics residual}
\I  $f(x)$ is the \emph{design objective}, always met.
\EIT
\myfig{intro}
\myfig{1}
\myfig{2}
\myfig{3}
\myfig{4}
\myfig{5}

\oursection{Ob-1, metamaterials also do, you must.}
\BIT 
\I  Try two classes: cloaks and mimics.
\I  Only modification: periodic boundary conditions on top and bottom.
\I  Allows for plane waves, and
\I  consequently only need $f_\text{ideal}$ at left and right (not top and bottom).
\EIT
\vfill
\newpage

\myfig{c1}
\myfig{c6}
\myfig{c2}
\myfig{c4}
\myfig{c7}
\myfig{c5}
\myfig{m1}
\myfig{m3}
\myfig{m4}
\myfig{m5}
\myfig{m2}
\myfig{m6}

% \oursection{One more thing...}
% \myfig{L3med}
% \myfig{wg1}
% \myfig{wg2}
% \myfig{wg3}
% \myfig{disk1}
% \myfig{disk2}
% \myfig{disk3}
% \myfig{disk4}
% \myfig{disk5}
% \myfig{disk6}
% \myfig{disk7}
% \myfig{med1}
% \myfig{med2}
% \myfig{med3}
% \myfig{med4}
% \myfig{med5}
% \myfig{hi1}
% \myfig{hi2}
% \myfig{hi3}

\end{document}
