\documentclass[landscape]{foils}
\usepackage{graphicx}
\usepackage{amsmath}
\usepackage{hyperref}
\usepackage{pstricks}
\usepackage{pst-grad}
\input defs.tex
\raggedright
\special{! TeXDict begin /landplus90{true}store end }
\renewcommand{\oursection}[1]{
\foilhead[-1.0cm]{#1}
}

\title{Maxwell: bringing cloud-powered electromagnetic simulations to Matlab}
\author{Advanced interface tutorial}
\MyLogo{Jesse Lu, Jelena Vuckovic group, Stanford University}
\date{}

\begin{document}
\setlength{\parskip}{0cm}
\maketitle

\BIT \itemsep -1pt
\I  Advanced interface quick-start
\I  Examples: Maxwell in action
\I  How Maxwell uses the cloud (EC2)
\I  How Maxwell solves electromagnetics
\EIT

\vfill

\oursection{Definitions}
\BIT
\I  What is Maxwell?
    \BIT
    \I  a Matlab toolset 
    \I  that uses Amazon's Elastic Compute Cloud (EC2)
    \I  to solve 3D frequency-domain electromagnetic simulations.
    \EIT

\I  Features:
    \BIT
    \I  Cryptographically-secure communication (https)
    \I  Full control over all simulation parameters
    \I  GPU-acceleration provided by Nvidia Tesla GPUs
    \I  Queueing system to allow for full usage of cluster
    \I  Scalable to hundreds of simultaneous simulations 
        running on hundreds of nodes.
    \EIT
\EIT

\newpage
Maxwell provides two user interfaces: Advanced and other
\BIT
\I  advanced:
\I  other:
\EIT
This presentation covers the advanced interface 

\oursection{Advanced interface quick-start}
\BIT \I Sign up at \EIT
\begin{verbatim}
% Download maxwell.m
>> urlwrite('m.lightlabs.co', 'maxwell.m');

% Provide AWS credentials and launch a 2-node cluster.
>> maxwell.aws_credentials('aws-access-id', 'aws-secret-key');
>> maxwell.launch('cluster-name', 2); 

% Run simulation on 1 node.
>> [E, H] = maxwell.solve('cluster-name', 1, ...); 

% Terminate cluster
>> maxwell.terminate('cluster-name');
\end{verbatim}

{Wait, what just happened?}
\BIT
\I  \verb=urlwrite()= downloaded the advanced interface for Maxwell,
\I  \verb=maxwell.aws_credentials()= provided the AWS credentials that
\I  \verb=maxwell.launch()= needed to create a cluster on EC2.
\I  \verb=maxwell.solve()= solved the electromagnetic simulation on
    the cluster and downloaded the resulting electromagnetic fields, and
\I  \verb=maxwell.terminate()= terminated the EC2 cluster.
\EIT

\oursection{Examples: Maxwell in action}

\oursection{How Maxwell uses the cloud (EC2)}
\BIT
\I  Maxwell uses your Amazon Web Services (AWS) account to
    \BIT
    \I  Create a custom Amazon EC2 cluster, and
    \I  Solve electromagnetic simulations on it;
    \EIT
    all without leaving your local Matlab environment.

 

\I  To get started, you need to
    \BIT
    \I  sign up for an AWS account,
    \I  retrieve your AWS security credentials, and
    \I  purchase the custom Maxwell Amazon Machine Image (AMI).
    \EIT
    For detailed instructions see Website.

\newpage
\I  Maxwell's advanced interface comprises of just five commands:
    \BIT
    \I  \verb+maxwell.aws_credentials()+
    \I  \verb+maxwell.launch()+
    \I  \verb+maxwell.solve()+
    \I  \verb+maxwell.solve_async()+
    \I  \verb+maxwell.terminate()+
    \EIT
    for full documentation use ``\verb+doc maxwell.command+'' in Matlab.

\I  \verb+maxwell.aws_credentials('aws-key-id', 'aws-secret-key');+
    \BIT
    \I  Stores the security credentials linked to your AWS account locally
    \I  Security credentials are used to launch and terminate clusters
    \I  Transmitted over https and never stored on server-side
    \I  Tutorial on obtaining your credentials at Website.
    \EIT
\EIT

\newpage
% Launch diagram.
\begin{center}
\psset{unit=2cm}
\begin{pspicture}(6,4)(-5,-1.0)
\psset{gridcolor=green, subgridcolor=yellow}
    \let\psgrid\relax
    % \psgrid
    \psframe[linestyle=none,
            fillstyle=gradient,
            gradbegin=white,gradend=lightgray,
            gradmidpoint=0.5,
            gradangle=0](0,-1.0)(6,3.8)
    \psline[linestyle=dotted](0,3)(0,-0.2)
    \rput(-2.5,3.5){Matlab}
    \rput(3,3.5){Amazon EC2}

    \psverbboxtrue
    \rput(-2.5,2.5){\small \verb=>> maxwell.launch(...)=}
    \psverbboxfalse

    
    \psline[linewidth=1pt](-0.6,2.5)(3,2.5) \rput[b](1.45, 2.6){\small{launch 3-node cluster}}
    \psline{->, linewidth=1pt}(3,2.5)(3,2)

    \psframe[linestyle=dashed](2.2,1.8)(3.8,0.8) \rput(3,1.3){master}

    \psframe[linestyle=dashed](0.4,0.6)(2,-0.4) \rput(1.2,0.1){node001}
    \psframe[linestyle=dashed](2.2,0.6)(3.8,-0.4) \rput(3,0.1){node002}
    \psframe[linestyle=dashed](4.0,0.6)(5.6,-0.4) \rput(4.8,0.1){node003}
\end{pspicture}
\end{center}

\BIT
\I  \verb+maxwell.launch('cluster-name', num_nodes);+
    \BIT
    \I  Creates an EC2 cluster consisting of 1 master node and \verb+num_nodes+ worker nodes
    \I  \verb+'cluster-name'+ parameter allows for using multiple clusters at once.
    \I  The launch can be monitored manually from the EC2 Management Console at \url{console.aws.amazon.com/ec2}
    \EIT
\I  The master node is launched 
    \BIT
    \I  with the paid Maxwell Amazon Machine Image (AMI),
    \I  as an on-demand instance, and 
    \I  as an \texttt{m1.medium} instance.
    \EIT
\I  The worker nodes are launched
    \BIT
    \I  using a public (free) AMI,
    \I  as spot request instances (in order to achieve up to 80\% savings), and
    \I  as \texttt{cg1.4xlarge} instances.
    \EIT
\EIT
Note that the use of spot requests for worker nodes may result in sudden cluster termination, 
in this case the cluster will need to be terminated and a new cluster should be started.
\BIT
\I  Naturally, a cluster can only run simulations on the nodes that it contains
    (i.e. you can't run a 5-node simulation on a 4-node cluster)
\I  However, each cluster contains a built-in queueing system that allows for 
    multiple jobs to be handled simultaneously (i.e. 10 2-node simulations on a 4-node cluster is okay)
\I  Lastly, Amazon, by default, caps the number of worker nodes allows at 10;
    to request more, go to \url{aws.amazon.com/contact-us/ec2-request/}
    and request the limit of \verb+cg1.4xlarge+ spot requests to be increased for your account.
\EIT


\newpage
% Solve diagram.
\begin{center}
\psset{unit=2cm}
\begin{pspicture}(6,4)(-5,-1.0)
\psset{gridcolor=green, subgridcolor=yellow}
    \let\psgrid\relax
    % \psgrid
    \psframe[linestyle=none,
            fillstyle=gradient,
            gradbegin=white,gradend=lightgray,
            gradmidpoint=0.5,
            gradangle=0](0,-1.0)(6,3.8)
    \psline[linestyle=dotted](0,3)(0,-0.2)
    \rput(-2.5,3.5){Matlab}
    \rput(3,3.5){Amazon EC2}

    \psverbboxtrue
    \rput(-2.5,2.5){\small \verb+>> [E, H] = maxwell.solve(...)+}
    \psverbboxfalse

    
    \psline{->, linewidth=1pt}(-0.1,2.5)(2.4,2.5) \rput[b](1.15,2.55){\small inputs}
    \psline[linewidth=1.0pt](2.4,2.1)(-3.9,2.1) \rput [t](0.6,2.0){\small solution}
    \psline{->, linewidth=1pt}(-3.9,2.1)(-3.9,2.3)
    \psframe[linestyle=solid](2.2,2.8)(3.8,1.8) \rput(3,2.3){master}

    \psline[linestyle=dashed](3,1.8)(3,1)
    \psline[linestyle=dashed](3,1.2)(1.6,1.2)
    \psline[linestyle=dashed](1.6,1.2)(1.6,1)
    \rput[br](2.9,1.3){\small 2 nodes used}
    \psframe[linestyle=solid](0.4,1.0)(2,0.0) \rput(1.2,0.5){node001}
    \psframe[linestyle=solid](2.2,1.0)(3.8,0.0) \rput(3,0.5){node002}
    \psframe[linestyle=solid](4.0,1.0)(5.6,0.0) \rput(4.8,0.5){node003}

\end{pspicture}
\end{center}
\BIT 
\I  \verb+[E, H] = maxwell.solve('cluster-name', n, ...);+
    \BIT
    \I  Solves an electromagnetic simulation on \verb+n+ nodes of cluster \verb+'cluster-name'+
    \I  Additional simulation parameters ``\verb+...+'' described in following section
    \I  Returns as solution both electric and magnetic fields
    \I  For full documentation of this function see Website
    \EIT

\I  \verb+maxwell.solve()+ proceeds as follows:
    \BIT
    \I  Transfers simulation parameters to the specified cluster
    \I  Waits for worker nodes to be provisioned for the simulation
    \I  Continues to wait as simulation is executed on worker nodes
    \I  Retrieves simulation results back to Matlab
    \EIT

\I  \verb+maxwell.solve()+ features
    \BIT
    \I  complete integration within Matlab
    \I  no need to deal with simulations files
    \I  real-time plot of simulation progress
    \EIT
% 
% \I  Although attempting to use more nodes than available in the cluster will
%     result in an error, the provided queueing system does allow for the 
%     \emph{total} number of requested nodes to exceed the number of nodes in the cluster.
\EIT



\newpage
% Solve_async diagram.
\begin{center}
\psset{unit=2cm}
\begin{pspicture}(6,4)(-5,-1.0)
\psset{gridcolor=green, subgridcolor=yellow}
    \let\psgrid\relax
    % \psgrid
    \psframe[linestyle=none,
            fillstyle=gradient,
            gradbegin=white,gradend=lightgray,
            gradmidpoint=0.5,
            gradangle=0](0,-1.0)(6,3.8)
    \psline[linestyle=dotted](0,3)(0,-0.2)
    \rput(-2.5,3.5){Matlab}
    \rput(3,3.5){Amazon EC2}

    \psverbboxtrue
    \rput[l](-5.2,2.5){\small \verb+>> cb = maxwell.solve_async(...)+}
    \rput[l](-5.2,1.5){\small \verb+>> [done, E, H] = cb()+}
    \psverbboxfalse

    
    \psline{->, linewidth=1.0pt}(-0.1,2.5)(2.4,2.5) \rput[b](1.15,2.55){\small inputs}
    \psline[linewidth=1.0pt](2.4,2.1)(-3.5,2.1) \rput [t](0.6,2.0){\small solution}
    \psline{->, linewidth=1.0pt}(-3.5,2.1)(-3.5,1.8)
    \psframe[linestyle=solid](2.2,2.8)(3.8,1.8) \rput(3,2.3){master}

    \psline[linestyle=dashed](3,1.8)(3,1)
    \psline[linestyle=dashed](3,1.2)(1.6,1.2)
    \psline[linestyle=dashed](1.6,1.2)(1.6,1)
    \rput[br](2.9,1.3){\small 2 nodes used}
    \psframe[linestyle=solid](0.4,1.0)(2,0.0) \rput(1.2,0.5){node001}
    \psframe[linestyle=solid](2.2,1.0)(3.8,0.0) \rput(3,0.5){node002}
    \psframe[linestyle=solid](4.0,1.0)(5.6,0.0) \rput(4.8,0.5){node003}

\end{pspicture}
\end{center}

\BIT 
\I  \verb+callback = maxwell.solve_async('cluster-name', n, ...);+
    \BIT
    \I  Asynchronous solve that returns a callback function instead of waiting
        for the simulation to complete 
    \I  The callback function is then used to check for solve completion and 
        to retrieve the simulation results:
        \begin{verbatim}
        [is_finished, E, H] = callback();
        \end{verbatim}
    \I  If the solve has not finished, \verb+is_finished+ is set to \verb+false+ and 
        \verb+E+ and \verb+H+ both return empty cell arrays
    \I  If the solve has indeed finished, \verb+is_finished+ is set to \verb+true+
        and \verb+E+ and \verb+H+ contain the solution fields to the problem
    \I  The additional simulation parameters ``\verb+...+'' are identical to those 
        used in \verb+maxwell.solve()+ and are detailed in the following section
    \EIT
\newpage
\I  The purpose of \verb+maxwell.solve_async()+ is to allow even single-threaded Matlab users 
    to simultaneously execute a virtually unlimited number of simulations.
        \begin{verbatim}
        % Start n simulations.
        cb{1} = maxwell.solve_async(...);
        ...
        cb{n} = maxwell.solve_async(...);

        % Wait for the simulations to complete.
        [is_finished, E, H] = cb{1}();
        ...
        CHECK THIS!!
        \end{verbatim}
%     \BIT
%     \I  \verb+maxwell.solve_async()+ proceeds by uploading the simulation to the cluster
%         and then immediately returns the function \verb+callback+.
%     \EIT
\EIT

\newpage
\BIT
\I  \verb+maxwell.terminate('cluster-name');+
    \BIT
    \I  Terminates the cluster \verb+'cluster-name'+
    \I  Note that AWS instances are charged by the hour and that partial hours
        are charged the full hour.
    \EIT
\I  Terminations (and launches) can be monitored manually from the EC2 management
    console at \url{console.aws.amazon.com/ec2}
\I  Terminations can also be executed manually from the console, 
    although it is recommended to perform the full termination from Matlab
    in order to tie up all loose ends.
\EIT

\oursection{How Maxwell solves electromagnetics}
\BIT
\I  In this section we detail the simulation parameters used by the 
    \verb+maxwell.solve()+ and \verb+maxwell.solve_async()+ functions
\I  Along with the \verb+'cluster-name'+ and \verb+n+ parameters, both
    functions support the following parameters which describe the physical simulation:
    \BIT
    \I  \verb+omega+
    \I  \verb+d_prim, d_dual, s_prim, s_dual+
    \I  \verb+mu, epsilon+
    \I  \verb+E, J+
    \I  \verb+max_iters, err_thresh+
    \EIT
    and can be understood from the master equation:
    \BEQ
    \nabla \times \mu^{-1} \nabla \times E - \omega^2 \epsilon E = -i \omega J
    \EEQ

\newpage
\I  Frequency parameter: \verb+omega+
    \BIT
    \I  The angular frequency of the simulation
    \I  Since Maxwell employs a frequency-domain solver, the exact frequency for the simulation
        can be set with this parameter
    \I  is equal to $2 \pi f$ where $f$ is in units of Hz
    \EIT
    \newpage

\begin{center}
The Yee cell \\
\psset{unit=1.3cm}
\begin{pspicture}(-2,-2)(7,7)
\psset{gridcolor=green, subgridcolor=yellow}
    \let\psgrid\relax
    
    % Origin.
    \psdot(0,0) \rput[br](-0.1,0.1){$(0,0,0)$}

    % Principle axes.
    \psline[linestyle=dashed](-1,0)(6,0) \rput(6.3,0){$x$}
    \psline[linestyle=dashed](0,-1)(0,6) \rput(0,6.3){$z$}
    \psline[linestyle=dashed](-0.8,-0.6)(4.8,3.6) \rput(5.05,3.75){$y$}

    % Intermediate axes.
    \psline[linestyle=dashed](0,3)(3,3) % Hy point.
    \psline[linestyle=dashed](3,3)(3,0)

    \psline[linestyle=dashed](0,3)(2.4,4.8) % Hx point.
    \psline[linestyle=dashed](2.4,4.8)(2.4,1.8)

    \psline[linestyle=dashed](2.4,1.8)(5.4,1.8) % Hz point.
    \psline[linestyle=dashed](5.4,1.8)(3,0)

    % E-field vectors.
    \psline{->, linewidth=2pt}(2.6,0)(3.4,0) \rput(3,-0.4){$E_x$}
    \psline{->, linewidth=2pt}(0,2.6)(0,3.4) \rput(-0.4,3){$E_z$}
    \psline{->, linewidth=2pt}(2.1,1.55)(2.7,2.05) \rput(2.05,2.05){$E_y$}

    % H-field vectors.
    \psline{->, linewidth=2pt}(2,4.8)(2.8,4.8) \rput(2.4,5.2){$H_x$}
    \psline{->, linewidth=2pt}(5.4,1.4)(5.4,2.2) \rput(5.8,1.8){$H_z$}
    \psline{->, linewidth=2pt}(2.7,2.75)(3.3,3.25) \rput(2.7,3.3){$H_y$}
\end{pspicture}
\end{center}

\newpage
\I  Spatial grid parameters: \verb+d_prim, d_dual, s_prim, s_dual+
    \BIT 
    \I  Controls the spatial grid of the simulation
    \I  \verb+d_prim+ and \verb+d_dual+ are typically used to set 
        the spatial grid within the simulation
    \I  \verb+s_prim+ and \verb+s_dual+ are typically used to set 
        the spatial grid within the absorbing layers at the edges of the simulation.
    \I  In the end, both \verb+d+ and \verb+s+ parameters are essentially interchangeable
    \I  \verb+d_prim+ and \verb+s_prim+ refer to the distances between $E_w$ in direction $w$ 
        where $w = x,y,z$ (e.g. distances between $E_x$ in the $x$ direction)
    \I  \verb+d_dual+ and \verb+s_dual+ refer to the distances in directions other than $w$ of 
        adjacent $E_w$ (e.g. distances between $E_y$ in the $x$ direction)
    \I  Lastly, Maxwell uses a periodic wrap-around grid, so that the first \verb+_prim+
        and last \verb+_dual+ entries correspond to the the wrap-around distances from 
        the last to the first Yee cells
    \EIT

\newpage
\I  Material parameters: \verb+mu, epsilon+
    \BIT
    \I  defined at $H$- and $E$- field points on the Yee grid respectively
    \EIT

\I  Field parameters: \verb+E, J+
    \BIT
    \I  \verb+E+ allows you to set the initial value of the E-field for the solver.
    \I  \verb+E = 0+ works for most cases, but a random initial field is needed
        in other cases.
    \I  \verb+J+ describes the current source in the master equation and is situated
        at the $E$-field point on the Yee grid.
    \EIT

\I  Convergence parameters: \verb+max_iters, err_thresh+
    \BIT
    \I  \verb+max_iters+ determines the maximum number of iterations before the solver terminates
    \I  \verb+err_thresh+ determines the error threshold below with the solver termiantes,
        typically set to \verb+1e-6+.
    \EIT
%  [E, H, err] = fds(omega, d_prim, d_dual, s_prim, s_dual, mu, epsilon, E, J, max_iters, err_thresh, view_progress)
\EIT


\end{document}
