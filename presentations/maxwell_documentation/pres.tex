\documentclass[landscape]{foils}
\usepackage{graphicx}
\usepackage{amsmath}
\usepackage{hyperref}
\usepackage{pstricks}
\usepackage{pst-grad}
\input defs.tex
\raggedright
\special{! TeXDict begin /landplus90{true}store end }
\renewcommand{\oursection}[1]{
\foilhead[-1.0cm]{#1}
}

\title{Maxwell: bringing cloud-powered electromagnetic simulations to Matlab}
\author{Advanced user interface tutorial}
\MyLogo{Jesse Lu, Jelena Vuckovic group, Stanford University}
\date{}

\begin{document}
\setlength{\parskip}{0cm}
\maketitle

\BIT \itemsep -1pt
\I  Quick-start
\I  Examples
\I  How Maxwell uses the cloud (EC2)
\I  How Maxwell solves electromagnetics
\EIT

\vfill

\oursection{Definitions}
\BIT
\I  What is Maxwell?
    \BIT
    \I  a Matlab toolset 
    \I  that uses Amazon's Elastic Compute Cloud (EC2)
    \I  to solve 3D frequency-domain electromagnetic simulations.
    \EIT

\I  Features:
    \BIT
    \I  Cryptographically-secure communication (https)
    \I  Full control over all simulation parameters
    \I  GPU-acceleration provided by Nvidia Tesla GPUs
    \I  Queueing system to allow for full usage of cluster
    \I  Scalable to hundreds of simultaneous simulations 
        running on hundreds of nodes.
    \EIT
\EIT

\newpage
Maxwell provides two user interfaces: advanced and other
\BIT
\I  advanced:
\I  other:
\EIT
This presentation covers the advanced interface 

\oursection{Quick-start}
\BIT \I Sign up at \EIT
\begin{verbatim}
% Download maxwell.m
>> urlwrite('m.lightlabs.co', 'maxwell.m');

% Provide AWS credentials and launch a 2-node cluster.
>> maxwell.aws_credentials('aws-access-id', 'aws-secret-key');
>> maxwell.launch('cluster-name', 2); 

% Run simulation on 1 node.
>> [E, H] = maxwell.solve('cluster-name', 1, ...); 

% Terminate cluster
>> maxwell.terminate('cluster-name');
\end{verbatim}

{Wait, what just happened?}
\BIT
\I  \verb=urlwrite()= downloaded the advanced interface for Maxwell,
\I  \verb=maxwell.aws_credentials()= provided the AWS credentials that
\I  \verb=maxwell.launch()= needed to create a cluster on EC2.
\I  \verb=maxwell.solve()= solved the electromagnetic simulation on
    the cluster and downloaded the resulting electromagnetic fields, and
\I  \verb=maxwell.terminate()= terminated the EC2 cluster.
\EIT

\oursection{Examples}

\oursection{How Maxwell uses the cloud (EC2)}
\BIT
\I  Maxwell uses your Amazon Web Services (AWS) account to
    \BIT
    \I  Create a custom Amazon EC2 cluster, and
    \I  Solve electromagnetic simulations on it;
    \EIT
    all without leaving your local Matlab environment.

 

\I  To get started, you need to
    \BIT
    \I  sign up for an AWS account,
    \I  retrieve your AWS security credentials, and
    \I  purchase the custom Maxwell Amazon Machine Image (AMI).
    \EIT
    For detailed instructions see Website.

\newpage
\I  Maxwell's advanced interface comprises of just five commands:
    \BIT
    \I  \verb+maxwell.aws_credentials()+
    \I  \verb+maxwell.launch()+
    \I  \verb+maxwell.solve()+
    \I  \verb+maxwell.solve_async()+
    \I  \verb+maxwell.terminate()+
    \EIT

\I  \verb+maxwell.aws_credentials('aws-key-id', 'aws-secret-key');+
    \BIT
    \I  Stores the security credentials linked to your AWS account locally
    \I  Security credentials are used to launch and terminate clusters
    \I  Transmitted over https and never stored on server-side
    \I  Tutorial on obtaining your credentials at Website.
    \EIT
\EIT

\newpage
% Launch diagram.
\begin{center}
\psset{unit=2cm}
\begin{pspicture}(6,4)(-5,-1.0)
\psset{gridcolor=green, subgridcolor=yellow}
    \let\psgrid\relax
    % \psgrid
    \psframe[linestyle=none,
            fillstyle=gradient,
            gradbegin=white,gradend=lightgray,
            gradmidpoint=0.5,
            gradangle=0](0,-1.0)(6,3.8)
    \psline[linestyle=dotted](0,3)(0,-0.2)
    \rput(-2.5,3.5){Matlab}
    \rput(3,3.5){Amazon EC2}

    \psverbboxtrue
    \rput(-2.5,2.5){\small \verb=>> maxwell.launch(...)=}
    \psverbboxfalse

    
    \psline[linewidth=1pt](-0.6,2.5)(3,2.5) \rput[b](1.45, 2.6){\small{launch 3-node cluster}}
    \psline{->, linewidth=1pt}(3,2.5)(3,2)

    \psframe[linestyle=dashed](2.2,1.8)(3.8,0.8) \rput(3,1.3){master}

    \psframe[linestyle=dashed](0.4,0.6)(2,-0.4) \rput(1.2,0.1){node001}
    \psframe[linestyle=dashed](2.2,0.6)(3.8,-0.4) \rput(3,0.1){node002}
    \psframe[linestyle=dashed](4.0,0.6)(5.6,-0.4) \rput(4.8,0.1){node003}
\end{pspicture}
\end{center}

\BIT
\I  \verb+maxwell.launch('cluster-name', num_nodes);+
    \BIT
    \I  Creates an EC2 cluster consisting of 1 master node and \verb+num_nodes+ worker nodes
    \I  \verb+'cluster-name'+ parameter allows for using multiple clusters at once.
    \I  The launch can be monitored manually from the EC2 Management Console at \url{console.aws.amazon.com/ec2}
    \EIT
\I  The master node is launched 
    \BIT
    \I  with the paid Maxwell Amazon Machine Image (AMI),
    \I  as an on-demand instance, and 
    \I  as an \texttt{m1.medium} instance.
    \EIT
\I  The worker nodes are launched
    \BIT
    \I  using a public (free) AMI,
    \I  as spot request instances (in order to achieve up to 80\% savings), and
    \I  as \texttt{cg1.4xlarge} instances.
    \EIT
\EIT
Note that the use of spot requests for worker nodes may result in sudden cluster termination, 
in this case the cluster will need to be terminated and a new cluster should be started.


\newpage
% Solve diagram.
\begin{center}
\psset{unit=2cm}
\begin{pspicture}(6,4)(-5,-1.0)
\psset{gridcolor=green, subgridcolor=yellow}
    \let\psgrid\relax
    % \psgrid
    \psframe[linestyle=none,
            fillstyle=gradient,
            gradbegin=white,gradend=lightgray,
            gradmidpoint=0.5,
            gradangle=0](0,-1.0)(6,3.8)
    \psline[linestyle=dotted](0,3)(0,-0.2)
    \rput(-2.5,3.5){Matlab}
    \rput(3,3.5){Amazon EC2}

    \psverbboxtrue
    \rput(-2.5,2.5){\small \verb+>> [E, H] = maxwell.solve(...)+}
    \psverbboxfalse

    
    \psline{->, linewidth=1pt}(-0.1,2.5)(2.4,2.5) \rput[b](1.15,2.55){\small inputs}
    \psline[linewidth=1.0pt](2.4,2.1)(-3.9,2.1) \rput [t](0.6,2.0){\small solution}
    \psline{->, linewidth=1pt}(-3.9,2.1)(-3.9,2.3)
    \psframe[linestyle=solid](2.2,2.8)(3.8,1.8) \rput(3,2.3){master}

    \psline[linestyle=dashed](3,1.8)(3,1)
    \psline[linestyle=dashed](3,1.2)(1.6,1.2)
    \psline[linestyle=dashed](1.6,1.2)(1.6,1)
    \rput[br](2.9,1.3){\small 2 nodes used}
    \psframe[linestyle=solid](0.4,1.0)(2,0.0) \rput(1.2,0.5){node001}
    \psframe[linestyle=solid](2.2,1.0)(3.8,0.0) \rput(3,0.5){node002}
    \psframe[linestyle=solid](4.0,1.0)(5.6,0.0) \rput(4.8,0.5){node003}

\end{pspicture}
\end{center}
\BIT 
\I  \verb+[E, H] = maxwell.solve('cluster-name', n, ...);+
    \BIT
    \I  Solves an electromagnetic simulation on \verb+n+ nodes of cluster \verb+'cluster-name'+
    \I  Additional simulation parameters ``\verb+...+'' described in following section
    \I  Returns as solution both electric and magnetic fields
    \I  For full documentation of this function see Website
    \EIT

\I  \verb+maxwell.solve()+ proceeds as follows:
    \BIT
    \I  Transfers simulation parameters to the specified cluster
    \I  Waits for worker nodes to be provisioned for the simulation
    \I  Continues to wait as simulation is executed on worker nodes
    \I  Retrieves simulation results back to Matlab
    \EIT

\I  Although attempting to use more nodes than available in the cluster will
    result in an error, the provided queueing system does allow for the 
    \emph{total} number of requested nodes to exceed the number of nodes in the cluster.
\EIT



\newpage
% Solve_async diagram.
\begin{center}
\psset{unit=2cm}
\begin{pspicture}(6,4)(-5,-1.0)
\psset{gridcolor=green, subgridcolor=yellow}
    \let\psgrid\relax
    % \psgrid
    \psframe[linestyle=none,
            fillstyle=gradient,
            gradbegin=white,gradend=lightgray,
            gradmidpoint=0.5,
            gradangle=0](0,-1.0)(6,3.8)
    \psline[linestyle=dotted](0,3)(0,-0.2)
    \rput(-2.5,3.5){Matlab}
    \rput(3,3.5){Amazon EC2}

    \psverbboxtrue
    \rput[l](-5.2,2.5){\small \verb+>> cb = maxwell.solve_async(...)+}
    \rput[l](-5.2,1.5){\small \verb+>> [done, E, H] = cb()+}
    \psverbboxfalse

    
    \psline{->, linewidth=1.0pt}(-0.1,2.5)(2.4,2.5) \rput[b](1.15,2.55){\small inputs}
    \psline[linewidth=1.0pt](2.4,2.1)(-3.5,2.1) \rput [t](0.6,2.0){\small solution}
    \psline{->, linewidth=1.0pt}(-3.5,2.1)(-3.5,1.8)
    \psframe[linestyle=solid](2.2,2.8)(3.8,1.8) \rput(3,2.3){master}

    \psline[linestyle=dashed](3,1.8)(3,1)
    \psline[linestyle=dashed](3,1.2)(1.6,1.2)
    \psline[linestyle=dashed](1.6,1.2)(1.6,1)
    \rput[br](2.9,1.3){\small 2 nodes used}
    \psframe[linestyle=solid](0.4,1.0)(2,0.0) \rput(1.2,0.5){node001}
    \psframe[linestyle=solid](2.2,1.0)(3.8,0.0) \rput(3,0.5){node002}
    \psframe[linestyle=solid](4.0,1.0)(5.6,0.0) \rput(4.8,0.5){node003}

\end{pspicture}
\end{center}

\BIT 
\I  \verb+callback = maxwell.solve_async('cluster-name', n, ...);+
    \BIT
    \I  Asynchronous solve that returns a callback function instead of waiting
        for the simulation to complete 
    \I  The callback function is then used to check for solve completion and 
        to retrieve the simulation results:
        \begin{verbatim}
        [is_finished, E, H] = callback();
        \end{verbatim}
    \I  If the solve has not finished, \verb+is_finished+ returns \verb+false+ and 
        \verb+E+ and \verb+H+ both return empty cell arrays.
    \I  The additional simulation parameters ``\verb+...+'' are identical to those 
        used in \verb+maxwell.solve()+ and are detailed in the following section
    \EIT
\I  \verb+maxwell.solve_async()+ allows even single-threaded Matlab users to simultaneously execute
    a virtually unlimited number of simulations.
    \BIT
    \I  \verb+maxwell.solve_async()+ proceeds by uploading the simulation to the cluster
        and then immediately returns the function \verb+callback+.
    \EIT
\EIT

\newpage
\BIT
\I  \verb+maxwell.terminate('cluster-name');+
    \BIT
    \I  Terminates the cluster \verb+'cluster-name'+
    \I  Note that AWS instances are charged by the hour and that partial hours
        are charged the full hour.
    \EIT
\EIT

\oursection{How Maxwell solves electromagnetics}
\BIT
\I  In this section we detail the simulation parameters used by the 
    \verb+maxwell.solve()+ and \verb+maxwell.solve_async()+ functions
\EIT


\end{document}
