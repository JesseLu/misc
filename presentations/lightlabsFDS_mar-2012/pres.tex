\documentclass[landscape]{foils}
\usepackage{graphicx}
\usepackage{amsmath}
% \usepackage[pdf]{pstricks}
\input defs.tex
\raggedright
\special{! TeXDict begin /landplus90{true}store end }
\renewcommand{\oursection}[1]{
\foilhead[-1.0cm]{#1}
}

\title{An introduction to lightlabsFDS}
\author{}
\MyLogo{}
\date{}

\begin{document}
\setlength{\parskip}{0cm}
\maketitle

\BIT \itemsep -1pt
\I  The lightlabsFDS mission
\I  Solver
\I  Interface
\I  Hardware
\I  End it
\EIT

\vfill


\oursection{The lightlabsFDS mission}
\textbf{To enable engineers and scientists to characterize optical structures 
        quickly, simply, and cost-effectively.}
\vspace{10 mm}

Innovate on three fronts to make this a reality:
\BIT
\I  Solve for electromagnetic fields in the frequency domain,
\I  Run simulations from pre-installed scientific software (Matlab), 
\I  Offload computation to centralized custom-tuned hardware.
\EIT

\oursection{Solving electromagnetics in the frequency domain}
\BIT
\I  lightlabsFDS: Frequency-Domain Solver
\I  Solves 
    \BEQ \nabla \times \mu^{-1} \nabla \times E - \omega^2 \epsilon E 
            = -i \omega J. \EEQ
    \BIT
    \I  Inputs: frequency $(\omega)$, structure $(\mu, \epsilon)$, 
            and excitation $(J)$.
    \I  Outputs: electromagnetic fields $(E, H, D, B)$.
    \EIT
\I  \emph{Many} practical advantages over existing time-domain solvers.
\EIT
\newpage
   
\oursection{Example}

% \psset{unit=1.5cm}
% \psset{gridcolor=green, subgridcolor=yellow}
% \begin{center}
% \begin{pspicture}(8,5)
%     \let\psgrid\relax
% 
%     \psframe*[linecolor=gray, fillcolor=gray, fillstyle=solid] (1,2.2)(4,2.8)
%     \psframe*[linecolor=gray, fillcolor=gray, fillstyle=solid] (3,1)(5,4)
%     \psframe*[linecolor=gray, fillcolor=gray, fillstyle=solid] (4,1.4)(7,2)
%     \psframe*[linecolor=gray, fillcolor=gray, fillstyle=solid] (4,3.0)(7,3.6)
%     \rput[B](4,4.1){Device}
%     \psline[linestyle=dashed](1.6,0.6)(1.6,4.0)
%     \rput[B](1.6,4.1){in}
%     \psline[linestyle=dashed](6.4,0.6)(6.4,4.0)
%     \rput[B](6.4,4.1){out}
% \end{pspicture}
% \end{center}
Time-domain issues include
\BIT
\I  Input: clean excitation at input requires an auxiliary simulation
\I  Device: approximations required for material dispersion
\I  Output: overlap integrals must be repeatedly calculated 
        \emph{during} the simulation
\EIT
\newpage

% \begin{center}
% \begin{pspicture}(8,5)
%     \psgrid
% 
%     \psframe*[linecolor=gray, fillcolor=gray, fillstyle=solid] (1,2.2)(4,2.8)
%     \psframe*[linecolor=gray, fillcolor=gray, fillstyle=solid] (3,1)(5,4)
%     \psframe*[linecolor=gray, fillcolor=gray, fillstyle=solid] (4,1.4)(7,2)
%     \psframe*[linecolor=gray, fillcolor=gray, fillstyle=solid] (4,3.0)(7,3.6)
%     \rput[B](4,4.1){Device}
%     \psline[linestyle=dashed](1.6,0.6)(1.6,4.0)
%     \rput[B](1.6,4.1){in}
%     \psline[linestyle=dashed](6.4,0.6)(6.4,4.0)
%     \rput[B](6.4,4.1){out}
% \end{pspicture}
% \end{center}
\BIT
\I  Fundamental problem: trying to use a time-domain solver as a 
        frequency-domain solver.
\I  Additionally, no method to measure simulation error!
\EIT
\newpage

\BIT
\I  Allow direct access to frequency-domain data.
\I  Take care of input and output fields outside of the simulation.
\I  Material dispersion explicitly defined.
\I  Simulation error well defined.
\EIT
\newpage

Picture from paper here.
\BIT
\I  Frequency-domain solver made possible by correct choice of PML and 
        linear algebra algorithm.
\I  See: Wonseok Shin, Shanhui Fan, 
            ``Choice of the perfectly matched layer boundary condition for 
            frequency-domain Maxwell's equations solvers'',
            Journal of Computational Physics (January 2012).
\EIT


\oursection{Interface}
\textbf{We want engineers and scientists to absolutely love using FDS
        and, at the same time, to forget about it because it just works.}
\BIT
\I  No installation, just start up Matlab and
\begin{verbatim}
>>> [E, H] = fds(omega, epsilon, J); % Done.
\end{verbatim}
\I  Helper functions to
    \BIT
    \I  construct the optical structures $(\epsilon, \mu)$,
    \I  define the input excitations $(J)$, and
    \I  analyze and visualize the output fields $(E,H,D,B)$
    \EIT
    are all included and open-sourced.
\I  Additionally, lots of examples and tutorials.
\EIT

\oursection{Hardware}
Hub and spokes picture here.
\BIT
\I  Centralized, shared, custom-tuned server able to deliver the performance
        of a large cluster
\I  Performance achieved via heavily optimized GPU code
\I  Multiple servers can be clustered with nearly $100\%$ computational
        efficiency.
\EIT

\oursection{Current status}
\BIT
\I  Algorithm: Implemented on GPUs, still optimizing (Jesse)
\I  Interface: (Wonseok)
\I  Hardware: Prototype ordered and being built (Jesse)
\EIT
\end{document}
