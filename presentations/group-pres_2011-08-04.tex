\documentclass[dvips,landscape]{foils}
\usepackage{graphicx,psfrag}
\usepackage{amsmath}
\input defs.tex
\raggedright
\special{! TeXDict begin /landplus90{true}store end }
\renewcommand{\oursection}[1]{
\foilhead[-1.0cm]{#1}
}

\title{Objective-first optimization for nanophotonics}
\author{}
\MyLogo{Jesse Lu, Jelena Vuckovic group, Stanford University}
\date{}

\begin{document}
\setlength{\parskip}{0cm}
\maketitle

\BIT \itemsep -1pt
\item adjoint method
\item objective-first approach
\item field setup: boundary-value problem
\item structure setup: level-set formulation
\item example
\item ongoing and future work
\EIT

\vfill

\oursection{Adjoint method}
Typical formulation of a structural optimization problem:
\BEAS
\mbox{decrease} & f(x) \\
\subjectto & g(x,p) = 0
\EEAS
\BIT
\item $f(x): \comps^n \to \reals$ is the \emph{design objective}
\item $g(x,p): \comps^n \times \reals^n \to \comps^n$ is the \emph{governing physics}
\item $x \in \comps^n$ is the field 
\item $p \in \reals^n$ is the structure 
\item $x$ is the dependant variable, $p$ is the independant variable
\EIT
\newpage

\BIT
\item problem is non-convex, so optimize using first-order approximations
\BEAS
f(x_0+dx) \approx f(x_0) + \pf{f}{x} dx \\
g(x_0+dx,p_0+dp) \approx g(x_0,p_0) + \pf{g}{x} dx + \pf{g}{p} dp
\EEAS
\item assuming that $g(x_0,p_0) = 0$, the equality constraint is satisfied via
    \BEQ  dx = -\left(\pf{g}{x}\right)^{-1} \pf{g}{p} dp \EEQ
\EIT
\newpage

\BIT
\item now we can decrease $f(x_0+dx)$,
    \begin{align} 
    f(x_0+dx) & \approx f(x_0) + \pf{f}{x} dx \\
        & \approx f(x_0) - \pf{f}{x} \left(\pf{g}{x}\right)^{-1} \pf{g}{p} dp \\
        & \approx f(x_0) + \pf{f}{p} dp
    \end{align}
    by choosing $dp$ in the direction of $dp \propto -\pf{f}{p}$
\item computing $\pf{f}{p}$ can be reduced to a single field solve (\ie solving $g(x,p)$ for $x$ given $p$)
\EIT


\end{document}
