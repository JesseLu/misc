\section{Optical mimic design}
We now apply our objective-first strategy to the design of optical mimics.

We define an optical mimic to be a linear nanophotonic device which 
    mimics the output field of another device.
In this sense optical mimics are the anti-cloaks;
    whereas cloaks strive to make an object's electromagnetic presence vanish,
    mimics strive to implement an object's presence without that 
    object actually being there.

As such, the design of optical mimics provides a tantalyzing approach
    to the realization of practical metamaterial devices.
That is to say, if one can reliably produce practical optical mimics,
    then producing metamaterials (which are often based on fictitious materials)
    can be accomplished by simply producing an optical mimic of that material.

In a more general sense, 
    designing optical mimics is really just a recasting of the thrust of 
    the objective-first design strategy in its purest form:
    the design of a nanophotonic device based purely 
    on the electromagnetic fields one wishes to have it produce.
As such, devices which perform well-known optical functions
    (e.g. focusing, lithography) can also be designed.

\subsection{Application of the objective-first strategy}
The objective-first design of optical mimics proceeds in virtually
    an identical way to the design of optical cloaks,
    the only difference being that the output modes are 
    specifically chosen to be those which produce the desired function.
For most of the examples provided, the input illumination is still an incident plane wave.

Lastly, instead of measuring efficiency, 
    we measure the relative error of the simulated field
    against that of a perfect target field
    at a relevant plane of some distance away from the device.
The location of this plane is identified as a dotted line in the subsequent figures.

\subsection{Plasmonic cylinder mimic}
Our first mimic is simply to mimic the plasmonic cylinder 
    which we cloaked in the previous section.
\myfig{mimic/m1}{Plasmonic cylinder mimic (see \fR{cloak/c6} for the original object). 
                Error: 8.1\%,
                device footprint: $40 \times 120$ grid points,
                wavelength: 42 grid points.}

\FR{mimic/m1} shows the result of the design.
The final structure is shown in the upper right plot,
    while the ideal field and the simulated field
    are shown in the middle and bottom plots.
Note that the ideal field is cut off to emphasize 
    the fields to the right of the device (the output fields).
Also, the magnitude of the fields are compared at the 
    dotted black line at which point the relative error
    is also calculated.
For this simple, initial mimic, the simulated field is
    quite closely imitates that produced by a single plasmonic cylinder.

\subsection{Diffraction-limited lens mimic}
We now design a mimic for a typical diffraction-limited lens.
In this case, the object which we wish to mimic does not require simulation
    since the fields of a lens can be readily computed.
For the three figures below, 
    the computed ideal fields are shown as the target fields.

\FR{mimic/m3} shows the mimic of a lens with a moderate focus spot. 
In such a lens, the focusing action is gradual and easily discernable
    by eye.

\myfig{mimic/m3}{Full-width-half-max at focus: 1.5 $\lambda$,
                focus depth: 100 grid points.
                Error: 12.0\%,
                device footprint: $40 \times 120$ grid points (1.6 $\lambda$ thick),
                wavelength: 25 grid points.}

In constrast, \fR{mimic/m4} and \fR{mimic/m5} are both mimics of
    a lens with a smaller half-wavelength spot size.
Such a lens is much harder to design, 
    because of the high-frequency spatial components involved;
    and yet, we show that an objective-first strategy can 
    produce successful designs with both smaller and larger focus depths.
\myfig{mimic/m4}{Full-width-half-max at focus: 0.5 $\lambda$,
                focus depth: 50 grid points.
                Error: 5.6\%,
                device footprint: $40 \times 120$ grid points (1.6 $\lambda$ thick),
                wavelength: 25 grid points.}
\myfig{mimic/m5}{Full-width-half-max at focus: 0.5 $\lambda$,
                focus depth: 150 grid points.
                Error: 1.4\%,
                device footprint: $40 \times 120$ grid points (1.6 $\lambda$ thick),
                wavelength: 25 grid points.}
\clearpage

\subsection{Sub-diffraction lens mimic}
Our method is now employed to mimic the effect of a sub-diffraction lens.
Since such a lens can be created using a negative-index material
    this mimic can be viewed as an imitation of a negative-index material,
    in that the following device recreates the sub-diffraction target-field 
    at the output plane (dotted line)
    when illuminated by the same target field at the input of the device.
In other words,
    this device is an image-specific sub-diffraction imager,
    which is another way of saying that it is a single-mode imager.
\myfig{mimic/m2}{Sub-diffraction lens mimic.
                The target field has a full-width half-maximum of 0.14 $\lambda$.
                Error: 28.6\%,
                device footprint: $60 \times 120$ grid points (1.43 $\lambda$ thick),
                wavelength: 42 grid points.}

As \fR{mimic/m2} shows,
   we are able to recreate the target field at the output.
Note that the target field is created simply by placing 
    the imaging field at the output plane.
Also note that, as expected, 
    the output field decays very quickly since,
    for such a deeply subwavelength field,
    it is composed primarily of evanescently decaying modes.

\subsection{Sub-diffraction optical mask}
Lastly, we extend the idea of a sub-diffraction lens mimic
    one step further and
    design a sub-diffraction optical mask.
Such a device takes a plane wave as its input and
    produces a sub-diffraction image at its output plane.
Of course, akin to its lens counterpart,
    this output plane must lie within the near-field 
    of the device (specifically, two computational cells away)
    because of its sub-wavelength nature.
\myfig{mimic/m6}{Sub-diffraction optical mask.
                The three central peaks in the target field are each
                separated by 0.28 $\lambda$.
                Error: 19.8\%,
                device footprint: $40 \times 120$ grid points,
                wavelength: 25 grid points.}

\FR{mimic/m6} shows the design of a simple mask which 
    successfully produces three peaks at its output.

