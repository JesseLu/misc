
\section{The objective-first design problem}
We now build off of the field-solver and the structure-solver,
    as previously outlined,
    by formulating the design problem
    and outlining the objective-first strategy.

\subsection{Design objectives}
A design objective, $f(x)$, is simply defined as
    a function we wish to be minimal 
    for the design to be produced.

For instance, in the design of a device
    which must transmit efficiently into a particular mode,
    we could choose $f(x)$ to be the negative power transmitted into that mode.
Or, if the device was to be a low-loss resonator,
    we could choose $f(x)$ to be the amount of power leaking 
    out of the device.

In general, there are multiple choices of $f(x)$
    which can be used to describe the same objective.
For example, $f(x)$ for a transmissive device 
    may not only be the negative power transmitted into the desired output mode,
    but it could also be the amount of power lost to other modes,
    or even the error in the field values at the output port
    relative to the field values needed for perfect transmission.
These design objectives are equivalent in the sense that, if minimized, 
    all would produce structures with good performance.
At the same time, we must consider that the computational cost and complexity
    of using one $f(x)$ over another may indeed vary greatly.

\subsection{Convexity}
Before formulating the design problem,
    we would like to inject a note regarding the complexity of various 
    optimization problems.

Specifically, we want to introduce the notion of \emph{convexity} %TODO: ref.
    and to simply note the difference between problems that
    are convex and those which are not.
The difference is simply this:
    convex problems have a single optimum point
    (only one local optimum, which is therefore the global optimum)
    which we can reliably find using existing numerical software,
    whereas non-convex problems typically have multiple optima 
    and are thus much more difficult to reliably solve.

That a convex problem can be reliably solved, in this case, 
    means that regardless of the starting guess,
    convex optimization software will 
    always arrive at the globally optimal solution
    and will be able to numerically prove global optimality as well.
Thus, formulating a design problem in terms of convex optimization problems
    virtually eliminates any ideas of chance or randomness.

For the examples presented in this chapter,
    we use CVX, a convex optimization software written for Matlab. % Ref.


\subsection{Typical design formulation}
The typical, and most straightforward formulation
    of the design problem is
\BA \minimize{x,p} f(x) \\
    \subto A(p)x - b(p) = 0, \label{eq:typ:eqcon} \EA
    which states that we would like to vary $x$ and $p$ simultaneously
    in order to decrease $f(x)$
    while always satisfying physics (the electromagnetic wave equation).

Such a formulation, if solved using a steepest-descent method,
    is the well-known adjoint optimization method. % Ref.

\subsection{Objective-first design formulation}
In contrast, the objective-first formulation is 
\BA \minimize{x,p} \| A(p) x - b(p) \|^2 \\
    \subto f(x) = f_\text{ideal}, \EA
    where the roles of the wave equation and the design objective are switched.


This is a bi-convex problem, which we solve using an alternating directions method.

\subsection{Field sub-problem}
\BA \minimize{x} \| A(p) x - b(p) \|^2 \\
    \subto f(x) = f_\text{ideal} \EA

\subsection{Structure sub-problem}
\BA \minimize{p} \| B(x) p - d(x) \|^2 \\
    \subto 0 \le p \le 1 \EA

