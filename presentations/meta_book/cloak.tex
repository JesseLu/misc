\section{Optical cloak design}
In the previous section,
    we showed that couplers between virtually any two waveguide modes
    could be constructed using the objective-first design method,
    and based on the generality of the method
    one can guess that it may also be
    able to generate designs for any linear nanophotonic device.

Now, we extend the applicability of our method
    to the design of metamaterial devices which operate in free-space.
In particular,
    we adapt the waveguide coupler algorithm to the 
    to the design of optical cloaks.

\subsection{Application of the objective-first strategy}
Adapting the method used in \sR{wg} to the design of optical cloaks
    really only requires one to change the simulation environment
    to allow for free-space modes.
This is accomplished by modifying the upper and lower boundaries
    of the simulation domain from absorbing boundary conditions
    to periodic boundary conditions,
    which allows for plane-wave modes to propagate without loss
    until reaching the left or right boundaries,
    where absorbing boundary conditions are still maintained.

In terms of the design objective, 
    we allow the device to span the entire height of the simulation domain,
    and thus consider only the leftmost and rightmost planes as boundary values.
Specifically, for this section the input and output modes are plane waves
    with normal incidence, 
    as can be expected for good cloaking devices.
The achieved results all yield high efficiency,
    although we note that the cloaking effect is only measured
    for a specific input mode.
That is to say, just as the waveguide couplers previously designed
    were single-mode devices,
    so the cloaks designed in this section are also ``single-mode'' cloaks.

An additional modification, as compared to \sR{wg}, is that
    we now disallow the structure to be modified in certain areas
    which, naturally, contain the object to be cloaked.

With these simple changes we continue to solve \ER{ob1}
    with the alternating directions method
    in order to now design optical cloaks
    instead of waveguide couplers.
Once again, as in \sR{wg}, each design is run for 400 iterations
    with a uniform initial value of $p = 1/9$ for the structure
    (where the structure is allowed to vary),
    and the range of $p$ is limited to $1/12.25 \le p \le 1$.

\subsection{Anti-reflection coating}
As a first example,
    we attempt to design the simplest and most elementary ``cloaking'' device available,
    which, we argue, is a simple anti-reflection coating;
    in which case the object to be cloaked is nothing more than
    the interface between two dielectric materials.
In this case we use the interface between air and silicon, as shown in \fR{cloak/c1}
\myfig{cloak/c1}{Anti-reflection coating.
                Efficiency: 99.99\%,
                footprint: $60 \times 100$ grid points,
                wavelength: 63 grid points.}

Unsurprisingly for such a simple case, 
    we achieve a very high efficiency device.
Note also that the efficiency of the device can be deduced by eye,
    based on the absence of reflections or standing waves 
    in bottom two plots of \fR{cloak/c1}.

\subsection{Wrap-around cloak}
Next, we design a cloak for a plasmonic cylinder,
    which is quite effective at scattering light
    as can be seen from \fR{cloak/c6}.
\myfig{cloak/c6}{Plasmonic cylinder to be cloaked. 
                68.5\% of light is diverted away from the desired output mode.}

In designing the wrap-around cloak,
    we allow the structure to vary at all points within the design area
    except in the immediate vicinity of the plasmonic cylinder.
Application of the objective-first strategy results
    in an efficient device as seen in \fR{cloak/c2}.
\myfig{cloak/c2}{Wrap-around cloak.
                Efficiency: 99.99\%,
                footprint: $60 \times 100$ grid points,
                wavelength: 42 grid points.}

Note that our cloak employs only isotropic, non-magnetic materials,
    and at the same time it is specific to a particular input
    and to a particular object.

\subsection{Open-channel cloak}
With a simple modification, from the previous section,
    we can design a cloak which features an open channel
    to the exterior electromagnetic environment.
This simple modification is forcing an air tunnel
    to be opened which connects the cylinder to the outside world
    both toward its front and back.
\myfig{cloak/c4}{Open-channel cloak.
                Efficiency: 99.8\%,
                footprint: $60 \times 100$ grid points,
                wavelength: 42 grid points.}

Such a design is still very efficient
    and exhibits the usefulness of the objective-first strategy
    in cases where other methods, such as transformation optics,
    may not be able to be applied.

\subsection{Channeling cloak}
Our last cloaking example replaces the plasmonic cylinder 
    with a thin metallic wall in which is etched a sub-wavelength channel.
Such a metallic wall is very effective at blocking incoming light
    (as can be seen from \fR{cloak/c7}) 
    because of its large negative permittivity ($\epsilon = -20$),
    meaning that any cloaking device would be forced to channel
    all the input light into a very small aperture
    and then to flatten that light out into a plane wave again.
\myfig{cloak/c7}{Metallic wall with sub-wavelength channel to be cloaked.
                99.9\% of the light is blocked from the desired output plane-wave.}

Once again, our method is still able to produce a very efficient design,
    as shown in \fR{cloak/c5}.
\myfig{cloak/c5}{Channeling cloak.
                Efficiency: 99.9\%,
                footprint: $60 \times 100$ grid points,
                wavelength: 42 grid points.}

