\section{Optical cloak design}
In the previous section,
    we showed that couplers between virtually any two waveguide modes
    could be constructed using the objective-first design method,
    and based on the generality of the method
    one can guess that it may also be
    able to generate designs for any linear nanophotonic device.

Now, we extend the applicability of our method
    to the design of metamaterial devices which operate in free-space.
In particular,
    we adapt the waveguide coupler algorithm to the 
    to the design of optical cloaks.

\subsection{Application of the objective-first strategy}
Adapting the method used in \sR{wg} to the design of optical cloaks
    really only requires one to change the simulation environment
    to allow for free-space modes.
This is accomplished by modifying the upper and lower boundaries
    of the simulation domain from absorbing boundary conditions
    to periodic boundary conditions,
    which allows for plane-wave modes to propagate without loss
    until reaching the left or right boundaries,
    where absorbing boundary conditions are still maintained.

In terms of the design objective, 
    we allow the device to span the entire height of the simulation domain,
    and thus consider only the leftmost and rightmost planes as boundary values.

An additional modification, as compared to \sR{wg}, is that
    we now disallow the structure to be modified in certain areas
    which, naturally, contain the object to be cloaked.

With these simple changes we continue to solve \ER{ob1}
    with the alternating directions method
    in order to now design optical cloaks
    instead of waveguide couplers.
Once again, as in \sR{wg}, each design is run for 400 iterations
    with a uniform initial value of $p = 1/9$ for the structure
    (where the structure is allowed to vary).

Lastly, 

\myfig{cloak/c6}{test}
\myfig{cloak/c1}{test}
\myfig{cloak/c2}{test}
\myfig{cloak/c3}{test}
\myfig{cloak/c4}{test}
\myfig{cloak/c5}{test}
\myfig{cloak/c7}{test}

