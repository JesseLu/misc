
\section{Extending the method}
The objective-first method, as applied in the examples in this chapter,
    really only represent a small foray into the area of nanophotonic design.
Several key extensions to what is presented here are needed to
    fully address real-world nanophotonic design challenges.

\subsection{Three-dimensional design}
The first of these is the need to design fully three-dimensional structures.
Doing so provides no inherent difficulties aside from the matrices in
    \ER{Ab} becoming very large.
This is not insurmountable as electromagnetic simulation software for 
    three-dimensional nanophotonic structures already exists.

In fact, for certain choices of the design objective 
    (i.e. those of low-rank) 
    \ER{Fsub} can be efficiently solved by a small number of calls
    to unmodified simulation software. 
Of course, for general design objectives, such software will need 
    to be modified in order to solve \ER{Fsub}.

On the other hand, specialized software to solve \ER{Bd} in 
    any number of dimensions does not exist,
    although this was not a problem in two dimensions since generic 
    linear algebra solvers are more than accurate.
In three dimensions, 
    the large size of matrix $B(x)$ can be greatly compressed by considering
    only fabrication processes which modify a structure in-plane.
In this way, the degrees of freedom in $p$ can be greatly reduced and the original
    methods used in this chapter can still be applied.
This work-around is especially viable since in-plane structures are 
    of most interest from a practical standpoint.
    
\subsection{Multi-mode}
A second necessary extension is to be able to consider the multiple fields
    that a structure produces in response to input fields of differing frequency
    and spatial distribution.
Such an extension is straightforward in the objective-first formulation
    and results in the following modified problem statement,
\BA \minimize{x_i,p} \sum_i \| A(p) x_i - b(p) \|^2 \notag \\
    \subto f(x_i) = f_{i,\text{ideal}} \quad \text{i = 1, \ldots, n} \\
        & p_0 \le p \le p_1, \notag \EA
    which can be separated into field and structure sub-problems 
    as in the single-mode formulation.
In the multi-mode case, this results in one structure sub-problem 
    and $n$ field sub-problems.
Interestingly, the $n$ field sub-problems lend themselves 
    naturally to parallelization
    since they can be solved independently,
    leading to the possibility that a multi-mode design completing in
    roughly the same time as a single-mode design.

\subsection{Binary structure}
Another necessary extension of our method is to force 
    the values of $p$ to be discrete.
This is not trivial since a naive restatement of \ER{ob1}
    which includes such a constraint,
\BA \minimize{x,p} \| A(p) x - b(p) \|^2 \notag \\
    \subto f(x) = f_\text{ideal} \\
        & p \in \{p_0, p_1\}, \notag \EA
    results in a very difficult combinatorial problem.

Tractable approaches include penalizing intermediate values of $p$\cite{simp}
    or even transferring to a level-set method\cite{miller}
    where the distinction between materials is explicit.

\subsection{Robustness}
Lastly, the design of structures which are robust to both
    fabrication imperfections and fluctuations in environmental parameters
    is also a necessity for practical real-world devices.

It seems likely in this case that a heurestic approach may
    be most successful in this case, rather than to tackle 
    the problem head-on.
For instance, to account for fluctuating material parameters
    induced by temperature changes
    one may design a device to operate for a larger-than-necessary 
    frequency range.

