% Some custom commands.
\newcommand{\BE}{\begin{equation}}
\newcommand{\EE}{\end{equation}}
\newcommand{\BA}{\begin{eqnarray}}
\newcommand{\EA}{\end{eqnarray}}
\newcommand{\curl}{\nabla\times}
\newcommand{\minimize}[1]{\JLUminimize_{#1}\;&}
\newcommand{\subto}{\text{subject to}\;&}

% Custom figures.
% smallfig/ directory is for lo-fi figures,
% fig/ directory is for full resolution.
\newcommand{\myfig}[2]{\begin{figure}[!h]\includegraphics[width=\textwidth]{fig/#1.jpg}\caption{#2}\label{#1}\end{figure}}
\newcommand{\BI}{\begin{itemize}\item}
\renewcommand{\I}{\item}
\newcommand{\EI}{\end{itemize}}
\newcommand{\ER}[1]{\eqref{eq:#1}}
\chapter{Objective-First Nanophotonic Design}
\label{intro}

\abstract{The abstract for the book.}

In this chapter, we introduce an ``objective-first'' strategy 
    for designing nanophotonic devices.

General description:
\BI Only uses desired fields.
\I  Can attempt to design any linear device.
\I  Allows for non-physical fields. \EI

This strategy is unique in that it
    asks the user only for what electromagnetic field
    the device should produce,
    and then attempts to generate the design
    without further user intervention.

\section{The electromagnetic wave equation}
In this section, we outline the wave equation
    that is central to the application of our method,
    with the end-result being to show that it is 
    separably linear (bi-linear) in the field and structure variables.
We do this by first formulating this wave equation in the language of physics,
    and then discretizing it in order to achieve numerical solutions.
We then show how one can not only obtain the solution for the field,
    but how one can obtain the solution for the structure using
    simple, standard numerical tools.

\subsection{Physics formulation}
First, let's derive our wave equation,
    starting with the differential form of Maxwell's equations, 
\BA \curl E = - \mu_0 \frac{\partial H}{\partial t} \\
    \curl H = J + \epsilon \frac{\partial E}{\partial t}, \EA
    where $E$, $H$, and $J$ are 
    the electric, magnetic and electric current % TODO: Check this.
    vector fields, respectively,
    $\epsilon$ is the permittivity
    and $\mu_0$ is the permeability, which we assume to be 
    that of vacuum everywhere.


We use the time dependence $\exp(-i \omega t)$, 
    where $\omega$ is the angular frequency,
    these become
\BA \curl E = - i \mu_0 \omega H \\
    \curl H = J + i \epsilon \omega E, \label{eq:H2E} \EA
    which we can combine to form our (time-harmonic) wave equation,
\BE \curl \epsilon^{-1} \curl H - \mu_0 \omega^2 H = \curl \epsilon^{-1} J. 
    \label{eq:wave} \EE

In this chapter, we are going to only consider the two-dimensional form
    of this equation; and specifically 
    the two-dimensional transverse magnetic (TM) mode. % Ref.
In this case \ER{wave} is simplified not only in that
    the vector fields are two-dimensional, 
    but that only the $z$-component of $H$ is non-zero.

Nevertheless, \ER{wave}, in a single equation,
    represents all the physics which we take into account in this chapter.

\subsection{Numerical formulation}
Now, on top of the analytical formulation of the wave equation \ER{wave}
    we will now add a numerical, or discretized, formulation.
This will be needed in order to solve for arbitrary structures
    for which there are not analytical solutions.

The salient step in order to do so
    is to the use of the Yee grid, % TODO: Reference!
    which allows us to easily define the curl $(\curl)$
    operators in \ER{wave}. % TODO: ref.
Since both the individual curl operators and the equation as a whole
    is linear in $H$, it naturally follows to formulate \ER{wave},
    with a change of variables, as
\BE A(p)x = b(p), \label{eq:Ab} \EE 
    where $H \to x$, $\epsilon^{-1} \to p$; 
    and where
\BE A(p) = \curl \epsilon^{-1} \curl - \mu_0 \omega^2 \EE 
    and
\BE b(p) = \curl \epsilon^{-1} J. \EE
Note that our use of $A(p)$ and $b(p)$ instead of $A$ and $b$
    simply serves to clarify the dependence
    of both $A$ and $b$ to $p$.

Apart from using the Yee grid, the only other salient implementation detail
    is the use of
    stretched-coordinate perfectly matched layers %ref
    where necessary, in order to prevent unwanted reflections
    at the boundaries of the simulation domain.
The effect of such layers is to modify the curl operators,
    although their linear property is still maintained.

\subsection{Solving for $H$}
With our numerical formulation, we can now solve for the $H$-field
    (the $E$-field can be computed from the $H$-field using \ER{H2E})
    by applying general linear algebra solvers to \ER{Ab}.
Recall that since we have chosen a time-harmonic formulation,
    solving for $x$ in \ER{Ab} is actually performing what is simply known as
    a time-harmonic or a finite-difference frequency-domain (FDFD) simulation.
Furthermore, since we have limited ourselves to the two-dimensional case,
    \ER{Ab} is easily solved using the standard sparse solver
    included in Matlab on a single desktop computer.

We call the routine that solves for $x$ in \ER{Ab} given $p$ a field-solver,
    or a simulator.
    


\subsection{Solving for $\epsilon^{-1}$}
Now, the next step,
    after having built a field-solver or simulator
    (finds $x$ given $p$) for our wave equation,
    is to build a structure-solver for it.
In other words, we need to be able to solve for $p$ given $x$.

To do so, we return to \ER{wave}
    and remark that 
    $\epsilon^{-1} (\curl H) = (\curl H) \epsilon^{-1}$ and
    $\epsilon^{-1} J = J \epsilon^{-1}$ 
    since scalar multiplication is communicative.
This allows us to rearrange \ER{wave} as
\BE \curl (\curl H) \epsilon^{-1} - \curl J \epsilon^{-1}  = \mu_0 \omega^2 H  \EE
which we now write as 
\BE B(x)p = d(x), \label{eq:Bd} \EE 
    where
\BE B(x) = \curl (\curl H) - \curl J\EE
    and 
\BE d(x)  = \mu_0 \omega^2 H.  \EE

With this extremely simple trick,
    we have shown that we can seemingly solve for $p$ given $x$
    with approximately the same ease as solving for $x$ given $p$!
We see this because the dimensions and complexity of $B(x)$ is basically
    equivalent to that of $A(p)$,
    and this implies that the same simple tools used in our field-solver
    should be applicable to solving \ER{Bd}.
This is indeed what we find, although the later addition of constraints on $p$
    will require the use of more powerful (but just as dependable) numerical tools.

\subsection{Bi-linearity of the wave equation}
Although additional mathematical machinery must still be added
    in order to get a useful design tool,
    we have really shown so far is that the wave equation is 
    separately linear or \emph{bi-linear} in $x$ and $p$.
Namely that,
\BE A(p)x-b(p) = B(x)p - d(x). \label{eq:bilinear} \EE
In other words, fixing $p$ makes solving the wave equation for $x$
    a linear problem, and vice versa.
Note that the joint problem,
    where both $x$ and $p$ are allowed to vary,
    is not linear.

The bi-linearity of the wave equation
    is \emph{absolutely fundamental} in our objective-first strategy
    because it relies on the fact
    that, although simultaneously solving for $x$ and $p$ is very difficult,
    we already know how to solve linear systems ($x$ and $p$ separately) well.
In fact, it is this very property which forms the natural division of labor
    which the objective-first method, which we now present, exploits.


\section{The objective-first design problem}
We now describe the remaining machinery used in the objective-first method,
    in addition to the field-solver and the structure-solver,
    as previously outlined.
Specifically, we introduce the idea of a design objective and a physics residual,
    and we reference the mathematical notion of convexity
    in order to motivate the need to divide the 
    objective-first problem into two separately convex sub-problems.

\subsection{Design objectives} \label{sec:desobj}
A design objective, $f(x)$, is simply defined as
    a function we wish to be minimal 
    for the design to be produced.

For instance, in the design of a device
    which must transmit efficiently into a particular mode,
    we could choose $f(x)$ to be the negative power flow into that mode.
Or, if the device was to be a low-loss resonator,
    we could choose $f(x)$ to be the amount of power leaking 
    out of the device.

In general, there are multiple choices of $f(x)$
    which can be used to describe the same objective.
For example, $f(x)$ for a transmissive device 
    may not only be the negative power transmitted into the desired output mode,
    but it could also be the amount of power lost to other modes,
    or even the error in the field values at the output port
    relative to the field values needed for perfect transmission.
These design objectives are equivalent in the sense that, if minimized, 
    all would produce structures with good performance.
At the same time, we must consider that the computational cost and complexity
    of using one $f(x)$ over another may indeed vary greatly.

\subsection{Convexity}
Before formulating the design problem,
    we would like to add a note regarding the complexity of various 
    optimization problems.

Specifically, we want to introduce the notion of \emph{convexity}\cite{boyd}
    and to note the difference between problems that
    are convex and those which are not.
The difference is simply this:
    convex problems have a single optimum point
    (only one local optimum, which is therefore the global optimum)
    which we can reliably find using existing numerical software,
    whereas non-convex problems typically have multiple optima 
    and are thus much more difficult to reliably solve.

That a convex problem can be reliably solved, in this case, 
    means that regardless of the starting guess,
    convex optimization software will 
    always arrive at the globally optimal solution
    and will be able to numerically prove global optimality as well.
Thus, the advantage in formulating a design problem 
    in terms of convex optimization problems
    is to eliminate both the need to circumvent local optima and 
    any notion of randomness.

On a practical level, there exist mature convex optimization software packages
    among which is CVX, a convex optimization package written for Matlab\cite{cvx}, 
    which we use for the examples in this chapter.


\subsection{Typical design formulation}
We now examine the typical, and most straightforward formulation
    of the design problem,
    in order to relate and contrast it to the objective-first formulation.
The design problem for a physical structure is typically formulated as
\BA \minimize{x,p} f(x) \label{eq:typform} \\
    \subto A(p)x - b(p) = 0, \notag \EA
    which states that we would like to vary $x$ and $p$ simultaneously
    in order to decrease $f(x)$
    while always satisfying physics (e.g. the electromagnetic wave equation).

Since solving \ER{typform} is quite difficult in the general sense
    (simultaneously varying $x$ and $p$ makes the problem non-convex),
    traditional approaches have relied on either brute-force parameter search,
    or a gradient-descent method utilizing first-order derivatives.
In the gradient-descent case, solving \ER{typform} results in the
    well-known adjoint optimization method\cite{miller}. 

\subsection{Objective-first design formulation}
In contrast with the typical formulation, the objective-first formulation 
    simply switches the roles of 
    the wave equation and the design objective with one another,
\BA \minimize{x,p} \| A(p) x - b(p) \|^2 \label{eq:ob1:1} \\
    \subto f(x) = f_\text{ideal}. \label{eq:ob1:2} \EA
Although such a switch may seem trivial,
    and even silly at first,
    we show that it fundamentally changes the nature of the design problem
    and actually gains us advantages in our efforts at finding a solution.

This first fundamental change, as seen from \ER{ob1:1},
    is that we allow for non-zero residual in the electromagnetic wave equation.
This literally means that we allow for \emph{non-physical} $x$ and $p$,
    since $A(p) x - b(p) \ne 0$ is permissible.
And since $A(p) x - b(p)$ can now be a non-zero entity, 
    we choose to call it the \emph{physics residual}.  
The second fundamental change
    is that we always force the device to exhibit ideal performance,
    as seen from \ER{ob1:2}.
This, of course, ties in very closely with \ER{ob1:1} since
    ideal performance is usually not obtainable unless one
    allows for some measure of error in the underlying physics
    (non-zero physics residual).
As such, our strategy will be to vary $x$ and $p$
    in order to decrease the physics residual \ER{ob1:1} to zero,
    while always maintaining ideal performance.

The primary advantage in the objective-first formulation is that,
    although the full problem is still non-convex,
    it allows us to form two convex sub-problems, as we outline below.
In contrast to an adjoint method approach,
    in doing we can still access information regarding
    second-order derivatives, which greatly speeds up finding a solution.
An additional advantage is that our insistence
    that ideal performance be always attained
    provides a mechanism
    which can potentially ``override'' local optima 
    in the optimization process.

To this end we have found that such a strategy
    actually allows us to design very unintuitive devices
    which exhibit very good performance,
    even when starting from completely non-functional initial guesses.
Furthermore, we have found this to be true
    even true when the physics residual is never brought to exactly zero.

% From a numerical standpoint, 
%     although the objective-first formulation is still non-convex 
%     in its original form,
%     the bi-linearity of the physics residual term allows us to 
%     naturally break the original problem into two sub-problems
%     which we outline below.

In practice, we add an additional constraint to the original formulation, \cite{opex1}
    which is to set hard-limits on the allowable values of $p$,
    namely $p_0 \le p \le p_1$.
This is actually a relaxation of the ideal constraint,
    which would be to allow $p$ to only have discrete values,
    $p \in p_0, p_1$,
    but such a constraint would be essentially force us to only
    be able to perform brute force trial-and-error.

Our objective-first formulation is thus,
\BA \minimize{x,p} \| A(p) x - b(p) \|^2 \notag \\
    \subto f(x) = f_\text{ideal} \label{eq:ob1} \\
        & p_0 \le p \le p_1, \notag \EA
    which is still non-convex, but can be broken down into 
    two convex sub-problems, 
    the motivation being that each of these will be 
    able to be easily and reliably solved.

\subsection{Field sub-problem}
The first of these is the field sub-problem, 
    which simply involves fixing $p$ and independently optimizing $x$,
\BA \minimize{x} \| A(p) x - b(p) \|^2 \label{eq:Fsub} \\
    \subto f(x) = f_\text{ideal}. \notag \EA
This problem is convex, and actually quadratic,
    which means that it can even be solved 
    using standard numerical tools, in the same way 
    as a simple least-squares problem.

The field sub-problem can be thought of as an update to $x$ (field)
    where we try to ``fit'' the electromagnetic fields to the structure ($p$).
Of course, if it were not for the hard-constraint on the design objective,
    the field sub-problem would be able to perfectly fit $x$ to $p$.
This, it turns out, would exactly be a simulation.

\subsection{Structure sub-problem}
The second sub-problem is formulated by fixing $x$ and
    independently optimizing $p$.
At the same time, we use the bi-linearity property
    of the physics residual from \ER{bilinear}
    to rewrite the problem in a way that makes
    its convexity explicit,
\BA \minimize{p} \| B(x) p - d(x) \|^2 \label{eq:Ssub} \\
    \subto p_0 \le p \le p_1. \notag \EA
The structure sub-problem is also convex, but not quadratic because of the 
    inequality constraints on $p$.
However, use of the CVX package still allows us to obtain the result
    quickly and reliably.

Note that in an analogous fashion to the field sub-problem,
    the structure sub-problem attempts to fit $p$ to $x$,
    and is prevented from perfectly doing so by its own constraint.

Because neither sub-problem is capable of completely reducing the physics residual
    to zero, they must be used in an iterative manner in order to
    gradually decrease the physics residual.
To this end, we employ the alternating directions optimization method.

\subsection{Alternating directions}
We use a simple alternating directions scheme 
    to piece together \ER{Fsub} and \ER{Ssub},
    which is to say that we simply
    alternately solve each and continue until we reach some stopping point,
    normally measured by how much the physics residual has decreased.
\BA \text{Loop:} & & \notag \\ 
& \minimize{x} \| A(p) x - b(p) \|^2 \notag \\
 &    \subto f(x) = f_\text{ideal}. \notag \\ 
    \\
& \minimize{p} \| B(x) p - d(x) \|^2 \notag \\
    & \subto p_0 \le p \le p_1. \notag \EA
The alternating directions scheme is extremely simple
    and does not require additional processing
    of $x$ or $p$ outside of the two sub-problems,
    nor does it require the use of auxiliary variables.

The advantage of such the alternating directions method
    is that the physics residual is guaranteed to
    monotonically decrease with every iteration,
    which is useful in that no safeguards
    are needed to guard against ``rogue'' steps
    in the optimization procedure.
Note that this robustness stems from the fact that,
    among other things,
    each sub-problem does not rely on previous values of 
    the variable which is being optimized,
    but only on the variable which is held constant.

The disadvantage of such a simple scheme is that 
    the convergence is quite slow,
    although we have found it to be sufficient in our cases.
Related methods, such as the Alternating Directions Method of Multipliers\cite{admm}, 
    exhibit far better convergence.

% \section{1D resonator design}
We now build up to the objective-first formulation by example. 
And from a different perspective.

We're going to take a step back and start with the most naive inverse design strategy possible.
Then we'll end with objective first.
This motivates the need for the alternating directions and objective-first strategy.


\subsection{Direct solve of structure}
This is the simplest thing you can do. Basically, set $x$ and solve for $p$.
Also, take out the relaxed constraint on $p$.
\BA \minimize{p} \| B(x) p - d(x) \|^2 \EA
\myfig{1d/leastsquares}{test}
We perfectly satisfy the field but...

\subsection{Regularized solve of structure}
We add a term to try to control $p$
\BA \minimize{p} \| B(x) p - d(x) \|^2 + \eta \|p - p_0\|^2 \EA
\myfig{1d/regularized}{test}
Better, but a trade-off between field accuracy and structure variation.

\subsection{Alternating directions solve}
Now we actually do both fields.
\BA \minimize{p} \| B(x) p - d(x) \|^2 + \eta_1 \|p - p_\text{prev}\|^2 \EA

\BA \minimize{x} \| A(p) x - b(p) \|^2 + \eta_2 \|x - x_\text{prev}\|^2 \EA
\myfig{1d/complementary}{test}

\subsection{Alternating directions solve with bounded $p$}
\BA \minimize{p} \| B(x) p - d(x) \|^2 \\
    \subto 0 \le p \le 1 \EA

\BA \minimize{x} \| A(p) x - b(p) \|^2 + \eta_2 \|x - x_\text{prev}\|^2 \EA
\myfig{1d/bounded}{test}



% \section{Resonator design}
Now that we see the usefulness of objective-first design strategy for a 1D resonator,
    we extend to 2D and approximate 3D as well.

We also transition design objectives:
    field everywhere,
    to certain field characteristics.

We continue to fill-out our understanding of objective-first,
    and we get a fuller flavor of what it can do.

\subsection{``S'' resonator}
We construct an ``S''-shaped field and create a resonator to make that field.
This uses the previous equation...
\myfig{res/S}{caption}

\subsection{2D}
Now we design for minimal mode-volume and maximal Q.
Although $Q^{-1}$ minimization should be a constraint we only do this in the next section.
\BA \minimize{x} \| A(p) x - b(p) \|^2 + \eta \| F x \|^2 \\
    \subto \| \text{diag}(\sqrt{p}) A_\text{curl} x \|^2 \le a_\text{mode} \EA
\BA \minimize{p} \| B(x) p - d(x) \|^2 \\
    \subto 0 \le p \le 1 \EA
\myfig{res/beam}{caption}
\myfig{res/circle}{caption}

\subsection{2.5D}
Although solving for the relevant matrices in 3D is really hard, we can make an approximation.

Here we make 
\BA \minimize{x} \| A(p) x - b(p) \|^2 + \eta \|\text{diag}(\sqrt{p}) A_\text{curl} x\|^2 \\
    \subto Fx = 0  \EA
\BA \minimize{p} \| B(x) p - d(x) \|^2 \\
    \subto 0 \le p \le 1 \EA

\myfig{res/target}{Add the actual field here!}



\section{Waveguide coupler design}
Finally, we now implement 
    the full objective-first design strategy
    by doing an alternating directions solve.

\subsection{Choice of design objective}
We choose, for generality,

\BE f(x) = \begin{cases}
        x & \text{at boundary}, \\
        0 & \text{elsewhere}.
        \end{cases} \EE

    where the boundary denotes the two outermost layers of the design space.
This means 

\BE f_\text{ideal} = \begin{cases}
        x_\text{perfect} & \text{at boundary}, \\
        0 & \text{elsewhere}.
        \end{cases} \EE

We then do

\BA \minimize{x} \| A(p) x - b(p) \|^2 \\
    \subto f(x) = f_\text{ideal} \EA

\BA \minimize{p} \| B(x) p - d(x) \|^2 \\
    \subto 0 \le p \le 1 \EA

\subsection{Coupling between dielectric waveguide modes}
\myfig{wg/1}{test}
\myfig{wg/2}{test}
\myfig{wg/3}{test}

\subsection{Coupling to plasmonic waveguide modes}
\myfig{wg/4}{test}
\myfig{wg/5}{test}




\section{Metamaterials design}
\subsection{Modification of the design objective}
\subsection{Cloak devices}
\subsection{Mimic devices}

\section{Extending the method}
\subsection{3D}
\subsection{Multi-mode}
\subsection{Robustness}
\subsection{Binary structure}


\section{Appendix}
\subsection{Full 3D curl}
\subsection{1D}
\subsection{2D}
\subsection{2.5D}
