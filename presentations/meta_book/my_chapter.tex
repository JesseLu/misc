% Some custom commands.
\newcommand{\BE}{\begin{equation}}
\newcommand{\EE}{\end{equation}}
\newcommand{\BA}{\begin{eqnarray}}
\newcommand{\EA}{\end{eqnarray}}
\newcommand{\curl}{\nabla\times}
\newcommand{\minimize}[1]{\JLUminimize_{#1}\;&}
\newcommand{\subto}{\text{subject to}\;&}

% Custom figures.
\newcommand{\myfig}[2]{\begin{figure}[!h]\includegraphics[width=0.95\textwidth]{fig/#1.jpg}\caption{#2}\label{fig:#1}\end{figure}}

\newcommand{\BI}{\begin{itemize}\item}
\renewcommand{\I}{\item}
\newcommand{\EI}{\end{itemize}}

\newcommand{\ER}[1]{\eqref{eq:#1}}
\newcommand{\SR}[1]{Section~\ref{sec:#1}}
\newcommand{\sR}[1]{section~\ref{sec:#1}}
\newcommand{\FR}[1]{Figure~\ref{fig:#1}}
\newcommand{\fR}[1]{figure~\ref{fig:#1}}
\chapter{Objective-First Nanophotonic Design}
\label{intro}

\abstract{We introduce an ``objective-first'' strategy 
    for designing nanophotonic devices,
    and we demonstrate the design of nanophotonic
    coupler, cloak, and mimic devices.}
    

Our initial foray into design methods for nanophotonic devices
    began with a very simple and naive question: that is,
    if, when presented with a nanophotonic structure,
    we can solve for the electromagnetic fields it produces;
    then, why don't we just make an inverse solver which,
    when given the electromagnetic fields we desire,
    returns the nanophotonic structure that will produce them? %ref.
In other words, since we already know how to solve for $E$ and $H$ 
    in Maxwell's equations, why can't we solve for $\epsilon$
    or even $\mu$ instead?

Not surprisingly, it did not take us long to find that such
    a simple strategy would inevitably run into many problems.

Over the subsequent years, we were able to come up with a better solution,
    which we call an ``objective-first'' strategy for nanophotonic design,
    and which we present in this chapter.
Although it is much more advanced than our original idea,
    objective-first design still carries the same fundamental concept;
    which is to specify the electromagnetic fields, and then
    to solve for a structure to produce them.
    
In this chapter, we present the simple theoretical underpinnings of 
    objective-first design in the first two sections,
    and then show examples of the method in action in the rest of the chapter.
We also include the source code that was used to generate
    all results presented herein. %ref!
    

\section{The electromagnetic wave equation}
In this section, we outline the wave equation
    that is central to the application of our method,
    with the end-result being to show that it is 
    separably linear (bi-linear) in the field and structure variables.
We do this by first formulating this wave equation in the language of physics,
    and then discretizing it in order to achieve numerical solutions.
We then show how one can not only obtain the solution for the field,
    but how one can obtain the solution for the structure using
    simple, standard numerical tools.

\subsection{Physics formulation}
First, let's derive our wave equation,
    starting with the differential form of Maxwell's equations, 
\BA \curl E = - \mu_0 \frac{\partial H}{\partial t} \\
    \curl H = J + \epsilon \frac{\partial E}{\partial t}, \EA
    where $E$, $H$, and $J$ are 
    the electric, magnetic and electric current % TODO: Check this.
    vector fields, respectively,
    $\epsilon$ is the permittivity
    and $\mu_0$ is the permeability, which we assume to be 
    that of vacuum everywhere.


We use the time dependence $\exp(-i \omega t)$, 
    where $\omega$ is the angular frequency,
    these become
\BA \curl E = - i \mu_0 \omega H \\
    \curl H = J + i \epsilon \omega E, \label{eq:H2E} \EA
    which we can combine to form our (time-harmonic) wave equation,
\BE \curl \epsilon^{-1} \curl H - \mu_0 \omega^2 H = \curl \epsilon^{-1} J. 
    \label{eq:wave} \EE

In this chapter, we are going to only consider the two-dimensional form
    of this equation; and specifically 
    the two-dimensional transverse magnetic (TM) mode. % Ref.
In this case \ER{wave} is simplified not only in that
    the vector fields are two-dimensional, 
    but that only the $z$-component of $H$ is non-zero.

Nevertheless, \ER{wave}, in a single equation,
    represents all the physics which we take into account in this chapter.

\subsection{Numerical formulation}
Now, on top of the analytical formulation of the wave equation \ER{wave}
    we will now add a numerical, or discretized, formulation.
This will be needed in order to solve for arbitrary structures
    for which there are not analytical solutions.

The salient step in order to do so
    is to the use of the Yee grid, % TODO: Reference!
    which allows us to easily define the curl $(\curl)$
    operators in \ER{wave}. % TODO: ref.
Since both the individual curl operators and the equation as a whole
    is linear in $H$, it naturally follows to formulate \ER{wave},
    with a change of variables, as
\BE A(p)x = b(p), \label{eq:Ab} \EE 
    where $H \to x$, $\epsilon^{-1} \to p$; 
    and where
\BE A(p) = \curl \epsilon^{-1} \curl - \mu_0 \omega^2 \EE 
    and
\BE b(p) = \curl \epsilon^{-1} J. \EE
Note that our use of $A(p)$ and $b(p)$ instead of $A$ and $b$
    simply serves to clarify the dependence
    of both $A$ and $b$ to $p$.

Apart from using the Yee grid, the only other salient implementation detail
    is the use of
    stretched-coordinate perfectly matched layers %ref
    where necessary, in order to prevent unwanted reflections
    at the boundaries of the simulation domain.
The effect of such layers is to modify the curl operators,
    although their linear property is still maintained.

\subsection{Solving for $H$}
With our numerical formulation, we can now solve for the $H$-field
    (the $E$-field can be computed from the $H$-field using \ER{H2E})
    by applying general linear algebra solvers to \ER{Ab}.
Recall that since we have chosen a time-harmonic formulation,
    solving for $x$ in \ER{Ab} is actually performing what is simply known as
    a time-harmonic or a finite-difference frequency-domain (FDFD) simulation.
Furthermore, since we have limited ourselves to the two-dimensional case,
    \ER{Ab} is easily solved using the standard sparse solver
    included in Matlab on a single desktop computer.

We call the routine that solves for $x$ in \ER{Ab} given $p$ a field-solver,
    or a simulator.
    


\subsection{Solving for $\epsilon^{-1}$}
Now, the next step,
    after having built a field-solver or simulator
    (finds $x$ given $p$) for our wave equation,
    is to build a structure-solver for it.
In other words, we need to be able to solve for $p$ given $x$.

To do so, we return to \ER{wave}
    and remark that 
    $\epsilon^{-1} (\curl H) = (\curl H) \epsilon^{-1}$ and
    $\epsilon^{-1} J = J \epsilon^{-1}$ 
    since scalar multiplication is communicative.
This allows us to rearrange \ER{wave} as
\BE \curl (\curl H) \epsilon^{-1} - \curl J \epsilon^{-1}  = \mu_0 \omega^2 H  \EE
which we now write as 
\BE B(x)p = d(x), \label{eq:Bd} \EE 
    where
\BE B(x) = \curl (\curl H) - \curl J\EE
    and 
\BE d(x)  = \mu_0 \omega^2 H.  \EE

With this extremely simple trick,
    we have shown that we can seemingly solve for $p$ given $x$
    with approximately the same ease as solving for $x$ given $p$!
We see this because the dimensions and complexity of $B(x)$ is basically
    equivalent to that of $A(p)$,
    and this implies that the same simple tools used in our field-solver
    should be applicable to solving \ER{Bd}.
This is indeed what we find, although the later addition of constraints on $p$
    will require the use of more powerful (but just as dependable) numerical tools.

\subsection{Bi-linearity of the wave equation}
Although additional mathematical machinery must still be added
    in order to get a useful design tool,
    we have really shown so far is that the wave equation is 
    separately linear or \emph{bi-linear} in $x$ and $p$.
Namely that,
\BE A(p)x-b(p) = B(x)p - d(x). \label{eq:bilinear} \EE
In other words, fixing $p$ makes solving the wave equation for $x$
    a linear problem, and vice versa.
Note that the joint problem,
    where both $x$ and $p$ are allowed to vary,
    is not linear.

The bi-linearity of the wave equation
    is \emph{absolutely fundamental} in our objective-first strategy
    because it relies on the fact
    that, although simultaneously solving for $x$ and $p$ is very difficult,
    we already know how to solve linear systems ($x$ and $p$ separately) well.
In fact, it is this very property which forms the natural division of labor
    which the objective-first method, which we now present, exploits.


\section{The objective-first design problem}
We now describe the remaining machinery used in the objective-first method,
    in addition to the field-solver and the structure-solver,
    as previously outlined.
Specifically, we introduce the idea of a design objective and a physics residual,
    and we reference the mathematical notion of convexity
    in order to motivate the need to divide the 
    objective-first problem into two separately convex sub-problems.

\subsection{Design objectives} \label{sec:desobj}
A design objective, $f(x)$, is simply defined as
    a function we wish to be minimal 
    for the design to be produced.

For instance, in the design of a device
    which must transmit efficiently into a particular mode,
    we could choose $f(x)$ to be the negative power flow into that mode.
Or, if the device was to be a low-loss resonator,
    we could choose $f(x)$ to be the amount of power leaking 
    out of the device.

In general, there are multiple choices of $f(x)$
    which can be used to describe the same objective.
For example, $f(x)$ for a transmissive device 
    may not only be the negative power transmitted into the desired output mode,
    but it could also be the amount of power lost to other modes,
    or even the error in the field values at the output port
    relative to the field values needed for perfect transmission.
These design objectives are equivalent in the sense that, if minimized, 
    all would produce structures with good performance.
At the same time, we must consider that the computational cost and complexity
    of using one $f(x)$ over another may indeed vary greatly.

\subsection{Convexity}
Before formulating the design problem,
    we would like to add a note regarding the complexity of various 
    optimization problems.

Specifically, we want to introduce the notion of \emph{convexity}\cite{boyd}
    and to note the difference between problems that
    are convex and those which are not.
The difference is simply this:
    convex problems have a single optimum point
    (only one local optimum, which is therefore the global optimum)
    which we can reliably find using existing numerical software,
    whereas non-convex problems typically have multiple optima 
    and are thus much more difficult to reliably solve.

That a convex problem can be reliably solved, in this case, 
    means that regardless of the starting guess,
    convex optimization software will 
    always arrive at the globally optimal solution
    and will be able to numerically prove global optimality as well.
Thus, the advantage in formulating a design problem 
    in terms of convex optimization problems
    is to eliminate both the need to circumvent local optima and 
    any notion of randomness.

On a practical level, there exist mature convex optimization software packages
    among which is CVX, a convex optimization package written for Matlab\cite{cvx}, 
    which we use for the examples in this chapter.


\subsection{Typical design formulation}
We now examine the typical, and most straightforward formulation
    of the design problem,
    in order to relate and contrast it to the objective-first formulation.
The design problem for a physical structure is typically formulated as
\BA \minimize{x,p} f(x) \label{eq:typform} \\
    \subto A(p)x - b(p) = 0, \notag \EA
    which states that we would like to vary $x$ and $p$ simultaneously
    in order to decrease $f(x)$
    while always satisfying physics (e.g. the electromagnetic wave equation).

Since solving \ER{typform} is quite difficult in the general sense
    (simultaneously varying $x$ and $p$ makes the problem non-convex),
    traditional approaches have relied on either brute-force parameter search,
    or a gradient-descent method utilizing first-order derivatives.
In the gradient-descent case, solving \ER{typform} results in the
    well-known adjoint optimization method\cite{miller}. 

\subsection{Objective-first design formulation}
In contrast with the typical formulation, the objective-first formulation 
    simply switches the roles of 
    the wave equation and the design objective with one another,
\BA \minimize{x,p} \| A(p) x - b(p) \|^2 \label{eq:ob1:1} \\
    \subto f(x) = f_\text{ideal}. \label{eq:ob1:2} \EA
Although such a switch may seem trivial,
    and even silly at first,
    we show that it fundamentally changes the nature of the design problem
    and actually gains us advantages in our efforts at finding a solution.

This first fundamental change, as seen from \ER{ob1:1},
    is that we allow for non-zero residual in the electromagnetic wave equation.
This literally means that we allow for \emph{non-physical} $x$ and $p$,
    since $A(p) x - b(p) \ne 0$ is permissible.
And since $A(p) x - b(p)$ can now be a non-zero entity, 
    we choose to call it the \emph{physics residual}.  
The second fundamental change
    is that we always force the device to exhibit ideal performance,
    as seen from \ER{ob1:2}.
This, of course, ties in very closely with \ER{ob1:1} since
    ideal performance is usually not obtainable unless one
    allows for some measure of error in the underlying physics
    (non-zero physics residual).
As such, our strategy will be to vary $x$ and $p$
    in order to decrease the physics residual \ER{ob1:1} to zero,
    while always maintaining ideal performance.

The primary advantage in the objective-first formulation is that,
    although the full problem is still non-convex,
    it allows us to form two convex sub-problems, as we outline below.
In contrast to an adjoint method approach,
    in doing we can still access information regarding
    second-order derivatives, which greatly speeds up finding a solution.
An additional advantage is that our insistence
    that ideal performance be always attained
    provides a mechanism
    which can potentially ``override'' local optima 
    in the optimization process.

To this end we have found that such a strategy
    actually allows us to design very unintuitive devices
    which exhibit very good performance,
    even when starting from completely non-functional initial guesses.
Furthermore, we have found this to be true
    even true when the physics residual is never brought to exactly zero.

% From a numerical standpoint, 
%     although the objective-first formulation is still non-convex 
%     in its original form,
%     the bi-linearity of the physics residual term allows us to 
%     naturally break the original problem into two sub-problems
%     which we outline below.

In practice, we add an additional constraint to the original formulation, \cite{opex1}
    which is to set hard-limits on the allowable values of $p$,
    namely $p_0 \le p \le p_1$.
This is actually a relaxation of the ideal constraint,
    which would be to allow $p$ to only have discrete values,
    $p \in p_0, p_1$,
    but such a constraint would be essentially force us to only
    be able to perform brute force trial-and-error.

Our objective-first formulation is thus,
\BA \minimize{x,p} \| A(p) x - b(p) \|^2 \notag \\
    \subto f(x) = f_\text{ideal} \label{eq:ob1} \\
        & p_0 \le p \le p_1, \notag \EA
    which is still non-convex, but can be broken down into 
    two convex sub-problems, 
    the motivation being that each of these will be 
    able to be easily and reliably solved.

\subsection{Field sub-problem}
The first of these is the field sub-problem, 
    which simply involves fixing $p$ and independently optimizing $x$,
\BA \minimize{x} \| A(p) x - b(p) \|^2 \label{eq:Fsub} \\
    \subto f(x) = f_\text{ideal}. \notag \EA
This problem is convex, and actually quadratic,
    which means that it can even be solved 
    using standard numerical tools, in the same way 
    as a simple least-squares problem.

The field sub-problem can be thought of as an update to $x$ (field)
    where we try to ``fit'' the electromagnetic fields to the structure ($p$).
Of course, if it were not for the hard-constraint on the design objective,
    the field sub-problem would be able to perfectly fit $x$ to $p$.
This, it turns out, would exactly be a simulation.

\subsection{Structure sub-problem}
The second sub-problem is formulated by fixing $x$ and
    independently optimizing $p$.
At the same time, we use the bi-linearity property
    of the physics residual from \ER{bilinear}
    to rewrite the problem in a way that makes
    its convexity explicit,
\BA \minimize{p} \| B(x) p - d(x) \|^2 \label{eq:Ssub} \\
    \subto p_0 \le p \le p_1. \notag \EA
The structure sub-problem is also convex, but not quadratic because of the 
    inequality constraints on $p$.
However, use of the CVX package still allows us to obtain the result
    quickly and reliably.

Note that in an analogous fashion to the field sub-problem,
    the structure sub-problem attempts to fit $p$ to $x$,
    and is prevented from perfectly doing so by its own constraint.

Because neither sub-problem is capable of completely reducing the physics residual
    to zero, they must be used in an iterative manner in order to
    gradually decrease the physics residual.
To this end, we employ the alternating directions optimization method.

\subsection{Alternating directions}
We use a simple alternating directions scheme 
    to piece together \ER{Fsub} and \ER{Ssub},
    which is to say that we simply
    alternately solve each and continue until we reach some stopping point,
    normally measured by how much the physics residual has decreased.
\BA \text{Loop:} & & \notag \\ 
& \minimize{x} \| A(p) x - b(p) \|^2 \notag \\
 &    \subto f(x) = f_\text{ideal}. \notag \\ 
    \\
& \minimize{p} \| B(x) p - d(x) \|^2 \notag \\
    & \subto p_0 \le p \le p_1. \notag \EA
The alternating directions scheme is extremely simple
    and does not require additional processing
    of $x$ or $p$ outside of the two sub-problems,
    nor does it require the use of auxiliary variables.

The advantage of such the alternating directions method
    is that the physics residual is guaranteed to
    monotonically decrease with every iteration,
    which is useful in that no safeguards
    are needed to guard against ``rogue'' steps
    in the optimization procedure.
Note that this robustness stems from the fact that,
    among other things,
    each sub-problem does not rely on previous values of 
    the variable which is being optimized,
    but only on the variable which is held constant.

The disadvantage of such a simple scheme is that 
    the convergence is quite slow,
    although we have found it to be sufficient in our cases.
Related methods, such as the Alternating Directions Method of Multipliers\cite{admm}, 
    exhibit far better convergence.

\section{Waveguide coupler design}
Finally, we now implement 
    the full objective-first design strategy
    by doing an alternating directions solve.

\subsection{Choice of design objective}
We choose, for generality,

\BE f(x) = \begin{cases}
        x & \text{at boundary}, \\
        0 & \text{elsewhere}.
        \end{cases} \EE

    where the boundary denotes the two outermost layers of the design space.
This means 

\BE f_\text{ideal} = \begin{cases}
        x_\text{perfect} & \text{at boundary}, \\
        0 & \text{elsewhere}.
        \end{cases} \EE

We then do

\BA \minimize{x} \| A(p) x - b(p) \|^2 \\
    \subto f(x) = f_\text{ideal} \EA

\BA \minimize{p} \| B(x) p - d(x) \|^2 \\
    \subto 0 \le p \le 1 \EA

\subsection{Coupling between dielectric waveguide modes}
\myfig{wg/1}{test}
\myfig{wg/2}{test}
\myfig{wg/3}{test}

\subsection{Coupling to plasmonic waveguide modes}
\myfig{wg/4}{test}
\myfig{wg/5}{test}



\section{Optical cloak design}
In the previous section,
    we showed that couplers between virtually any two waveguide modes
    could be constructed using the objective-first design method,
    and based on the generality of the method
    one can guess that it may also be
    able to generate designs for any linear nanophotonic device.

Now, we extend the applicability of our method
    to the design of metamaterial devices which operate in free-space.
In particular,
    we adapt the waveguide coupler algorithm to the 
    to the design of optical cloaks.

\subsection{Application of the objective-first strategy}
Adapting the method used in \sR{wg} to the design of optical cloaks
    really only requires one to change the simulation environment
    to allow for free-space modes.
This is accomplished by modifying the upper and lower boundaries
    of the simulation domain from absorbing boundary conditions
    to periodic boundary conditions,
    which allows for plane-wave modes to propagate without loss
    until reaching the left or right boundaries,
    where absorbing boundary conditions are still maintained.

In terms of the design objective, 
    we allow the device to span the entire height of the simulation domain,
    and thus consider only the leftmost and rightmost planes as boundary values.
Specifically, for this section the input and output modes are plane waves
    with normal incidence, 
    as can be expected for good cloaking devices.
The achieved results all yield high efficiency,
    although we note that the cloaking effect is only measured
    for a specific input mode.
That is to say, just as the waveguide couplers previously designed
    were single-mode devices,
    so the cloaks designed in this section are also ``single-mode'' cloaks.

An additional modification, as compared to \sR{wg}, is that
    we now disallow the structure to be modified in certain areas
    which, naturally, contain the object to be cloaked.

With these simple changes we continue to solve \ER{ob1}
    with the alternating directions method
    in order to now design optical cloaks
    instead of waveguide couplers.
Once again, as in \sR{wg}, each design is run for 400 iterations
    with a uniform initial value of $p = 1/9$ for the structure
    (where the structure is allowed to vary),
    and the range of $p$ is limited to $1/12.25 \le p \le 1$,
    implying a dielectric cloak.

\subsection{Anti-reflection coating}
As a first example,
    we attempt to design the simplest and most elementary ``cloaking'' device available,
    which, we argue, is a simple anti-reflection coating;
    in which case the object to be cloaked is nothing more than
    the interface between two dielectric materials.
In this case we use the interface between air and silicon, as shown in \fR{cloak/c1}
\myfig{cloak/c1}{Anti-reflection coating.
                Efficiency: 99.99\%,
                device footprint: $60 \times 100$ grid points,
                wavelength: 63 grid points.}

Unsurprisingly for such a simple case, 
    we achieve a very high efficiency device.
Note also that the efficiency of the device can be deduced by eye,
    based on the absence of reflections or standing waves 
    in bottom two plots of \fR{cloak/c1}.

\subsection{Wrap-around cloak}
Next, we design a cloak for a plasmonic cylinder,
    which is quite effective at scattering light
    as can be seen from \fR{cloak/c6}.
\myfig{cloak/c6}{Plasmonic cylinder to be cloaked. 
                68.5\% of light is diverted away from the desired output mode.}

In designing the wrap-around cloak,
    we allow the structure to vary at all points within the design area
    except in the immediate vicinity of the plasmonic cylinder.
Application of the objective-first strategy results
    in an efficient device as seen in \fR{cloak/c2}.
\myfig{cloak/c2}{Wrap-around cloak.
                Efficiency: 99.99\%,
                device footprint: $60 \times 100$ grid points,
                wavelength: 42 grid points.}

Note that our cloak employs only isotropic, non-magnetic materials,
    and at the same time it is specific to a particular input
    and to a particular object.

\subsection{Open-channel cloak}
With a simple modification, from the previous section,
    we can design a cloak which features an open channel
    to the exterior electromagnetic environment.
This simple modification is forcing an air tunnel
    to be opened which connects the cylinder to the outside world
    both toward its front and back.
\myfig{cloak/c4}{Open-channel cloak.
                Efficiency: 99.8\%,
                device footprint: $60 \times 100$ grid points,
                wavelength: 42 grid points.}

Such a design is still very efficient
    and exhibits the usefulness of the objective-first strategy
    in cases where other methods, such as transformation optics,
    may not be able to be applied.

\subsection{Channeling cloak}
Our last cloaking example replaces the plasmonic cylinder 
    with a thin metallic wall in which a sub-wavelength channel is etched.
Such a metallic wall is very effective at blocking incoming light
    (as can be seen from \fR{cloak/c7}) 
    because of its large negative permittivity ($\epsilon = -20$),
    meaning that any cloaking device would be forced to channel
    all the input light into a very small aperture
    and then to flatten that light out into a plane wave again.
\myfig{cloak/c7}{Metallic wall with sub-wavelength channel to be cloaked.
                99.9\% of the light is blocked from the desired output plane-wave.}

Once again, our method is still able to produce a very efficient design,
    as shown in \fR{cloak/c5}.
\myfig{cloak/c5}{Channeling cloak.
                Efficiency: 99.9\%,
                device footprint: $60 \times 100$ grid points,
                wavelength: 42 grid points.}


\section{Optical mimic design}
We now apply our objective-first strategy to the design of optical mimics.

We define an optical mimic to be a linear nanophotonic device which 
    mimics the output field of another device.
In this sense optical mimics are the anti-cloak;
    whereas cloaks strive to make an object's electromagnetic presence vanish,
    mimics strive to implement an object's presence without that 
    object actually being there.

As such, the design of optical mimics provides a tantalyzing approach
    to the realization of practical metamaterial devices.
That is to say, if one can reliably produce practical optical mimics,
    then producing metamaterials (which are often based on fictitious materials)
    can be accomplished by simply producing an optical mimic of that material.

In a more general sense, 
    optical mimics is really just a recasting of the thrust of 
    the objective-first design strategy in its purest form:
    the design of a nanophotonic device based purely 
    on the electromagnetic fields one wishes to have it produce.
As such, devices which perform well-known optical functions
    (e.g. focusing, lithography) can also be designed.

\subsection{Application of the objective-first strategy}
The objective-first design of optical mimics proceeds in virtually
    an identical way to the design of optical cloaks,
    the only difference being that the output modes are 
    specifically chosen to be those which produce the desired function.
For most of the examples provided, the input illumination is still an incident plane wave.

Lastly, instead of measuring efficiency, 
    we measure the relative error of the simulated field
    against that of a perfect target field
    at a relevant plane of some distance away from the device.
The location of this plane is identified as a dotted line in the subsequent figures.

\subsection{Plasmonic cylinder mimic}
Our first mimic is simply to mimic the plasmonic cylinder 
    which we cloaked in the previous section.
\myfig{mimic/m1}{Plasmonic cylinder mimic. 
                Error: 8.1\%,
                footprint: $40 \times 120$ grid points,
                wavelength: 42 grid points.}

\subsection{Diffraction-limited lens mimic}
\myfig{mimic/m3}{Full-width-half-max at focus: 1.5 $\lambda$,
                focus depth: 100 grid points.
                Error: 12.0\%,
                footprint: $40 \times 120$ grid points,
                wavelength: 25 grid points.}
\myfig{mimic/m4}{Full-width-half-max at focus: 0.5 $\lambda$,
                focus depth: 50 grid points.
                Error: 5.6\%,
                footprint: $40 \times 120$ grid points,
                wavelength: 25 grid points.}
\myfig{mimic/m5}{Full-width-half-max at focus: 0.5 $\lambda$,
                focus depth: 150 grid points.
                Error: 1.4\%,
                footprint: $40 \times 120$ grid points,
                wavelength: 25 grid points.}

\subsection{Sub-diffraction lithographic mask}
\myfig{mimic/m6}{Error: 19.8\%}

\subsection{Perfect lens mimic}
\myfig{mimic/m2}{Perfect lens mimic.
                Error: 28.6\%}




\section{Extending the method}
The objective-first method, as applied in the examples in this chapter,
    really only represent a small foray into the area of nanophotonic design.
Several key extensions to what is presented here are needed to
    fully address real-world nanophotonic design challenges.

\subsection{3D}
The first of these is the need to design fully three-dimensional structures.
Doing so provides no inherent difficulties aside from the matrices in
    \ER{Ab} becoming very large.
This is not insurmountable as electromagnetic simulation software for 
    three-dimensional nanophotonic structures already exists.

Also software to solve \ER{Bd} in three-dimensions does not exist,
    the size of matrix $B(x)$ can be greatly compressed by considering
    only fabrication processes which modify a structure in-plane.
Thus, the degrees of freedom in $p$ can be greatly reduced and the original
    methods used in this chapter can still be applied.
    
\subsection{Multi-mode}
A second necessary extension is to be able to consider the multiple fields
    that a structure produces.
This includes fields of different frequencies and inputs.

\subsection{Robustness}
\subsection{Binary structure}

