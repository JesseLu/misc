% Some custom commands.
\newcommand{\BE}{\begin{equation}}
\newcommand{\EE}{\end{equation}}
\newcommand{\BA}{\begin{eqnarray}}
\newcommand{\EA}{\end{eqnarray}}
\newcommand{\curl}{\nabla\times}
\newcommand{\minimize}[1]{\JLUminimize_{#1}\;&}
\newcommand{\subto}{\text{subject to}\;&}

% Custom figures.
% smallfig/ directory is for lo-fi figures,
% fig/ directory is for full resolution.
\newcommand{\myfig}[2]{\begin{figure}[!h]\includegraphics[width=\textwidth]{fig/#1}\caption{#2}\label{#1}\end{figure}}
\newcommand{\BI}{\begin{itemize}\item}
\renewcommand{\I}{\item}
\newcommand{\EI}{\end{itemize}}
\newcommand{\ER}[1]{\eqref{eq:#1}}
\chapter{Objective-First Nanophotonic Design}
\label{intro}

\abstract{The abstract for the book.}

In this chapter, we introduce an ``objective-first'' strategy 
    for designing nanophotonic devices.

General description:
\BI Only uses desired fields.
\I  Can attempt to design any linear device.
\I  Allows for non-physical fields. \EI

This strategy is unique in that it
    asks the user only for what electromagnetic field
    the device should produce,
    and then attempts to generate the design
    without further user intervention.

\section{The electromagnetic wave equation}

We start by understanding the underlying physical equation.
We extend it beyond its usual use in simulation, 
    and use it for design.

\subsection{Physics formulation}
First, let's derive our wave equation,
    starting with the differential form of Maxwell's equations, 
\BA \curl E = - \mu_0 \frac{\partial H}{\partial t} \\
    \curl H = J + \epsilon \frac{\partial E}{\partial t}, \EA
    where $E$, $H$, and $J$ are 
    the electric, magnetic and electric current % TODO: Check this.
    vector fields, respectively,
    $\epsilon$ is the permittivity
    and $\mu_0$ is the permeability, which we assume to be 
    that of vacuum everywhere.


For time dependence $\exp(-i \omega t)$, 
    where $\omega$ is the angular frequency,
    these become
\BA \curl E = - i \mu_0 \omega H \\
    \curl H = J + i \epsilon \omega E, \label{eq:H2E} \EA
    which we can combine to form our wave equation,
\BE \curl \epsilon^{-1} \curl H - \mu_0 \omega^2 H = \curl \epsilon^{-1} J. 
    \label{eq:wave} \EE
For further information,
    as well as simplifications to the wave equation in reduced dimensions,
    please see the Appendix. % TODO: Use reference.

\subsection{Numerical formulation}
Now, on top of the analytical formulation of the wave equation \ER{wave}
    we will now add a numerical, or discretized, formulation.
This will be needed in order to solve for arbitrary structures.

First, we discretize our computational space 
    according to the Yee grid, % TODO: Reference!
    which allows us to easily define the curl $(\curl)$
    operators in \ER{wave} as described in the Appendix. % TODO: ref.
This allows us, with a change of variables to formulate \ER{wave} as
\BE A(p)x = b(p), \label{eq:Ab} \EE 
    where $H \to x$, $\epsilon^{-1} \to p$; 
    and where
\BE A(p) = \curl \epsilon^{-1} \curl - \mu_0 \omega^2 \EE 
    and
\BE b(p) = \curl \epsilon^{-1} J. \EE
Note that our use of $A(p)$ and $b(p)$ instead of $A$ and $b$
    simply serves to clarify the dependence
    of both $A$ and $b$ to $p$.

Additionally,  we use periodic boundary conditions
    with stretched-coordinate % TODO: ref.
    perfectly matched layers where necessary for our examples.

\subsection{Solving for $H$}
With our numerical formulation, we can now solve for the $H$-field
    (the $E$-field can be computed from the $H$-field using \ER{H2E})
    by using general linear algebra solvers.
Doing so is also simply known as a time-harmonic or 
    a finite-difference frequency-domain (FDFD) simulation.

Now, while a full three-dimensional problem
    is computationally quite taxing;
    in one- and two-dimensions, 
    \ER{Ab} is easily solved using the standard sparse solver
    included in Matlab,
    and this technique is regularly employed 
    in the examples which follow.


\subsection{Solving for $\epsilon^{-1}$}
The next step,
    after having built a field-solver or simulator
    (finds $x$ given $p$) for our wave equation,
    is to build a structure-solver for it.
In other words, we need to be able to solve for $p$ given $x$.

To do so, we return to \ER{wave}
    and remark that 
    $\epsilon^{-1} (\curl H) = (\curl H) \epsilon^{-1}$ and
    $\epsilon^{-1} J = J \epsilon^{-1}$ 
    since scalar multiplication is communicative.
This allows us to rearrange \ER{wave} as
\BE \curl (\curl H) \epsilon^{-1} - \curl J \epsilon^{-1}  = \mu_0 \omega^2 H  \EE
which we now write as 
\BE B(x)p = d(x), \label{eq:Bd} \EE 
    where
\BE B(x) = \curl (\curl H) - \curl J\EE
    and 
\BE d(x)  = \mu_0 \omega^2 H.  \EE

Solving this system would now seem to allow us
    to choose an electromagnetic field
    and then find the structure to produce it;
    which strongly suggests that it will be useful in
    the design of nanophotonic devices.

In terms of computational complexity, as with \ER{Ab}, 
    \ER{Bd} in its current form can be solved using standard tools.

\subsection{Bi-linearity of the wave equation}
Although additional mathematical machinery must still be added
    in order to get a useful design tool,
    we have really shown so far is that the wave equation is 
    separately linear or \emph{bi-linear} in $x$ and $p$.
Namely that,
\BE A(p)x-b(p) = B(x)p - d(x). \EE
In other words, fixing $p$ makes solving the wave equation for $x$
    a linear problem, and vice versa.
Note that the joint problem,
    where both $x$ and $p$ are allowed to vary,
    is not linear.

The bi-linearity of the wave equation
    is fundamental in our objective-first strategy
    which relies on the fact
    that we already know how to solve linear systems well,
    and is the reason why we chose $\epsilon^{-1} \to p$
    instead of the more natual $\epsilon \to p$.
Indeed, this property forms a natural division of labor
    in the objective-first scheme, which we outline below.



\section{The objective-first design problem}
We now build off of the field-solver and the structure-solver,
    as previously outlined,
    by formulating the design problem
    and outlining the objective-first strategy.

\subsection{Design objectives}
A design objective, $f(x)$, is simply defined as
    a function we wish to be minimal 
    for the design to be produced.

For instance, in the design of a device
    which must transmit efficiently into a particular mode,
    we could choose $f(x)$ to be the negative power transmitted into that mode.
Or, if the device was to be a low-loss resonator,
    we could choose $f(x)$ to be the amount of power leaking 
    out of the device.

In general, there are multiple choices of $f(x)$
    which can be used to describe the same objective.
For example, $f(x)$ for a transmissive device 
    may not only be the negative power transmitted into the desired output mode,
    but it could also be the amount of power lost to other modes,
    or even the error in the field values at the output port
    relative to the field values needed for perfect transmission.
These design objectives are equivalent in the sense that, if minimized, 
    all would produce structures with good performance.
At the same time, we must consider that the computational cost and complexity
    of using one $f(x)$ over another may indeed vary greatly.

\subsection{Convexity}
Before formulating the design problem,
    we would like to inject a note regarding the complexity of various 
    optimization problems.

Specifically, we want to introduce the notion of \emph{convexity} %TODO: ref.
    and to simply note the difference between problems that
    are convex and those which are not.
The difference is simply this:
    convex problems have a single optimum point
    (only one local optimum, which is therefore the global optimum)
    which we can reliably find using existing numerical software,
    whereas non-convex problems typically have multiple optima 
    and are thus much more difficult to reliably solve.

That a convex problem can be reliably solved, in this case, 
    means that regardless of the starting guess,
    convex optimization software will 
    always arrive at the globally optimal solution
    and will be able to numerically prove global optimality as well.
Thus, formulating a design problem in terms of convex optimization problems
    virtually eliminates any ideas of chance or randomness.

For the examples presented in this chapter,
    we use CVX, a convex optimization software written for Matlab. % Ref.


\subsection{Typical design formulation}
The typical, and most straightforward formulation
    of the design problem is
\BA \minimize{x,p} f(x) \\
    \subto A(p)x - b(p) = 0, \label{eq:typ:eqcon} \EA
    which states that we would like to vary $x$ and $p$ simultaneously
    in order to decrease $f(x)$
    while always satisfying physics (the electromagnetic wave equation).

Such a formulation, if solved using a steepest-descent method,
    is the well-known adjoint optimization method. % Ref.

\subsection{Objective-first design formulation}
In contrast, the objective-first formulation is 
\BA \minimize{x,p} \| A(p) x - b(p) \|^2 \\
    \subto f(x) = f_\text{ideal}, \EA
    where the roles of the wave equation and the design objective are switched.


This is a bi-convex problem, which we solve using an alternating directions method.

\subsection{Field sub-problem}
\BA \minimize{x} \| A(p) x - b(p) \|^2 \\
    \subto f(x) = f_\text{ideal} \EA

\subsection{Structure sub-problem}
\BA \minimize{p} \| B(x) p - d(x) \|^2 \\
    \subto 0 \le p \le 1 \EA


\section{1D resonator design}
We now build up to the objective-first formulation by example. 
And from a different perspective.

We're going to take a step back and start with the most naive inverse design strategy possible.
Then we'll end with objective first.
This motivates the need for the alternating directions and objective-first strategy.


\subsection{Direct solve of structure}
This is the simplest thing you can do. Basically, set $x$ and solve for $p$.
Also, take out the relaxed constraint on $p$.
\BA \minimize{p} \| B(x) p - d(x) \|^2 \EA
\myfig{1d/leastsquares}{test}
We perfectly satisfy the field but...

\subsection{Regularized solve of structure}
We add a term to try to control $p$
\BA \minimize{p} \| B(x) p - d(x) \|^2 + \eta \|p - p_0\|^2 \EA
\myfig{1d/regularized}{test}
Better, but a trade-off between field accuracy and structure variation.

\subsection{Alternating directions solve}
Now we actually do both fields.
\BA \minimize{p} \| B(x) p - d(x) \|^2 + \eta_1 \|p - p_\text{prev}\|^2 \EA

\BA \minimize{x} \| A(p) x - b(p) \|^2 + \eta_2 \|x - x_\text{prev}\|^2 \EA
\myfig{1d/complementary}{test}

\subsection{Alternating directions solve with bounded $p$}
\BA \minimize{p} \| B(x) p - d(x) \|^2 \\
    \subto 0 \le p \le 1 \EA

\BA \minimize{x} \| A(p) x - b(p) \|^2 + \eta_2 \|x - x_\text{prev}\|^2 \EA
\myfig{1d/bounded}{test}



% RESUME DOCUMENT STYLE -- Released 23 Nov 1989
%    for LaTeX version 2.09
% Copyright (C) 1988,1989 by Michael DeCorte

\typeout{Document Style `res' <26 Sep 89>.}

%%%%%%%%%%%%%%%%%%%%%%%%%%%%%%%%%%%%%%%%%%%%%%%%%%%%%%%%%%%%
% res.sty
%
% \documentstyle{res}
%
% Copyright (c) 1988 by Michael DeCorte
% Permission to copy all or part of this work is granted, provided
% that the copies are not made or distributed for resale, and that
% the copyright notice and this notice are retained.
%
% THIS WORK IS PROVIDED ON AN "AS IS" BASIS.  THE AUTHOR PROVIDES NO
% WARRANTY WHATSOEVER, EITHER EXPRESS OR IMPLIED, REGARDING THE WORK,
% INCLUDING WARRANTIES WITH RESPECT TO ITS MERCHANTABILITY OR FITNESS
% FOR ANY PARTICULAR PURPOSE.
%
% If you make any improvements, I'd like to hear about them.
%
% Michael DeCorte
% P.O. Box 652
% Potsdam NY 13676
% mrd@sun.soe.clarkson.edu
% mrd@clutx.bitnet
%
% Changes for LaTeX2e -- Venkat Krishnamurthy (Aug 7, 2001)
%
%%%%%%%%%%%%%%%%%%%%%%%%%%%%%%%%%%%%%%%%%%%%%%%%%%%%%%%%%%%%
% You can have multiple style options the legal options ones are:
%
%   centered	the name and address are centered at the top of the
%		page (default)
%
%   line	the name is the left with a horizontal line then 
%		the address to the right
%
%   overlapped	the section titles overlap the body text (default)
%
%   margin	the section titles are to the left of the body text
%		
%   11pt	use 11 point fonts instead of 10 point fonts
%
%   12pt	use 12 point fonts instead of 10 point fonts
%
%%%%%%%%%%%%%%%%%%%%%%%%%%%%%%%%%%%%%%%%%%%%%%%%%%%%%%%%%%%%%
%   Commands
%%%%%%%%%%%%%%%%%%%%%%%%%%%%%%%%%%%%%%%%%%%%%%%%%%%%%%%%%%%%%
%   \Resume	prints the word resume but typeset nicely
%
%   \newsectionwidth{dimen}
%		defines the amount of space the labels extend
%		into the left margin.
%		DO NOT TRY to change any of the dimensions
%		yourself.  You will probably confuse the style file.
%
%   \name{text} defines your name
%
%   \address{text}
%		defines your address
%		this can be called twice if you have two addresses
%		use \\'s to indicate where either line breaks or
%		comas should go
%
%   \opening	this prints your name and address at that spot
%		this is not normally needed, as \begin{resume}
%		does this but is provided just in case you need
%		to do something odd
%
%   \begin{resume} ... \end{resume}
%		all of the resume should go inside of this
%		environment
%
%   \section{text}
%		This prints 'text' in the left hand margin.
%		Its exact placement depends on what the style 
%		options has been set to. (overlapped or margin)
%		You should use \\ to start a new line.	If the
%		style option is margin, the \\ is converted
%		to a space.
%		To use this in any of the list environments, put
%		the \section after the \item[] but before the 
%		text.
%		Eg.
%		\begin{itemize}
%		\item\section{text}
%		text
%		\end{itemize}
%
%   \begin{ncolumn}{n} ... \end{ncolumn}
%		creates a tabular environment with n equally
%		spaced columns.  Separate columns by & and
%		end them with \\
%
%   \begin{position} ... \end{position}
%		this is used to print a job description.  There should
%		be only one job description in it.  Information
%		related to the job (such as title, dates...) will
%		be printed.
%
%   \begin{format} ... \end{format}
%		used to change the default format for the position
%		environment.  Within it the recognized commands are:
%		\title{option}
%		\employer{option}
%		\location{option}
%		\dates{option}
%		\body
%		\\
%		where option is one of l,r,c standing for left, right, center.
%		The format will eventually be used to make several
%		tabular environments and you are defining the number of columns
%		and the placement of text within the columns of the tabulars.
%		Each row is terminated by a \\.  Any number of options can 
%		be on a line, they will each be set in their own columns.
%		Any of the options except \body may be left out.
%
%		Eg.
%		\begin{format}
%		\title{l}\employer{r}\\
%		\dates{r}\\
%		\body\\
%		\location{l}\\
%		\end{format}
%
%		In this example the title and employer information
%		are set in 2 columns left justified and right justified
%		respectively.  Then the date is set right justified.
%		Then the body is set.  Then the location is set left
%		justified.
%
% \employer{text}
% \title{text}
% \dates{text}
% \location{text}
%		declare text for the next invocation of the position
%		environment
%
%%%%%%%%%%%%%%%%%%%%%%%%%%%%%%%%%%%%%%%%%%%%%%%%%%%%%%%%%%%%
% Glue
%%%%%%%%%%%%%%%%%%%%%%%%%%%%%%%%%%%%%%%%%%%%%%%%%%%%%%%%%%%%
%
% sectionskip	the amount of horizontal before a section
%
%%%%%%%%%%%%%%%%%%%%%%%%%%%%%%%%%%%%%%%%%%%%%%%%%%%%%%%%%%%%
% Dimensions
%%%%%%%%%%%%%%%%%%%%%%%%%%%%%%%%%%%%%%%%%%%%%%%%%%%%%%%%%%%%
%
% sectionwidth	the amount that the section titles go in the
%		left margin
%
% resumewidth	the width of the total resume from the left
%		margin to the right.  Don't use textwidth
%
%%%%%%%%%%%%%%%%%%%%%%%%%%%%%%%%%%%%%%%%%%%%%%%%%%%%%%%%%%%%
% Definitions
%%%%%%%%%%%%%%%%%%%%%%%%%%%%%%%%%%%%%%%%%%%%%%%%%%%%%%%%%%%%
%
% sectionfont	the font used to print section titles
%		use \renewcommand to change it
%
% namefont	the font used to print your name
%		use \renewcommand to change it
%
%%%%%%%%%%%%%%%%%%%%%%%%%%%%%%%%%%%%%%%%%%%%%%%%%%%%%%%%%%%%
% THINGS TO DO
%
% add lm,rm options to format style to allow things to be placed
% in the left or right margin respectivly
%
% add capability so that \body doesn't have to be proceeded (followed)
% by a \\ allowing part of the description (eg. location) to be the
% first (last) thing of the body
%
% clean up the list building procedures
%
% write docs to tell people how to use this

\NeedsTeXFormat{LaTeX2e}[1995/12/01]
\ProvidesClass{res}[2000/05/19 v1.4b Resume class]

%\DeclareOption{11pt}{\renewcommand\@ptsize{1}}
%\DeclareOption{12pt}{\renewcommand\@ptsize{2}}

\PassOptionsToClass{11pt,12pt}{article}
\LoadClassWithOptions{article}

\newif\if@line
\newif\if@margin

\DeclareOption{line}{\@linetrue}
\DeclareOption{centered}{\@linefalse}

\DeclareOption{margin}{\@margintrue}
\DeclareOption{overlapped}{\@marginfalse}

\ExecuteOptions{overlapped,centered}
\ProcessOptions\relax

\nofiles	     % resume's don't need .aux files


\newtoks\tabular@text		    % holds the current list being processed
\newtoks\tabular@head		    % holds the head tabular list
\newtoks\tabular@tail		    % holds the tail tabular list
\newtoks\@ta			    % used by \@append
\newtoks\undefined@token\undefined@token={}

%%%%%%%%%%%%%%%%%%%%%%%%%%%%%%%%%%%%%%%%%%%%%%%%%%%%%%%%%%%%
% prints a centered name with the address centered
% or the two address on opposite sides of the page
%
\def\@printcentername{\begingroup
  % print the name centered
  \leavevmode\hbox to \textwidth{\hfil\@tablebox{\namefont\@name}\hfil}\par
  \@ifundefined{@addressone}{%
    % do nothing
  }{%
    \@ifundefined{@addresstwo}{
      % only one address
      \leavevmode\hbox to \textwidth{\hfil\@tablebox{\@addressone}\hfil}\par
    }{
      % two addresses
      \leavevmode\hbox to \textwidth{\@tablebox{\@addressone}\hfil
				     \@tablebox{\@addresstwo}}\par
    }%
  }%
\endgroup}

%%%%%%%%%%%%%%%%%%%%%%%%%%%%%%%%%%%%%%%%%%%%%%%%%%%%%%%%%%%%
% this is used to print the name and address at the top of
% the page with a hline in between
%
\def\namefont{\large\bf}
\def\@linename{\begingroup
  \def\\{, }
  {\namefont\@name}
  \vskip 2pt
  \fullline
  \vskip 2pt
  % where do you live?
  \@ifundefined{@addressone}{%
    % do nothing
  }{%
    \leavevmode\hbox to \textwidth
      {\hfill\vbox{\hbox{\@addressone}
		   \hbox{\@addresstwo}
		  }%
      }\par
  }
\endgroup}

%%%%%%%%%%%%%%%%%%%%%%%%%%%%%%%%%%%%%%%%%%%%%%%%%%%%%%%%%%%%
% HEADINGS:
% There are two types of headings:
% 1) one with the name centered and the address centered or
%    in the left and right side if there are two address
% 2) one where the name is in the upper left corner 
%    the a line accross the paper
%    then the address all on one line in the right corner
%    the second address will be directly below the first if defined
%
\let\print@name\relax
\def\ds@centered{\ifx\print@name\relax\let\print@name\@printcentername\fi}
\def\ds@line{\ifx\print@name\relax\let\print@name\@linename\fi}


%%%%%%%%%%%%%%%%%%%%%%%%%%%%%%%%%%%%%%%%%%%%%%%%%%%%%%%%%%%%
% Use this to set the sectionwidth.
% It adjust the width of the text as well as the hoffset
% You probably shouldn't touch any of the size paramaters
% unless you really understand all of this but it is not
% hard.  Either way, it can only be executed once
%
\def\sectionfont{\bf}
\newdimen\sectionwidth
\newskip\sectionskip
\newdimen\resumewidth

\resumewidth=6.5in
\sectionskip=3.5ex plus 1ex minus -.2ex % values stolen from LaTeX

\def\newsectionwidth#1{%
		     \sectionwidth=#1
		     \textwidth=\resumewidth
		     \advance\textwidth-\sectionwidth
		     \hsize=\textwidth
		     \hoffset=\sectionwidth
}


%%%%%%%%%%%%%%%%%%%%%%%%%%%%%%%%%%%%%%%%%%%%%%%%%%%%%%%%%%%%%
% This is for sectiontitles that are entirely in the left margin.
% multiline sectiontitles are permited and will be broken by
% \TeX{} to fit into a box \verb|\sectionwidth| wide.  It is advised
% that \verb|\\| be used to break the lines by hand as \TeX{} will
% probably not do what you want.
%
% When using this with {\tt tabbing} and {\tt list} (or anything
% that is made out of {\tt list}) put the \section's inside of
% the \verb|\begin{}|  and the \verb|\item|Eg.
% \begin{verbatim}
% \begin{trivlist}
% \item[]
% \section{foo}
% text
% \end{trivlist}
% \end{verbatim}
%
\def\boxed@sectiontitle#1{%
  % this macro may be called in a tabular.  Special code must be written
  % to accomodate this.  In LaTeX, a tabular is made out of hboxes.
  % TeX never goes into horizontal mode because of this; it only
  % gets into vertical mode and restricted horizontal mode.  Certain 
  % indenting problems must be handled because of this.  They
  % are delt with at the end of this routine.
  % It is also necessary to close the hbox that was created before
  % the \section and create a new one when this macro has ended.
  % This macro therefore simulates a \kill, so that any text before
  % the \section not be printed.  The proper use is therefore
  % ...
  % text\\
  % \section{foo}
  % more text\\
  % ...
  \ifx\\\@tabcr    % is this in a tabular? (this *should* work but is a cludge)
    \@stopfield    % the is the first part of a \kill
   \else
     \@@par	     % This will end the previous paragraph if needed and
		   % go into vertical mode.  If this was already in
		   % vertical mode then the \par does nothing.

  \fi
  \begingroup
  \everypar={}%
  \def\par{\@@par}%
  \let\\=\@normalcr
  \addpenalty{\@secpenalty} % this would be a good place for a page break
			    % \@secpenalty is what LaTeX uses before its
			    % section's.  It happens to be -300
  \addvspace{\sectionskip}  % put in a bit of glue
  % The following hbox will be contributed to the page list without going
  % into horizontal mode.  Therefore, any \parindent's, \parshape's, \leftskip
  % will be ignored but \hoffset's are not.  The result is that the box will
  % only by \hoffset. This is what I want
  \hbox to 0pt{%
    \hss	 % this is an llap.  In other words, this glue
		 % will shrink by the width of the stuff in the vbox
		 % (\sectionwidth) into the left margin and then
		 % insert the contents of the vbox.
    \vtop to 0pt{% make a 0pt height paragraph, with the baseline at the
		 % lined up with the baseline of the first box in the list
      \leftskip=0pt
      \hsize=\sectionwidth
      \textwidth=\sectionwidth
      \raggedright     % you don't want this filled out to the right margin
      \sectionfont
      #1\vss	       % Go into horizontal mode; do the paragraph;
		       % go into vertical mode; add some negative glue 
		       % to give a box of 0pt height and depth
    }%
  }%
  \addpenalty{-\@secpenalty} % this would be a bad place for a page break
  \vskip-\baselineskip % when the next box is processed, baselineskip glue
		       % glue will be added (the box has no depth because of
		       % the \vss; therefore, we don't have to worry about
		       % \lineskiplimit).  This -\baselineskip glue
		       % is to undo this.  \nointerlineskip doesn't
		       % work because the baseline of this line would be lined
		       % up with the top of the top of the next box.  We
		       % want the baselines lined up.
		       %
		       % It may have been possible to do this by forcing the
		       % baseline of this box to be the top of the box but
		       % then the interline skip between this box and the
		       % previous box would be off as the baselines of the
		       % this box (the one that being made above) and the
		       % previous line would be separated by \baselineskip
		       % (probably, it may be separted by the depth of the
		       % previous box + \lineskip) but as the baseline of
		       % this box has been moved to the top, the box's would
		       % separted by to much glue.  The exact amount being
		       % the height of this box.
  \endgroup
  \ifx\\\@tabcr   % is this in a tabular? (this *should* work but is a cludge)
    % this is the second part of the \kill; it starts the next tabbing line
    % Because this routine will never get into paragraph mode when used in
    % tabbing the \parskip that is discussed below will never be inserted.
    % Therefore it should not be negated as done below.
    \@startline
    \ignorespaces
  \else
    \vskip-\parskip    % The next thing to be contributed will be a paragraph.
		       % Right before being contributed though a \vskip\parskip
		       % will be inserted.  This is to negate it.
		       %
		       % I do consider this to be a bit of a cludge but
		       % I can not find a way to write \unskipfutureskip
		       % or a way to make TeX think that nothing has
		       % been contributed to the page list.
  \fi
}


%%%%%%%%%%%%%%%%%%%%%%%%%%%%%%%%%%%%%%%%%%%%%%%%%%%%%%%%%%%%
% This is for sectiontitles that are entirely above the section text
%
\def\unboxed@sectiontitle#1{%
  \ifx\\\@tabcr % see boxed@sectiontitle for explation
    \@stopfield
   \else
     \@@par
  \fi
  \begingroup
  \everypar={}%
  \def\par{\@@par}%
  \def\\{ }
  \addpenalty{\@secpenalty}
  \addvspace{\sectionskip}
  \hbox to 0pt{\hss\hbox to \sectionwidth{\sectionfont#1\hss}}
  \addpenalty{-\@secpenalty} % this would be a bad place for a page break
  \endgroup
  \ifx\\\@tabcr   % see boxed@sectiontitle for explation
    \@startline
  \else
    \vskip-\parskip
  \fi
  \ignorespaces
}

%%%%%%%%%%%%%%%%%%%%%%%%%%%%%%%%%%%%%%%%%%%%%%%%%%%%%%%%%%%%
% There are two types of section headings:
% 1) the section heading is all on one line and directly
%    below it, is the body of the text
% 2) the section heading is entirely in the left margin
%    (possibly taking multiple lines) with the body of
%    the text next to it
%
\let\@@section\relax
\def\ds@overlapped{\ifx\@@section\relax\newsectionwidth{0.5in}\let
  \@@section\unboxed@sectiontitle\fi}
\def\ds@margin{\ifx\@@section\relax\newsectionwidth{1.3in}\let
  \@@section\boxed@sectiontitle\fi}

%%%%%%%%%%%%%%%%%%%%%%%%%%%%%%%%%%%%%%%%%%%%%%%%%%%%%%%%%%%%
% DEFAULTS: (some of them)
%
% centered name
% overlapped section titles
%
% format is:
%    title	 employer
%    location	 dates
%    body
% with everything in the left of its column

%\input article.sty

\if@line\ds@line\else\ds@centered\fi
\if@margin\ds@margin\else\ds@overlapped\fi


%%%%%%%%%%%%%%%%%%%%%%%%%%%%%%%%%%%%%%%%%%%%%%%%%%%%%%%%%%%%%%%%%%%%%%
% typeset resume all nice and pretty
%
\def\Resume{{R\'{e}sum\'{e}}}

%%%%%%%%%%%%%%%%%%%%%%%%%%%%%%%%%%%%%%%%%%%%%%%%%%%%%%%%%%%%
% makes a line of width \textwidth starting at -\hoffset
%
\def\fullline{		% hrules only listen to \hoffset
  \nointerlineskip	% so I have this code	  
  \moveleft\hoffset\vbox{\hrule width\textwidth} 
  \nointerlineskip
}

%%%%%%%%%%%%%%%%%%%%%%%%%%%%%%%%%%%%%%%%%%%%%%%%%%%%%%%%%%%%
% create a multiline box.
%
\def\@tablebox#1{\begin{tabular}[t]{@{}l@{\extracolsep{\fill}}}#1\end{tabular}}

%%%%%%%%%%%%%%%%%%%%%%%%%%%%%%%%%%%%%%%%%%%%%%%%%%%%%%%%%%%%
% use this to define your name
%
\def\name#1{\def\@name{#1}}

\def\@name{}

%%%%%%%%%%%%%%%%%%%%%%%%%%%%%%%%%%%%%%%%%%%%%%%%%%%%%%%%%%%%%
% use this to define your address, this may be called more than once.
%
\let\@addressone\relax
\let\@addresstwo\relax

\def\address#1{
  \@ifundefined{@addressone}{\def\@addressone{#1}}{\def\@addresstwo{#1}}}

%%%%%%%%%%%%%%%%%%%%%%%%%%%%%%%%%%%%%%%%%%%%%%%%%%%%%%%%%%%%%
% if you want to print your name and address is a slightly
% different format than sugessted, then this can be used
% to place it exactly where you want
%
\def\opening{\def\@opening{}
  \begingroup
  \leftskip=-\hoffset	     % I use leftskip to move things to the left as 
  \advance\textwidth\hoffset % changing hoffset doesn't work. But this
  \hsize=\textwidth	     % doesn't really work as hboxes are rules
			     % are unaffeted
  \let\par=\@@par
  \parindent=0pt
  \parskip=0pt
  \print@name
  \endgroup
}

%%%%%%%%%%%%%%%%%%%%%%%%%%%%%%%%%%%%%%%%%%%%%%%%%%%%%%%%%%%%
% all of the resume goes in the resume environment
%
\newenvironment{resume}{\begingroup
		       \@ifundefined{@opening}{\opening}{}
}{\endgroup}

%%%%%%%%%%%%%%%%%%%%%%%%%%%%%%%%%%%%%%%%%%%%%%%%%%%%%%%%%%%%
% gives you a tabular environment with n equally spaced columns
% \begin{ncolumn}{#} ... \end{ncolumn}
%
% The p option of LaTeX is broken in all but the newest verion
% of latex.tex, this is how to fix it
%
\def\@endpbox{\par\egroup\hfil}
\let\@@endpbox=\@endpbox

\newdimen\@columnwidth	  % the width of each column equal to
\def\ncolumn#1{%
  % \@columnwidth = \textwidth / #1
  \@columnwidth=\textwidth \divide\@columnwidth by #1
  \begin{tabular*}{\textwidth}[t]%
	{*{#1}{@{}p{\@columnwidth}@{\extracolsep{\fill}}}}
}

\def\endncolumn{\end{tabular*}}


%%%%%%%%%%%%%%%%%%%%%%%%%%%%%%%%%%%%%%%%%%%%%%%%%%%%%%%%%%%%
%   \employer{text} defines employer to be text
%   \location{text} defines location to be text
%   \dates{text}    defines dates    to be text
%   \title{text}    defines title    to be text
%   \body
%

\def\employer#1{\def\@employer{\print@employer{#1}}}
\def\location#1{\def\@location{\print@location{#1}}}
\def\dates#1{\def\@dates{\print@dates{#1}}}
\def\title#1{\def\@title{\print@title{#1}}}

\let\l@justify\raggedright
\let\r@justify\raggedleft
\let\c@justify\centering


%%%%%%%%%%%%%%%%%%%%%%%%%%%%%%%%%%%%%%%%%%%%%%%%%%%%%%%%%%%%
% \@format{name}{justify}
% will define \print@#1 to print it's one argument 
% justified according to #2 which can be
%	  l = left
%	  r = right
%	  c = center
%
% eg.
%    \@format{employer}{c}
%    is the same as \def\print@employer#1{{\centering #1\par}}
%
\def\@format#1#2{%
  \expandafter\gdef\csname print@#1\endcsname##1{%
    {\csname#2@justify\endcsname##1\par}}
}

%%%%%%%%%%%%%%%%%%%%%%%%%%%%%%%%%%%%%%%%%%%%%%%%%%%%%%%%%%%%
% this is used to define how the position environment should
% formated.
%
% \begin{format} positioning text \end{format}
% where positioning text may be
%  \employer{pos}
%  \location{pos}
%  \dates{pos}
%  \title{pos}
%  \body	    (for the body of the position environment)
%   where pos is 
%     l for left
%     r for right
%     c for center
% use \\ to break the line
% you don't have to use all of the options.
% on any one line, you should indicate what you want on that line
% and where it should go within its column.
% eg.
% the following prints the employer in the left with the location
% centered within that its column.  It then prints the date on the
% right.  Then it prints the body of the position environment. Then
% it prints the title centered within its column; as there is only
% one column here it is really just centered.
%
% \begin{format}
% \employer{l}\location{c}\\
% \dates{r}\\
% \body\\
% \title{c}\\
% \end{format}

\newcounter{numberofcolumns}
\newenvironment{format}{%
  \def\end@line@head{\append@tabular@head{tabular@text}\tabular@text={}%
    \c@numberofcolumns 0}
  \def\end@line@tail{\append@tabular@tail{tabular@text}\tabular@text={}%
    \c@numberofcolumns 0}
  \tabular@text={}
  \tabular@head={}
  \tabular@tail={}
  \c@numberofcolumns 0
  \let\\=\end@line@head
  \def\employer##1{\advance\c@numberofcolumns 1
		   \@format{employer}{##1}
		   \append@tabular@text{employer}}
  \def\location##1{\advance\c@numberofcolumns 1
		   \@format{location}{##1}
		   \append@tabular@text{location}}
  \def\dates##1{\advance\c@numberofcolumns 1
		\@format{dates}{##1}
		\append@tabular@text{dates}}
  \def\title##1{\advance\c@numberofcolumns 1
		 \@format{title}{##1}
		\append@tabular@text{title}}
  \def\body{\iftoks\tabular@head\undefined@token\then
	    \else
	      \@append{\noexpand\\}\to\tabular@head
	    \skotfi
	    \let\\=\end@line@tail}}{}

%%%%%%%%%%%%%%%%%%%%%%%%%%%%%%%%%%%%%%%%%%%%%%%%%%%%%%%%%%%%
%taken from page 378 of TeXbook but freely hacked
%
% appends the expansion of #1 to the token list #2

\def\@append#1\to#2{%
  \@ta=\expandafter{#1}%
  \xdef\@append@temp{\the#2\the\@ta}
  \global#2=\expandafter{\@append@temp}%
}


%%%%%%%%%%%%%%%%%%%%%%%%%%%%%%%%%%%%%%%%%%%%%%%%%%%%%%%%%%%%
% CHAA006%vaxb.rhbnc.ac.uk@NSS.Cs.Ucl.AC.UK
% texhax.88.078
% is used to see if two token lists are equal
% there must be a better way
%

\let \then = \empty
\def \iftoks #1#2\then #3\else #4\skotfi{
		\edef \1{\the #1}
		\edef \2{\the #2}
		\ifx \1\2\then #3\else #4\fi}


%%%%%%%%%%%%%%%%%%%%%%%%%%%%%%%%%%%%%%%%%%%%%%%%%%%%%%%%%%%%
% \append@tabular@text{command}
%
% appends command to the end of \tabular@text.
% NOTE: command MUST be a command but without the \
%	Eg. \append@tabular@text{relax}
%
% used to define \tabular@text for the tabular environment
% used by append@tabular@head and append@tabular@tail
%

\def\append@tabular@text#1{%
\iftoks\tabular@text\undefined@token\then
  \global\tabular@text=\expandafter{\csname @#1\endcsname}
\else
  \@append{&}\to\tabular@text
  \@append{\csname @#1\endcsname}\to\tabular@text
\skotfi
}

%%%%%%%%%%%%%%%%%%%%%%%%%%%%%%%%%%%%%%%%%%%%%%%%%%%%%%%%%%%%
% append@tabular@head
%
% appends command to the end of \tabular@text@head
% NOTE: command MUST be a command but without the \
%	Eg. \append@tabular@head{relax}
%
% used to define \tabular@head for the tabular environment
% used by the position environment
%
\def\append@tabular@head#1{%
  \ifnum\the\c@numberofcolumns=0\relax
  \else
    \iftoks\tabular@head\undefined@token\then
      \relax
    \else
      \@append{\noexpand\\}\to\tabular@head
      \@append{\noexpand\penalty-\@secpenalty}\to\tabular@head
    \skotfi
    \@append{\noexpand\begin{ncolumn}}\to\tabular@head
    \@append{\expandafter{\the\c@numberofcolumns}}\to\tabular@head
    \@append{\the\csname#1\endcsname}\to\tabular@head
    \@append{\noexpand\end{ncolumn}}\to\tabular@head
  \fi
}

%%%%%%%%%%%%%%%%%%%%%%%%%%%%%%%%%%%%%%%%%%%%%%%%%%%%%%%%%%%%
% append@tabular@tail
%
% appends command to the end of \tabular@text@htail
% NOTE: command MUST be a command but without the \
%	Eg. \append@tabular@tail{relax}
%
% used to define \tabular@tail for the tabular environment
% used by the position environment
%
\def\append@tabular@tail#1{%
  \ifnum\the\c@numberofcolumns=0\relax
  \else
    \iftoks\tabular@tail\undefined@token\then
    \else
      \@append{\noexpand\\}\to\tabular@tail
      \@append{\noexpand\penalty-\@secpenalty}\to\tabular@tail
    \skotfi
    \@append{\noexpand\begin{ncolumn}}\to\tabular@tail
    \@append{\expandafter{\the\c@numberofcolumns}}\to\tabular@tail
    \@append{\the\csname#1\endcsname}\to\tabular@tail
    \@append{\noexpand\end{ncolumn}}\to\tabular@tail
  \fi
}

%%%%%%%%%%%%%%%%%%%%%%%%%%%%%%%%%%%%%%%%%%%%%%%%%%
% put the actual job descriptions here
% \begin{postion} ... \end{position}
% in the ... describe the position.
% don't put the \dates \location etc in here. define them before hand
\newenvironment{position}%
  {%
   \begingroup
     \par
       \the\tabular@head
%     \addpenalty{-\@secpenalty}% bad place for a page break
     \penalty -\@secpenalty % bad place for a page break
     \penalty 10000
     \ignorespaces
  }{%
       \the\tabular@tail
%     \addpenalty{\@secpenalty}% good place for a page break
     \penalty \@secpenalty % good place for a page break
    \endgroup
}

%%%%%%%%%%%%%%%%%%%%%%%%%%%%%%%%%%%%%%%%%%%%%%%%%%%%%%%%%%%%
% DEFAULTS: (the rest of them)
%
% centered name
% overlapped section titles
%
% format is:
%    title	 employer
%    location	 dates
%    body
% with everything in the left of its column

\@secpenalty = -500
\topmargin 0pt
\headheight 0pt
\headsep 0pt
\textheight 9in
\parindent 0pt
\topmargin 0in
\oddsidemargin 0pt
\evensidemargin 0pt
\marginparwidth 0pt
\parindent 0pt
\parskip \baselineskip
\setcounter{secnumdepth}{0}
\def\@listI{\leftmargin\leftmargini
\topsep 0pt 
\parskip 0pt
\partopsep 2pt plus 2pt
\parsep 2pt plus 2pt
\itemsep \parsep}

\pagestyle{empty}  % don't want page numbers

\begin{format}
\title{l}\employer{r}\\
\location{l}\dates{r}\\
\body\\
\end{format}

\let\section\@@section


\section{Waveguide coupler design}\label{sec:wg}
We first apply the objective-first formulation
    with the alternating directions algorithm
    to the design of nanophotonic waveguide couplers
    in two dimensions,
    where our goal is to couple light from
    a single input waveguide mode
    to a single output waveguide mode
    with as close to unity efficiency as possible.
We would also like to allow the user to choose arbitrary
    input and output waveguides,
    as well as to select 
    arbitrary modes within those waveguides
    (as opposed to allowing only the fundamental mode, for example).

This problem is very general and, in essence,
    encompasses the design of all linear nanophotonic components,
    because the function or performance of all such components
    is simply to convert a defined set of input modes
    into a defined set of output modes.
Such a broad, general problem is ideally suited for 
    an objective-first strategy,
    since no approximations or simplifications
    of the electromagnetic fields are required;
    we only make the simplification of working in two dimensions
    (transverse magnetic mode)
    and dealing only with a single input and output mode.

\subsection{Choice of design objective}
As mentioned in \sR{desobj} multiple equivalent choices
    of design objective exist which should allow one
    to achieve the same device performance;
    however, we will choose, for generality, the following design objective,
\BE f(x) = \begin{cases}
        x - x_\text{perfect} & \text{at boundary}, \\
        0 & \text{elsewhere},
        \end{cases} \EE
That is, $f(x)$ simply selects the outermost values of the field
    in the design space
    and compares them to values of a perfect device.

Furthermore, we choose $f_\text{ideal} = 0$ so that
    when placed into the objective-first problem \ER{ob1},
    this will result in fixing the boundary values of the field
    at the edge of the design space
    to those of an ideal device,
    as shown in \fR{wg/intro}.
In this case, we choose such an ideal device
    to have perfect (unity) coupling efficiency,
and these ideal fields are simply obtained by using
    the input and output mode profiles at the corresponding ports
    and using values of zero at the remaining ports.

\myfig{wg/intro}{Formulation of the design objective.}

Such a design objective is general
    in the sense that the boundary values of the device 
    contain all the information necessary to determine
    how the device will interact with its environment,
    when excited with the input mode in question.
In other words,
    we only need to know the boundary field values,
    and not the interior field values to determine 
    the performance of the device;
    and thus, it would be conceivable that such a scheme
    might be generally applied to linear nanophotonic devices beyond 
    just waveguide mode couplers.

In our case,
    we only need to know the value of $H_z$ and 
    its derivative along the normal direction, $\partial H_z / \partial n$,
    along the design boundary
    in order to completely characterize its performance.
Alternatively,
    one can, of course, use the outermost two layers of the $H_z$
    instead of calculating a spatial derivative.

\subsection{Application of the objective-first strategy}
Having chosen our design objective we apply
    alternating directions to \ER{ob1} which 
    results in solving the following two sub-problems iteratively:
\BA \minimize{x} \| A(p) x - b(p) \|^2 \\
    \subto x = x_\text{perfect} \text{, at boundary} \notag \EA
\BA \minimize{p} \| B(x) p - d(x) \|^2 \\
    \subto p_0 \le p \le p_1 \notag \EA

For the results throughout this chapter, 
    we uniformily choose $p_0 = 1/12.25$ and $p_0 = 1$,
    corresponding to $\epsilon^{-1}$ of silicon and air respectively.
Additionally, since a starting value for $p$ is initially required,
    we always choose to use a uniform value of $p = 1/9$ 
    across the entire design space.
There is nothing really unique about such a choice,
    although we have noticed that initial value of $p$ near 1 
    often result in poor designs.
Note, that, unlike $p$, we do not require an initial guess for $x$.

The only other significant value that needs to be set initially
    is the frequency, or wavelength of light.
We use free space wavelengths in the range of 25 to 63 grid points for
    the results in this chapter.

Lastly, for all the examples presented in the chapter,
    we run the alternating directions algorithm for 400 iterations.
Although we do not present the convergence results here,
    such information can be obtained by inspecting the source code\cite{code}.

% Also, small footprint and computation time need to be mentioned.
\subsection{Coupling to a wide, low-index waveguide}
As a first example, we design a coupler from 
    the fundamental mode of a narrow, high-index waveguide
    to the fundamental mode of a wide, low-index waveguide.
Such a coupler would be useful for coupling from 
    an on-chip nanophotonic waveguide to
    an off-chip fiber for example.

\myfig{wg/1}{Coupler to a wide low-index waveguide.
            Efficiency: 99.8\%, 
            device footprint: $36 \times 76$ grid points, 
            wavelength: 42 grid points.}

The input and output mode profiles used as
    the ideal fields are shown in the upper-left corner of \fR{wg/1}.
The final structure is shown in the upper right plot, and
    the simulated $H_z$ fields,
    under excitation of the input mode in this final structure,
    are shown in the bottom plots.

\FR{wg/1} then shows that the design structure has nearly unity efficiency
    and converts between the input and output modes
    within a very small footprint.

\subsection{Mode converter}
In addition to coupling to a low-index waveguide,
    we show that we can successfully apply the objective-first method
    to convert between modes of a waveguide.
We do this by simply selecting the output mode 
    in the design objective to be the second-order waveguide mode,
    as seen in \fR{wg/2}.
\myfig{wg/2}{Mode converter.
            Efficiency: 98.0\%, 
            device footprint: $36 \times 76$ grid points, 
            wavelength: 42 grid points.}

Note that the design of this coupler is made challenging
    because of the opposite symmetries of the input and output modes.
Moreover, because our initial structure is symmetric,
    we initially have exactly 0\% efficiency to begin with.
Fortunately, the objective-first method can still design
    an efficient coupler in this case as well.

\subsection{Coupling to an air-core waveguide mode}
We can then continue to elucidate the generality of our method
    by coupling between waveguides which confine light 
    in completely different ways.
\myfig{wg/3}{Coupler to a wide low-index waveguide.
            Efficiency: 98.9\%, 
            device footprint: $36 \times 76$ grid points, 
            wavelength: 25 grid points.}

\FR{wg/3} shows  a high-efficiency coupling device between 
    an index-guided input waveguide and
    a ``air-core'' output waveguide, 
    in which the waveguiding effect is achieved using distributed Bragg reflection
    (instead of total internal reflection as in the input waveguide).

\subsection{Coupling to a metal-insulator-metal waveguide }
Additionally, our design method can also generate couplers
    between different material systems such as 
    between dielectric and metallic (plasmonic) waveguides,
    as shown in \fR{wg/4}.
\myfig{wg/4}{Coupler to a plasmonic metla-insulator-metal waveguide.
            Efficiency: 97.5\%, 
            device footprint: $36 \times 76$ grid points, 
            wavelength: 25 grid points.}

In this case, the permittivity of the metal ($\epsilon = -2$) is chosen to be 
    near the plasmonic resonance ($\epsilon = -1$).

\subsection{Coupling to a metal wire plasmonic waveguide mode}
\myfig{wg/5}{Coupler to a plasmonic wire waveguide.
            Efficiency: 99.1\%, 
            device footprint: $36 \times 76$ grid points, 
            wavelength: 25 grid points.}
Lastly, \fR{wg/5} shows that efficiently coupling to a plasmonic wire
    is achievable as well.




\section{Metamaterials design}
\subsection{Modification of the design objective}
\subsection{Cloak devices}
\subsection{Mimic devices}

\section{Extending the method}
\subsection{3D}
\subsection{Multi-mode}
\subsection{Robustness}
\subsection{Binary structure}


\section{Appendix}
\subsection{Full 3D curl}
\subsection{1D}
\subsection{2D}
\subsection{2.5D}
