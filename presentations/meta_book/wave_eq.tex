\section{The electromagnetic wave equation}
\subsection{Physics formulation}
Let's talk about the electromagnetic wave equation from a physics standpoint.
Let's start from Maxwell's equations without currents.
\BA \curl E = - \mu_0 \frac{\partial H}{\partial t} \\
    \curl H = J + \epsilon \frac{\partial E}{\partial t} \EA

\BA \curl E = - i \mu_0 \omega H \\
    \curl H = J + i \epsilon \omega E \EA

\BE \curl \epsilon^{-1} \curl H - \mu_0 \omega^2 H = \curl \epsilon^{-1} J \EE

Talk about reduced dimensions here. Refer reader to appendix.

\subsection{Numerical formulation}
Now let's talk about the electromagnetic wave equation from a computational perspective.

To solve the wave equation on a computer we need to use the Yee grid.

To make things easier we will use weird units.

We also need to take care of boundary conditions.

\subsection{Solving for $H$}
The wave equation from a mathematical perspective.

\BE (\curl \epsilon^{-1} \curl - \mu_0 \omega^2) H = \curl \epsilon^{-1} J \EE

Becomes, with a change of variables.

\BE A(p)x = b(p) \EE 

This can be solved directly in 1D and 2D. 
Special methods needed in 3D. 
We just do 1D and 2D.

\subsection{Solving for $\epsilon^{-1}$}
The wave equation from a optimization perspective.

Because scalar multiplication is transitive 
    ($\epsilon^{-1} (\curl H) = (\curl H) \epsilon^{-1}$ and
    $\epsilon^{-1} J = J \epsilon^{-1}$)
\BE \curl (\curl H) \epsilon^{-1} - \curl J \epsilon^{-1}  = \mu_0 \omega^2 H  \EE

which we write as 

\BE B(x)p = d(x) \EE 

Special constraints... (binary)

\subsection{Insight}
Basically, we see that the electromagnetic wave equation is separably linear in $H$ and $\epsilon^{-1}$

This means that...


