\section{The electromagnetic wave equation}

We start by understanding the underlying physical equation.
We extend it beyond its usual use in simulation, 
    and use it for design.

\subsection{Physics formulation}
First, let's derive our wave equation,
    starting with the differential form of Maxwell's equations, 
\BA \curl E = - \mu_0 \frac{\partial H}{\partial t} \\
    \curl H = J + \epsilon \frac{\partial E}{\partial t}, \EA
    where $E$, $H$, and $J$ are 
    the electric, magnetic and electric current % TODO: Check this.
    vector fields, respectively,
    $\epsilon$ is the permittivity
    and $\mu_0$ is the permeability, which we assume to be 
    that of vacuum everywhere.


For time dependence $\exp(-i \omega t)$, 
    where $\omega$ is the angular frequency,
    these become
\BA \curl E = - i \mu_0 \omega H \\
    \curl H = J + i \epsilon \omega E, \label{eq:H2E} \EA
    which we can combine to form our wave equation,
\BE \curl \epsilon^{-1} \curl H - \mu_0 \omega^2 H = \curl \epsilon^{-1} J. 
    \label{eq:wave} \EE
For further information,
    as well as simplifications to the wave equation in reduced dimensions,
    please see the Appendix. % TODO: Use reference.

\subsection{Numerical formulation}
Now, on top of the analytical formulation of the wave equation \ER{wave}
    we will now add a numerical, or discretized, formulation.
This will be needed in order to solve for arbitrary structures.

First, we discretize our computational space 
    according to the Yee grid, % TODO: Reference!
    which allows us to easily define the curl $(\curl)$
    operators in \ER{wave} as described in the Appendix. % TODO: ref.
This allows us, with a change of variables to formulate \ER{wave} as
\BE A(p)x = b(p), \label{eq:Ab} \EE 
    where $H \to x$, $\epsilon^{-1} \to p$; 
    and where
\BE A(p) = \curl \epsilon^{-1} \curl - \mu_0 \omega^2 \EE 
    and
\BE b(p) = \curl \epsilon^{-1} J. \EE
Note that our use of $A(p)$ and $b(p)$ instead of $A$ and $b$
    simply serves to clarify the dependence
    of both $A$ and $b$ to $p$.

Additionally,  we use periodic boundary conditions
    with stretched-coordinate % TODO: ref.
    perfectly matched layers where necessary for our examples.

\subsection{Solving for $H$}
With our numerical formulation, we can now solve for the $H$-field
    (the $E$-field can be computed from the $H$-field using \ER{H2E})
    by using general linear algebra solvers.
Doing so is also simply known as a time-harmonic or 
    a finite-difference frequency-domain (FDFD) simulation.

Now, while a full three-dimensional problem
    is computationally quite taxing;
    in one- and two-dimensions, 
    \ER{Ab} is easily solved using the standard sparse solver
    included in Matlab,
    and this technique is regularly employed 
    in the examples which follow.


\subsection{Solving for $\epsilon^{-1}$}
The next step,
    after having built a field-solver or simulator
    (finds $x$ given $p$) for our wave equation,
    is to build a structure-solver for it.
In other words, we need to be able to solve for $p$ given $x$.

To do so, we return to \ER{wave}
    and remark that 
    $\epsilon^{-1} (\curl H) = (\curl H) \epsilon^{-1}$ and
    $\epsilon^{-1} J = J \epsilon^{-1}$ 
    since scalar multiplication is communicative.
This allows us to rearrange \ER{wave} as
\BE \curl (\curl H) \epsilon^{-1} - \curl J \epsilon^{-1}  = \mu_0 \omega^2 H  \EE
which we now write as 
\BE B(x)p = d(x), \label{eq:Bd} \EE 
    where
\BE B(x) = \curl (\curl H) - \curl J\EE
    and 
\BE d(x)  = \mu_0 \omega^2 H.  \EE

Solving this system would now seem to allow us
    to choose an electromagnetic field
    and then find the structure to produce it;
    which strongly suggests that it will be useful in
    the design of nanophotonic devices.

In terms of computational complexity, as with \ER{Ab}, 
    \ER{Bd} in its current form can be solved using standard tools.

\subsection{Bi-linearity of the wave equation}
Although additional mathematical machinery must still be added
    in order to get a useful design tool,
    we have really shown so far is that the wave equation is 
    separately linear or \emph{bi-linear} in $x$ and $p$.
Namely that,
\BE A(p)x-b(p) = B(x)p - d(x). \EE
In other words, fixing $p$ makes solving the wave equation for $x$
    a linear problem, and vice versa.
Note that the joint problem,
    where both $x$ and $p$ are allowed to vary,
    is not linear.

The bi-linearity of the wave equation
    is fundamental in our objective-first strategy
    which relies on the fact
    that we already know how to solve linear systems well,
    and is the reason why we chose $\epsilon^{-1} \to p$
    instead of the more natual $\epsilon \to p$.
Indeed, this property forms a natural division of labor
    in the objective-first scheme, which we outline below.

