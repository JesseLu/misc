\section{Waveguide coupler design}\label{sec:wg}
We first apply the objective-first formulation
    with the alternating directions algorithm
    to the design of nanophotonic waveguide couplers
    in two dimensions,
    where our goal is to couple light from
    a single input waveguide mode
    to a single output waveguide mode
    with as close to unity efficiency as possible.
We would also like to allow the user to choose arbitrary
    input and output waveguides,
    as well as to select 
    arbitrary modes within those waveguides
    (as opposed to allowing only the fundamental mode, for example).

This problem is very general and, in essence,
    encompasses the design of all linear nanophotonic components,
    because the function or performance of all such components
    is simply to convert a defined set of input modes
    into a defined set of output modes.
Such a broad, general problem is ideally suited for 
    an objective-first strategy,
    since no approximations or simplifications
    of the electromagnetic fields are required;
    we only make the simplification of working in two dimensions
    and dealing only with a single input and output mode.

Say something about TE mode in 2D. 

\subsection{Choice of design objective}
As mentioned in \sR{desobj} multiple equivalent choices
    of design objective exist which should allow one
    to achieve the same device performance;
    however, we will choose, for generality, the following design objective,
\BE f(x) = \begin{cases}
        x & \text{at boundary}, \\
        0 & \text{elsewhere},
        \end{cases} \EE
That is, $f(x)$ simply selects the outermost values of the field
    in the design space.

When placed into the objective-first problem \ER{ob1},
    this will result in fixing the boundary values of the field
    at the edge of the design space
    to those of an ideal device,
    as shown in \fR{wg/intro}.
In this case, we choose such an ideal device
    to have perfect (unity) coupling efficiency,
\BE f_\text{ideal} = \begin{cases}
        x_\text{perfect} & \text{at boundary}, \\
        0 & \text{elsewhere}.
        \end{cases} \EE
These ideal fields are simply obtained by using
    the input and output mode profiles at the corresponding ports
    and using values of zero at the remaining ports.

\myfig{wg/intro}{Formulation of the design objective.}

Such a design objective is general
    in the sense that the boundary values of the device 
    contain all the information necessary to determine
    how the device will interact with its environment,
    when excited with the input mode in question.
In other words,
    we only need to know the boundary field values,
    and not the interior field values to determine 
    the performance of the device;
    and thus, it would be conceivable that such a scheme
    might be generally applied to linear nanophotonic devices beyond 
    just waveguide mode couplers.

In our case,
    we only need to know the value of $H_z$ and 
    its derivative along the normal direction, $\partial H_z / \partial n$,
    along the design boundary
    in order to completely characterize its performance.
Alternatively,
    one can, of course, use the outermost two layers of the $H_z$
    instead of calculating a spatial derivative.

\subsection{Application of the objective-first strategy}
Having chosen our design objective we apply
    alternating directions to \ER{ob1} which 
    results in solving the following two sub-problems iteratively.
\BA \minimize{x} \| A(p) x - b(p) \|^2 \\
    \subto x = x_\text{perfect} \text{, at boundary} \notag \EA
\BA \minimize{p} \| B(x) p - d(x) \|^2 \\
    \subto p_0 \le p \le p_1 \notag \EA

For the results throughout this chapter, 
    we uniformily choose $p_0 = 1/12.25$ and $p_0 = 1$,
    corresponding to $\epsilon^{-1}$ of silicon and air respectively.
Additionally, since a starting value for $p$ is initially required,
    we always choose to use a uniform value of $p = 1/9$ 
    across the entire design space.
There is nothing really unique about such a choice,
    although we have noticed that initial value of $p$ near 1 
    often result in poor designs.
Note, that, unlike $p$, we do not require an initial guess for $x$.

The only other significant value that needs to be set initially
    is the frequency, or wavelength of light.
We use wavelengths in the range of 25 to 42 grid points for
    the results in this chapter.

Lastly, for all the examples presented in the chapter,
    we run the alternating directions algorithm for 400 iterations.
Although we do not present the convergence results here,
    such information can be obtained by inspecting the source. %Ref!

Also, small footprint and computation time need to be mentioned.
\subsection{Coupling to a wide, low-index waveguide}
As a first example, we design a coupler from 
    the fundamental mode of a narrow, high-index waveguide
    to the fundamental mode of a wide, low-index waveguide.
Such a coupler would be useful for coupling from 
    an on-chip nanophotonic waveguide to
    an off-chip fiber for example.

\myfig{wg/1}{Coupler to a wide low-index waveguide.
            Efficiency: 99.8\%, 
            footprint: $36 \times 76$ grid points, 
            wavelength: 42 grid points.}

The input and output mode profiles used as
    the ideal fields are shown in the upper-left corner of \fR{wg/1}.
The final structure is shown in the upper right plot, and
    the simulated $H_z$ fields,
    under excitation of the input mode,
    are shown in the bottom plots.

\FR{wg/1} then shows that the design structure has nearly unity efficiency
    and converts between the input and output modes
    within a very small footprint.

\subsection{Mode converter}
In addition to coupling to a low-index waveguide,
    we show that we can successfully apply the objective-first method
    to convert between modes of a waveguide.
We do this by simply selecting the output mode 
    in the design objective to be the second-order waveguide mode,
    as seen in \fR{wg/2}.
\myfig{wg/2}{Mode converter.
            Efficiency: 98.0\%, 
            footprint: $36 \times 76$ grid points, 
            wavelength: 42 grid points.}

Note that the design of this coupler is made challenging
    because of the opposite symmetries of the input and output modes.
Moreover, because our initial structure is symmetric,
    we initially have exactly 0\% efficiency to begin with.
Fortunately, the objective-first method can still design
    an efficient coupler in this case as well.

\subsection{Coupling to an air-core waveguide mode}
We can then continue to elucidate the generality of our method
    by coupling between waveguides which confine light 
    in completely different ways.
\myfig{wg/3}{Coupler to a wide low-index waveguide.
            Efficiency: 98.9\%, 
            footprint: $36 \times 76$ grid points, 
            wavelength: 25 grid points.}

\FR{wg/3} shows  a high-efficiency coupling device between 
    an index-guided input waveguide and
    a ``air-core'' output waveguide, 
    in which the waveguiding effect is achieved using distributed Bragg reflection.

\subsection{Coupling to a metal-insulator-metal waveguide }
Additionally, our design method can also generate couplers
    between different material systems such as 
    between dielectric and metallic (plasmonic) waveguides,
    as shown in \fR{wg/4}.
\myfig{wg/4}{Coupler to a wide low-index waveguide.
            Efficiency: 97.5\%, 
            footprint: $36 \times 76$ grid points, 
            wavelength: 25 grid points.}

In this case, the permittivity of the metal ($\epsilon = -2$) is chosen to be 
    near the plasmonic resonance ($\epsilon = -1$).

\subsection{Coupling to a metal wire plasmonic waveguide mode}
\myfig{wg/5}{Coupler to a wide low-index waveguide.
            Efficiency: 99.1\%, 
            footprint: $36 \times 76$ grid points, 
            wavelength: 25 grid points.}
Lastly, \fR{wg/5} shows that efficiently coupling to a plasmonic wire
    is achievable as well.


