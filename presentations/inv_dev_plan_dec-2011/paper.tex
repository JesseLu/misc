\documentclass{article}
\usepackage{graphicx}
\usepackage{amsmath}
\usepackage{hyperref}

\title{Nanophotonic Inverse Design}
\author{Jesse Lu}
\begin{document}
\maketitle
\tableofcontents

\section{Introduction}
Our goal is to produce
    a software package capable of designing virtually any nanophotonic device.
The software must produce designs
    \begin{itemize}
    \item with exceptional device performance,
    \item which are easily manufacturable, and
    \item with computational efficiency.
    \end{itemize}

Such a software package would be extremely useful 
    in designing the components needed to guide light on a computer chip
    such as couplers, filters, absorbers, and multiplexors.

However, the scale of the problem as well as 
    the difficulty of inverting the underlying wave equation
    have been major obstacles in producing successful design algorithms.


\subsection{Problem statement}
We formulate the design problem in the following way,
    \begin{subequations}\begin{align}
    \text{minimize} \quad   & f(x) + g(p) \\
    \text{subject to} \quad & r(x, p) = 0
    \end{align}\label{prob_stat}\end{subequations}
    where
    \begin{itemize}
    \item $x$ is the field variable,
    \item $p$ is the structure variable,
    \item $f(x)$ is the performance objective,
    \item $g(p)$ is the manufacturability objective,
    \item $f(x) + g(p)$ is generally referred to as the design objective, and
    \item $r(x,p)$ is the physics residual for which we use 
            the time-harmonic electromagnetic wave equation,
            \begin{equation}
            r(x,p) = 
                (\nabla \times \epsilon^{-1} \nabla \times - \mu \omega^2) H
                - \text{sources}.
            \end{equation}
    \end{itemize}


\subsection{Key insights}
Generic nonlinear optimization routines are usually unable
    to solve \eqref{prob_stat},
    because there are an extremely large (millions) of variables, 
    and because $r(x,p)$ often results in ill-conditioned matrices.
For this reason, we need to take advantage of key features of the problem.
\begin{itemize}
\item $r(x,p)$ is separably affine (bi-affine) in $x$ and $p$,
    \begin{equation}
    r(x,p) = A(p)x - b(p) = B(x) p - d(x),
    \end{equation}
        this allows us to form two simpler sub-problems.
\item Simulators which compute $A(p)^{-1} z$ are available,
        where $z$ is an arbitrary vector, 
        even for very large systems (millions of variables).
\item Solving $B(x) p - d(x) = 0$ is possible with generic software, 
        because manufacturing processes severely limit
        the degrees of freedom of $p$ 
        (decreasing $p$ to thousands of variables).
\end{itemize}
    


\section{Adjoint method}
The adjoint method is a steepest-descent method
    on the space $r(x,p) = 0$,
    and relies upon the following linear approximations
    of the design objective and the physics residual,
    \begin{subequations}\begin{align}
    (f(x + \Delta x) + g(p + \Delta p)) - (f(x) + g(p)) &\approx
        \nabla f^T \Delta x + \nabla g^T \Delta p \label{d lin} \\ 
    r(x + \Delta x, p + \Delta p) - r(x, p) &\approx
        \nabla_x r^T \Delta x + \nabla_p r^T \Delta p. \label{r lin}
    \end{align}\end{subequations}

Assuming a starting point already satisfying $r(x,p) = 0$,
    we note that \eqref{r lin} must equal zero,
    in order to keep the physics residual at zero.
This allows us to form the following relationship between
    $\Delta x$ and $\Delta p$,
    \begin{subequations}\begin{align}
        A(p) \Delta x + B(x) \Delta p &= A \Delta x + B \Delta p = 0 \\
        \Delta x &= - A^{-1} B \Delta p,
    \end{align}\end{subequations}
    which allows us to write \eqref{d lin} only in terms of $\Delta p$,
    \begin{equation}
    \nabla f^T \Delta x + \nabla g^T \Delta p =
        - (\nabla f^T A^{-1} B - \nabla g) \Delta p.
    \end{equation}
Thus, we see that the steepest-descent direction is
    \begin{equation}
    \Delta p = B^T A^{-T} \nabla f - \nabla g.
    \end{equation}
\subsection{Computational cost}
\subsection{Multi-mode formulation}
\section{Alternating directions method of multipliers (ADMM)}
\subsection{Computational cost of ADMM}
\subsection{Multi-mode ADMM}
\begin{appendix}
\section{Constructing the relevant matrices and vectors}
\section{Solving the matrices}
\end{appendix}
\end{document}
