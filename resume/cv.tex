% LaTeX file for resume 
% This file uses the resume document class (res.cls)

\documentclass{res} 
%\usepackage{helvetica} % uses helvetica postscript font (download helvetica.sty)
%\usepackage{newcent}   % uses new century schoolbook postscript font 
\setlength{\textheight}{9.5in} % increase text height to fit on 1-page 

\begin{document} 

\name{Jesse Lu -- Curriculum Vitae\\[12pt]}
\address{\texttt{jesselu@stanford.edu} \\ 66 Newell Rd. Apt. O \\ East Palo Alto, CA \\ (408) 568-9356}
                                  
\begin{resume}

\section{Education}          
    Stanford University, PhD, Electrical Engineering, June 2013 (in progress) \\
    Stanford University, Masters of Science, Electrical Engineering, May 2012 \\
    University of California Los Angeles, Bachelor of Science, Electrical Engineering, June 2006

\section{Honors}
    Stanford Graduate Fellowship, Stanford University, 2007 \\
    Dean's Honors List, University of California Los Angeles, five quarters

\section{PhD Research: Nanophotonic computational design}
Developed software to 
    design arbitrary linear nanophotonic devices which 
    are fully three-dimensional and multi-modal,
    exhibit novel functionality,
    have very compact footprints,
    exhibit high efficiency, 
    are strongly robust to fluctuations in 
    wavelength, temperature and fabrication error, and
    are manufacturable.
Critically, the developed software 
    does not require the user to be a nanophotonic expert or 
    to perform any manual tuning. 
Instead, devices are designed
    solely based on the user's desired performance specification for the device.

\section{Publications}          
    ``Objective-first design of high-efficiency, small-footprint couplers between arbitrary nanophotonic waveguide modes'' Jesse Lu, Jelena Vuckovic (Optics Express, 2012). \\
    ``Inverse design of a three-dimensional nanophotonic resonator'' Jesse Lu, Stephen Boyd, Jelena Vuckovic (Optics Express, 2011). \\
    ``Inverse design of nanophotonic structures using complementary convex optimization'' Jesse Lu, Jelena Vuckovic (Optics Express, 2010). \\
    ``Numerical optimization of a grating coupler for the efficient excitation of surface plasmons at an Ag-SiO2 interface'' Jesse Lu, Csaba Petre, Josh Conway, Eli Yablonovitch (JOSA B, 2007).
  
\section{Programming}          
    Fluent in Python, CUDA, and Matlab. \\
    Implemented cloud-based simulation service on Amazon EC2. \\
    Developed hardware-accelerated time- and frequency-domain
        electromagnetic solvers. \\
    Developed Matlab library for the design of arbitrary linear nanophotonic components.

\section{Research Experience}
   \vspace{-0.1in}	
   \begin{tabbing}
   \hspace{2.3in}\= \hspace{2.6in}\= \kill % set up two tab positions
    {\bf Graduate Researcher} \>Jelena Vuckovic Group \>2007-Present\\
                             \>Stanford University
   \end{tabbing}\vspace{-20pt}      % suppress blank line after tabbing
    Existing nanophotonic design process consisted of trial-and-error
        where each simulation about 24 hours to complete.
    Implemented FDTD and FDFD simulation software on hardware-accelerated platforms 
    to bring simulation times down to 10 minutes.
    Also developed a novel ``objective-first'' nanophotonic design algorithm
        and demonstrated automated design of arbitrary linear nanophotonic 
        devices.
   \begin{tabbing}
   \hspace{2.3in}\= \hspace{2.6in}\= \kill % set up two tab positions
    {\bf Undergraduate Researcher} \>Eli Yablonovitch Group \> 2005-2006\\
                          \>University of California, Los Angeles
   \end{tabbing}\vspace{-20pt}
    Surface plasmon focusing device required a grating coupler to inject energy
        into the surface plasmon mode.
    Wrote a Matlab program interfacing with a FEM package (COMSOL),
        developed a hierarchal optimization routine as well as an extended 
        transfer matrix model to produce first-ever thoroughly optimized 
        grating coupler design.
   
\end{resume}
\end{document}
