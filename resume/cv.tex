% LaTeX file for resume 
% This file uses the resume document class (res.cls)

\documentclass{res} 
%\usepackage{helvetica} % uses helvetica postscript font (download helvetica.sty)
%\usepackage{newcent}   % uses new century schoolbook postscript font 
\setlength{\textheight}{9.5in} % increase text height to fit on 1-page 

\begin{document} 

\name{Jesse Lu -- Curriculum Vitae\\[12pt]}
\address{\texttt{jesselu@stanford.edu} \\ 66 Newell Rd. Apt. O \\ East Palo Alto, CA \\ (408) 568-9356}
                                  
\begin{resume}

\section{Education}          
    Stanford University, Ph.D., Electrical Engineering, June 2013 (GPA: 3.657)\\
    Stanford University, M.S. , Electrical Engineering, May 2012 (GPA: 3.657)\\
    University of California Los Angeles, B.S., Electrical Engineering, June 2006 (GPA: 3.769)

\section{Honors}
    Stanford Graduate Fellowship, Stanford University, 2007 \\
    Dean's Honors List, University of California Los Angeles, five quarters

\section{PhD Research: Nanophotonic Computational Design and Optimization}
\begin{itemize}
\item Nanophotonic design and optimization was 
    a brute force, trial-and-error process
    which took \emph{weeks to months} to improve a single structure---even with 
    a good initial guess.
\item I built a system to design structures within a day,
    solely based on the performance specification of the desired device,
    and without good initial guesses.
\item I applied methods from the field of convex optimization 
    to the physical problem of linear nanophotonic design. 
    I sped up computation by implementing my method on GPUs,
    and achieved scale by leveraging the cloud (Amazon EC2).
\item This resulted in the design 
    of nanophotonic devices that were either
    more compact, more robust, or more efficient 
    than all previous designs.
    Critically, these designs only required a day (or two) of computation,
    and a specification for the intended performance (no good initial guess).
    \end{itemize}

\section{Publications}          
    ``Nanophotonic Computational Design'' Jesse Lu, Jelena Vuckovic (Optics Express, 2013). \\
    ``Objective-first design of high-efficiency, small-footprint couplers between arbitrary nanophotonic waveguide modes'' Jesse Lu, Jelena Vuckovic (Optics Express, 2012). \\
    ``Inverse design of a three-dimensional nanophotonic resonator'' Jesse Lu, Stephen Boyd, Jelena Vuckovic (Optics Express, 2011). \\
    ``Inverse design of nanophotonic structures using complementary convex optimization'' Jesse Lu, Jelena Vuckovic (Optics Express, 2010). \\
    ``Numerical optimization of a grating coupler for the efficient excitation of surface plasmons at an Ag-SiO2 interface'' Jesse Lu, Csaba Petre, Josh Conway, Eli Yablonovitch (JOSA B, 2007).
  
\section{Programming}          
    Fluent in Python, CUDA, Linux, C, C++, and Matlab. \\
    Implemented cloud-based simulation service on Amazon EC2. \\
    Developed hardware-accelerated time- and frequency-domain
        electromagnetic solvers. \\
    Developed Matlab library for the design of arbitrary linear nanophotonic components.

\section{Research Experience}
   \vspace{-0.1in}	
   \begin{tabbing}
   \hspace{2.3in}\= \hspace{2.6in}\= \kill % set up two tab positions
    {\bf Graduate Researcher} \>Jelena Vuckovic Group \>2007-Present\\
                             \>Stanford University
   \end{tabbing}\vspace{-20pt}      % suppress blank line after tabbing
%   Existing nanophotonic design process consisted of trial-and-error
%       where each simulation about 24 hours to complete.
%   Implemented FDTD and FDFD simulation software on hardware-accelerated platforms 
%   to bring simulation times down to 10 minutes.
%   Also developed a novel ``objective-first'' nanophotonic design algorithm
%       and demonstrated automated design of arbitrary linear nanophotonic 
%       devices.
   \begin{tabbing}
   \hspace{2.3in}\= \hspace{2.6in}\= \kill % set up two tab positions
    {\bf Undergraduate Researcher} \>Eli Yablonovitch Group \> 2005-2006\\
                          \>University of California, Los Angeles
   \end{tabbing}\vspace{-20pt}
%   Surface plasmon focusing device required a grating coupler to inject energy
%       into the surface plasmon mode.
%   Wrote a Matlab program interfacing with a FEM package (COMSOL),
%       developed a hierarchal optimization routine as well as an extended 
%       transfer matrix model to produce first-ever thoroughly optimized 
%       grating coupler design.
   
\end{resume}
\end{document}
