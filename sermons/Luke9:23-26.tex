\documentclass[12pt]{article}
\newcommand{\BI}{\begin{itemize}}
\newcommand{\EI}{\end{itemize}}
\newcommand{\I}{\item}
\newcommand{\Q}[1]{\begin{quote} #1 \end{quote}}
\begin{document}
\part*{Luke 9:23-26 -- Family Breakfast 2012}
\tableofcontents
\BI
\section*{Notes}
\I  Remember to drive the point home so that hearts are changed.
\I  Give comprehensive practical examples (application!).

\section{The radical, self-denying love we are to have for one another}
\I  We must first realize that, as His children, 
        God demands that we have 
        a radical, self-abasing love for one another.
\I  Philippians 2:3
    \Q{ Let nothing be done through selfish ambition or conceit, 
        but in lowliness of mind let each esteem others better than himself.}
\BI \I  It is a love that considers myself to be 
            the least important person in the room.
    \I  Everyone goes before me, because everyone is 
            more important and better than me.
    \I  Their needs, preferences, opinions, interests, and what they have to say
            are all more important than mine.
    \I  Last to be pleased, heard, considered, thanked, recognized, \ldots
    \I  First to serve, sacrifice, give way \dots 
        Since I am the most dispensable and least important one.
    \I  Goes against every part of my fleshly, worldly \EI

\I  John 13:14-15
    \Q{ If I then, your Lord and Teacher, have washed your feet, 
        you also ought to wash one another's feet.
        For I have given you an example, 
        that you should do as I have done to you.}
\BI \I  Not just an act of love on Jesus' behalf whereby He
            ``loved them to the end'' (John 13:1).
    \I  On the contrary, also an example for us to do as He did.
    \I  To be a common occurrence, even a pattern, among believers.
    \I  Similar to shining another's shoes,
            cleaning another's car,
            taking out another's garbage, or
            cleaning another's toilet -- but to the nth degree.
    \I  Jesus calls us to humiliate and debase ourselves for the good of another.
    \I  To treat ourselves as unworthy of respect and dignity,
            in order to work for the good of another. \EI

\I  1 John 3:16
    \Q{ By this we know love, because He laid down His life for us.
        And we also ought to lay down our lives for the brethren.}
\BI \I  The essence and definition of love.
    \I  Simply put, we are to die so that others may live,
            just as Christ laid down His life for us. \EI

\I  Is this what is missing in our churches, in our families, in our marriages, 
        and in our fellowship with one another?

\I  Namely, a lack of love in that we are not denying ourselves for one another?

\I  This is what our Lord Jesus Christ 
        calls, commands, and demands that we do for one another
        -- to die for one another.
\BI \I  This is what it means for a wife to submit to her husband:
    \BI \I  her preferences, desires, dreams 
        \I  where to go, what to do, how to live 
        \I  what she finds comfort and security in \EI
            all are sacrificed and lost in order to
            follow, obey, and submit to her husband.
    \I  This is what it means for a husband 
            to love his wife as Christ loved the Church:
    \BI \I  That your attitude is not that she exists to serve you,
            but that you exist to serve her.
        \I  That her needs, her wants, her likes and dislikes,
            her health, her happiness, and her well-being
            are more important than your own.
        \I  And so you give up your preferences, your ambitions,
            your dreams, your goals, your hobbies, your entertainment,
            and all rights to your time, and you die for her. \EI 
    \I  This is what it means for children to honor their parents:
    \BI \I  That you no longer decide what to do, how late to stay up, 
                who to hang out with, what to wear. \EI
        You give those up and die for your parents. \EI 
               
\I  This kind of radical, self-debasing love does not come easy.

\I  We find the necessary help to obey God's Word \emph{in} God's Word --
        He supplies what He commands.
        

\section{The source of self-denying love}
\I  I want to take us upstream to see where this kind of love originates from,
    in the hope that it might be stirred up in us.

\I  Luke 9:23-26
    \Q{ Then He said to them all, 
        ``If anyone desires to come after Me,
        let him deny himself, and take up his cross daily, and follow Me.
        For whoever desires to save his life will lose it,
        but whoever loses his life for My sake will save it.
        For what profit is it to a man if he gains the whole world,
        and is himself destroyed or lost?
        For whoever is ashamed of Me and My words,
        of him the Son of Man will be ashamed when He comes in His own glory,
        and in His Father's, and of the holy angels.}

\subsection{The context}
\Q{ Then He said to them all, ``If anyone desires to come after Me, \ldots}
\I  The topic is \emph{salvation}, not sanctification.
\BI \I  He is speaking to those who are deciding whether to follow Him,
            not to those who are following Him already.
    \I  Laying out the terms of discipleship.
    \I  Reinforced by the fact that Jesus is speaking to the crowd, 
            not just to the twelve disciples. \EI
\I  Rewinding the tape back to salvation (and specifically to this passage), 
        in order to find the power to love one another in sanctification.

\subsection{The terms of discipleship}
\Q{ \ldots let him deny himself, and take up his cross daily, and follow Me.}
\I  Jesus gives two conditions that must be met in order to be his disciple.
\BI \I  To deny himself: unconditional surrender, and
    \I  To take up his cross: unconditional service.
    \I  Note that following Jesus only comes \emph{after} the two conditions. \EI

\I  The first term, ``\ldots let him deny himself, \ldots''
\BI \I  A casting off and forfeiting of all that you are.
    \I  An unconditional surrender of 
    \BI \I  what we want to do, 
        \I  how and when we want to do it,
        \I  where we want to go,
        \I  how we want to live. \EI
    \I  Jesus demands that we disown ourselves and 
            cast off any right to our own lives.
    \BI \I  When a father disowns a son, 
                that son no longer has a share in the father's inheritance.
        \I  When we disown ourselves,
                we no longer have a share \emph{in ourselves}. \EI
    \I  To strip ourselves of every right and freedom,
            and to come to Jesus as a helpless child. \EI

\I  The second term, ``\ldots and take up his cross daily, \ldots''
\BI \I  The first term was a casting off and the second is a taking up.
    \I  Unconditional service wherever and in whatever Jesus commands,
            even unto death. 
    \I  The disciple's life is not a stroll in the park but a death march where
    \BI \I  we must faithfully bear whatever cross our Lord lays upon us, and
        \I  we must faithfully go wherever our Lord leads us, 
        \I  to whatever end He has chosen for us. \EI
    \I  The term is, ``You must faithfully do whatever I have for you,
            even unto death''
    \BI \I  Revelation 2:10b
            \Q{ Be faithful until death, and I will give you the crown of life.} \EI
\EI

\I  ``\ldots and follow Me.''
\BI \I  These terms are universal. 
        No one gets into God's kingdom without first submitting to these terms. 
    \I  They are Jesus' prerequisite to becoming His disciple.
    \I  And these are the terms you agreed to when 
            God saved you and imputed His Son's righteousness on your behalf
            if you are a true disciple of the Lord Jesus Christ. \EI

\subsection{The disciple's treasure}
\I  Now you may be wondering where the good news is,
        and it is in verse 24.
\Q{ For whoever desires to save his life will lose it,
    but whoever loses his life for My sake will save it.}
\I  ``For'' means that Jesus is explaining the previous verse.
    He is giving the principle that undergirds verse 23.
\I  He explains this principle from both a negative and positive
        perspective.

\I  Jesus starts with the negative,
    ``For whoever desires to save his life will lose it,\ldots''
\BI \I  Here, Jesus describes the one who, 
            upon hearing the terms of discipleship,
            no longer desires to be His disciple.
    \I  He is unwilling to surrender his life to serve Jesus.
    \I  Instead, this one desires to save his life.
    \BI \I  To use his time to pursue his own goals and to do his own will. 
        \I  Note the vanity of such a pursuit, that he cannot succeed in it,
                but he can only \emph{desire} to save his life. \EI
    \I  Instead, his end is that he loses the very life 
            he was attempting to save.
    \BI \I  Everything he gathered in this life is scattered upon his death, and
        \I  he now experiences eternal judgement in hell. \EI
    \EI

\I  Jesus then explains the positive, 
        ``\dots but whoever loses his life for My sake will save it.''
\BI \I  This is the one who agrees to Jesus' terms of unconditional 
            surrender and unconditional service.
    \I  He loses his life because he has denied himself and 
            has instead taken up a cross.
    \BI \I  He may no longer please himself, or
                even decide how to live.
        \I  However, note that he has not simply thrown his life away,
                but he has lost if \emph{for Jesus' sake}. \EI  
    \I  His outcome is salvation and eternal life. \EI

\I  This answers the question of why someone would
\BI \I  deny themselves,
    \I  take up their cross, and
    \I  lose their lives, \EI
    to follow Jesus.

\I  It is all in order to gain eternal life!
\BI \I  In fact, their sacrifice was never truly a sacrifice
            because of the surpassing value of the treasure
            they received in return. \EI

\I  Was this not true of you?
\BI \I  At the point of your salvation,
            what drove you to renounce all that you were,
            and to submissively commit yourself to all that Jesus
            would have for you?
    \I  Was it not because you were overwhelmed by the value of
            salvation, so that your loss seemed very small indeed?
    \I  Were you not willing, because of the riches of the treasure,
            to part with all that you had in order to obtain it? \EI

\I  The believer is empowered to deny himself by
        beholding the surpassing riches of God's promised blessing.
\BI \I  It is the greatness, extravagance, and sureness,
            of God's promise that empowers the believer to
            follow and obey! \EI

\I  And if so at the moment of salvation,
        should not it be ever more so after salvation?

\section{The power to love}
\I  Yes! Just as the power to deny ourselves and take up our cross was in
        the promise of eternal life,
        so the power to deny ourselves and obediently love one another now
        lies in \emph{the promise of eternal life now}.

\I  Remember, eternal life is not simply living forever, it is fellowship with God.
\BI \I John 17:3
        \Q{ And this is eternal life, that they may know You, 
            the only true God, and Jesus Christ whom You have sent.} 
    \I  Heaven is knowing God without any sin in the way! \EI
        
\I  Therefore, what empowers a believer unto obedience is 
        the blessing of a deeper and fuller communion with God.
\BI \I  Or, the believer obeys in order to 
            more richly experience eternal life \emph{now}. \EI

\I  Philippians 3:8
    \Q{ Yet indeed I also count all things loss for the excellence
            of the knowledge of Christ Jesus my Lord,
            and count them as rubbish, that I may gain Christ \ldots}

\I  Jeremiah 9:23-24, man's greatest glory
    \Q{ Thus says the LORD: "Let not the wise man glory in his wisdom, 
        let not the mighty man glory in his might, 
        nor let the rich man glory in his riches; 
        but let him who glories glory in this, 
        that he understands and knows Me, 
        That I am the LORD, exercising lovingkindness, judgment, 
        and righteousness in the earth. For in these I delight," says the LORD.}

\I  Ephesians 4:13, the goal of the Church
    \Q{ And He Himself gave some to be apostles, some prophets, 
        some evangelists, and some pastors and teachers, 
        for the equipping of the saints for the work of ministry, 
        for the edifying of the body of Christ, till we all come 
            to the unity of the faith 
        and of the knowledge of the Son of God, to a perfect man, 
        to the measure of the stature of the fullness of Christ; \ldots}

\EI

\end{document}
