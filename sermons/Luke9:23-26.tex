\documentclass[12pt]{article}
\newcommand{\BI}{\begin{itemize}}
\newcommand{\EI}{\end{itemize}}
\newcommand{\I}{\item}
\newcommand{\Q}[1]{\begin{quote} #1 \end{quote}}
\begin{document}
\part*{Luke 9:23-26 -- Family Breakfast 2012}
\tableofcontents
\BI
\section*{Notes}
\I  Remember to drive the point home so that hearts are changed.
\I  Give comprehensive practical examples (application!).

\section{The radical, self-denying love we are to have for one another}
\I  We must first realize that, as His children, 
        God demands that we have 
        a radical, self-abasing love for one another.
\I  Philippians 2:3
    \Q{ Let nothing be done through selfish ambition or conceit, 
        but in lowliness of mind let each esteem others better than himself.}
\BI \I  It is a love that considers myself to be 
            the least important person in the room.
    \I  Everyone goes before me, because everyone is 
            more important and better than me.
    \I  Their needs, preferences, opinions, interests, and what they have to say
            are all more important than mine.
    \I  Last to be pleased, heard, considered, thanked, recognized, \ldots
    \I  First to serve, sacrifice, give way \dots 
        Since I am the most dispensable and least important one.
    \I  Goes against every part of my fleshly, worldly \EI

\I  John 13:14-15
    \Q{ If I then, your Lord and Teacher, have washed your feet, 
        you also ought to wash one another's feet.
        For I have given you an example, 
        that you should do as I have done to you.}
\BI \I  Not just an act of love on Jesus' behalf whereby He
            ``loved them to the end'' (John 13:1).
    \I  On the contrary, also an example for us to do as He did.
    \I  To be a common occurrence, even a pattern, among believers.
    \I  Similar to shining another's shoes,
            cleaning another's car,
            taking out another's garbage, or
            cleaning another's toilet -- but to the nth degree.
    \I  Jesus calls us to humiliate and debase ourselves for the good of another.
    \I  To treat ourselves as unworthy of respect and dignity,
            in order to work for the good of another. \EI

\I  1 John 3:16
    \Q{ By this we know love, because He laid down His life for us.
        And we also ought to lay down our lives for the brethren.}
\BI \I  The essence and definition of love.
    \I  Simply put, we are to die so that others may live,
            just as Christ laid down His life for us. \EI

\I  Is this what is missing in our churches, in our families, in our marriages, 
        and in our fellowship with one another?

\I  Namely, a lack of love in that we are not denying ourselves for one another?

\I  This is what our Lord Jesus Christ 
        calls, commands, and demands that we do for one another
        -- to die for one another.
\BI \I  This is what it means for a wife to submit to her husband:
    \BI \I  her preferences, desires, dreams 
        \I  where to go, what to do, how to live 
        \I  what she finds comfort and security in \EI
            all are sacrificed and lost in order to
            follow, obey, and submit to her husband.
    \I  This is what it means for a husband 
            to love his wife as Christ loved the Church:
    \BI \I  That your attitude is not that she exists to serve you,
            but that you exist to serve her.
        \I  That her needs, her wants, her likes and dislikes,
            her health, her happiness, and her well-being
            are more important than your own.
        \I  And so you give up your preferences, your ambitions,
            your dreams, your goals, your hobbies, your entertainment,
            and all rights to your time, and you die for her. \EI 
    \I  This is what it means for children to honor their parents:
    \BI \I  That you no longer decide what to do, how late to stay up, 
                who to hang out with, what to wear. \EI
        You give those up and die for your parents. \EI 
               
\I  This kind of radical, self-debasing love does not come easy.

\I  We find the necessary help to obey God's Word \emph{in} God's Word --
        He supplies what He commands.
        

\section{The source of self-denying love}
\I  I want to take us upstream to see where this kind of love originates from,
    in the hope that it might be stirred up in us.

\I  Luke 9:23-26
    \Q{ Then He said to them all, 
        ``If anyone desires to come after Me,
        let him deny himself, and take up his cross daily, and follow Me.
        For whoever desires to save his life will lose it,
        but whoever loses his life for My sake will save it.
        For what profit is it to a man if he gains the whole world,
        and is himself destroyed or lost?
        For whoever is ashamed of Me and My words,
        of him the Son of Man will be ashamed when He comes in His own glory,
        and in His Father's, and of the holy angels.}

\subsection{The context}
\Q{ Then He said to them all, ``If anyone desires to come after Me, \ldots}
\I  The topic is \emph{salvation}, not sanctification.
\BI \I  He is speaking to those who are deciding whether to follow Him,
            not to those who are following Him already.
    \I  Laying out the terms of discipleship.
    \I  Reinforced by the fact that Jesus is speaking to the crowd, 
            not just to the twelve disciples. \EI
\I  Rewinding the tape back to salvation (and specifically to this passage), 
        in order to find the power to love one another in sanctification.

\subsection{The terms of discipleship}
\Q{ \ldots let him deny himself, and take up his cross daily, and follow Me.}
\I  Jesus gives two conditions that must be met in order to be his disciple.
\BI \I  To deny himself: unconditional surrender, and
    \I  To take up his cross: unconditional service.
    \I  Note that following Jesus only comes \emph{after} the two conditions. \EI

\I  The first term, ``\ldots let him deny himself, \ldots''
\BI \I  A casting off and forfeiting of all that you are.
    \I  An unconditional surrender of 
    \BI \I  what we want to do, 
        \I  how and when we want to do it,
        \I  where we want to go,
        \I  how we want to live. \EI
    \I  Jesus demands that we disown ourselves and 
            cast off any right to our own lives.
    \BI \I  When a father disowns a son, 
                that son no longer has a share in the father's inheritance.
        \I  When we disown ourselves,
                we no longer have a share \emph{in ourselves}. \EI
    \I  To strip ourselves of every right and freedom,
            and to come to Jesus as a helpless child. \EI

\I  The second term, ``\ldots and take up his cross daily, \ldots''
\BI \I  The first term was a casting off and the second is a taking up.
    \I  Unconditional service wherever and in whatever Jesus commands,
            even unto death. 
    \I  The disciple's life is not a stroll in the park but a death march where
    \BI \I  we must faithfully bear whatever cross our Lord lays upon us, and
        \I  we must faithfully go wherever our Lord leads us, 
        \I  to whatever end He has chosen for us. \EI
    \I  The term is, ``You must faithfully do whatever I have for you,
            even unto death''
    \BI \I  Revelation 2:10b
            \Q{ Be faithful until death, and I will give you the crown of life.} \EI
\EI

\ldots and follow Me.
\I  The conditions must be met before salvation and discipleship start.

\subsection{The principle}
For whoever desires to save his life will lost it, \ldots
\I  Note that no one can actually save their own life, 
        they can only desire/attempt to.

\ldots but whoever loses his life for My sake will save it.
\I  It must be for His sake!

\subsection{The reasoning}
For what profit is it to a man if he gains the whole world,
    and is himself destroyed or lost?
\I  We sacrifice only what we would inevitably lose anyways. 
    We make no \emph{true} sacrifice.

For whoever is ashamed of Me and My words,
    of him the Son of Man will be ashamed when He comes in His own glory,
    and in His Father's, and of the holy angels.

\section{The power of brotherly love}

\EI

\end{document}
