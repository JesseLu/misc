\documentclass[12pt]{article}
\newcommand{\BI}{\begin{itemize}}
\newcommand{\EI}{\end{itemize}}
\newcommand{\I}{\item}
\newcommand{\Q}[1]{\begin{quote} #1 \end{quote}}
\begin{document}
\part*{Revelation 1:4 --- Grace and Peace from God, Part 1}
\tableofcontents
\BI
\section*{Introduction}
\I Review v. 1-3:
\BI \I  Person of Revelation: Jesus Christ,
    \I  Purpose of Revelation: to show His servants,
    \I  Power of Revelation: the word of God,
    \I  Promise of Revelation: Blessed are those who hear and obey. \EI

\I  John then identifies Himself and His recipients and greets them with a blessing.
\I  Read v. 4-8, and introduce the two main points.




\section{The blessing of grace and peace}
\subsection{What is the blessing?}
\I  We can ask if the blessing is 
\BI \I  a well-wishing? No.
    \I  a prayer? No.
    \I  referring to some past or future event? No. 
    \I  the letter itself? Yes. \EI
\I  Revelation itself is grace and peace to believers.
\BI \I  In a way, John is saying ``Grace to you and peace'' while thrusting the letter into our hands. \EI
\I  John tells us what category the letter belongs in, and what effect it should have in our lives,
    namely grace and peace.
\BI \I  Not fear and anxiety, or
    \I  distress and mourning. \EI

\subsection{Why are these important?}
\I  These are two of the most dire needs of the believer.
\BI \I  Grace: Undeserved favor from God.
    \I  Peace: Steadfastness in trials. \EI
\I  In what areas of your life do you most need grace and peace?
\BI \I  In a family or marriage relationship?
    \I  In a difficult situation at work?
    \I  Dealing with a besetting sin?
    \I  Health issue? \EI
\I  John tells us that the source of grace and peace is God.
\BI \I  That is to say Revelation is grace and peace to us because it points us to God. \EI
\I  And so our most dire need is to know God, worship God, and obey God.
\BI \I  John immediately gives us help in this passage by showing us who God is. \EI

\section{The source of grace and peace}
\I  The source of grace and peace is the triune God.
\I  We will focus on the Father
\BI \I  The phrase ``Him who is and who was and who is to come'' is unique to Revelation.
    \I  Pregnant with meaning and endless in glory. \EI
\I  That God is ``Him who is nad who was and who is to come'' means
\BI \I  that He is eternal,
    \I  that He is transcendent,
    \I  that He is self-existent. \EI

\subsection{God is eternal}
\I  That is, He is eternally existent, He lives forever.
\BI \I    
\subsection{God is transcendent}
\subsection{God is present}
\EI

\end{document}
