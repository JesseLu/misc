\documentclass[12pt]{article}
\newcommand{\BI}{\begin{itemize}}
\newcommand{\EI}{\end{itemize}}
\newcommand{\I}{\item}
\newcommand{\Q}[1]{\begin{quote} #1 \end{quote}}
\begin{document}
\part*{Revelation 1:4 --- Grace and Peace from God, Part 2}
    \Q{\ldots and from the seven Spirits who are before His throne, \ldots}

\tableofcontents
\BI
\section*{Notes}
\I  The goal of this sermon is
\BI \I  to show who the Holy Spirit is
    \I  and to persuade us to be filled with Him. \EI

\section*{Introduction}
\I  The greeting in Revelation 1:4-8 is very unique.
\I  The majority of similar greetings in New Testament letters
    are concise and do not contain deep theology.
\BI \I  For instance, in the majority of Paul's epistles,
        he greets his recipients with a version of 
        \Q{Grace to you and peace from God our Father and the Lord Jesus Christ.}
    \I  And we find similar greetings used by Peter, Jude, and John. \EI
\I  However, Revelation 1:4-8 is much longer and contains deep theology. 
\BI \I  In fact, the two phrases in verse 4, 
        ``Him who is and who was and who is to come'' and 
        ``the seven Spirits before His throne''
        are first mentioned here and are only found in Revelation. \EI
\I  John is pointing to the God he just experienced as the
    source and substance of the believer's grace and peace.

\section{What it means}
\subsection{His identity}
\I  He is divine.
\BI \I  Revelation 1:4 
    \BI \I  He is sandwiched between the first and second members of the Trinity.
        \I  The source of blessing of grace and peace in every epistle is 
            God and God alone. \EI
    \I  Revelation 4:5, the scene John is most directly referring to,
        \Q{\ldots Seven lamps of fire were burning before the throne,
            which are the seven Spirits of God.} \EI
\I  Although seen as seven He is a single entity.
\BI \I  He appears as seven only in Revelation and only 
        when He is \emph{visually seen}.
    \I  When He speaks or works, He is always one.
    \BI \I  Revelation 14:13
            \Q{ Then I heard a voice from heaven saying to me,
                ``Write: `Blessed are the dead who die in the Lord from now on.' ''
                ``Yes,'' says the Spirit, 
                ``that they may rest from their labors, 
                and their works follow them.''} 
        \I  Revelation 22:17
            \Q{ And the Spirit and the bride say, ``Come!''
                And let him who thirsts come.
                Whoever desires, let him take the water of life freely.} \EI \EI
\I  Therefore we conclude, along with the rest of the Scripture,
    that the seven Spirits is the one Holy Spirit.
    
\subsection{His perfection}
\I  Note first that the sight of the Spirit as seven or seven-fold is
    \emph{not} a mistake.
\BI \I  The Spirit did not make a mistake in showing Himself to John
        as being seven-fold.
        In fact, he intentionally revealed Himself to John as such multiple times.
    \I  Moreover, John understood its importance and intentionally mentions
        the seven-fold Spirit at the very very outset of the letter.
    \I  Therefore, it must be important for us to understand what
        it means that the Spirit is seven-fold. \EI
\I  We can understand the significance of the seven-fold Spirit
    by observing the Biblical pattern 
    in which God uses a seven-fold repetition or progression to 
    signify completion or perfection.
\BI \I  Psalm 12:6, the perfection of God's word.
        \Q{ The words of the Lord are pure words, \\
            Like silver tried in a furnace of earth, \\
            Purified seven times.}
    \I  Genesis 1, the perfect, finished work of creation in seven days.
    \I  Leviticus 4:6, the finished atonement for a particular sin
        (of the high priest in this case),
        \Q{ The priest shall dip his finger in the blood and sprinkle some 
            of the blood seven times before the Lord, in front of the veil 
            of the sanctuary} \EI
\I  Of course, just because something is repeated or progresses in seven parts
    does not automatically make it perfect.
\BI \I  Matthew 18:21
        \Q{ Then Peter came to Him and said,
            ``Lord, how often shall my brother sin against me,
            and I forgive him? 
            Up to seven times?''} 
    \I  Rather, it is used by God to signify that something is perfect or complete. 
    \I  God sometimes uses a seven-fold repetition or progression
        to communicate perfection or completion to us. \EI
\I  It should be no surprise, then, that God working in a seven-fold manner
    abounds in Revelation, since it is a book of consummation.
\BI \I  This is most prominent in the three sets (seals, trumpets, bowls)
        of seven-fold judgements of God upon wicked humanity.
    \I  Revelation 15:1
        \Q{ Then I saw another sign in heaven, great and marvelous;
            seven angels having seven last plagues,
            for in them the wrath of God is complete.} \EI
\I  From this pattern, we understand that the seven-fold nature of the Spirit 
    communicates His perfection and completeness.
\BI \I  Not in any way lacking in any attribute of God.
    \I  He posesses all of the attributes of the Godhead to the fullest extent.
    \I  He is God undiminished and unabridged. 
        Nothing of God is toned down or omitted in Him.
    \I  He is an eternal, infinite being since 
        to be completely and perfectly divine is to be infinite.
    \I  He is all-knowing, all-powerful, all-capable.
    \I  He is holy, loving, and sovereign.
    \I  There is nothing He does not know, or does not control.
    \I  He is the complete infinity of Almighty God. \EI
\I  That He has a seven-fold nature means that
\BI \I  He is infinite in glory, and
    \I  that He is deserving of an eternity of worship. \EI
\I  It also means that if we were to study the Holy Spirit, 
    there would always be more to learn, and experience of Him.
\I  Show that He is all-capable in the salvation and sanctification of the believer.

\subsection{His submission}
\I  The Bible shows that the seven-fold Spirit of God is equal in person to 
    the Father and Son;
    however, it does not necessarily mean that He is equal in position, or role.
\I  From Revelation, we learn that He is lesser in authority than the Father.
\BI \I  From 1:4 and 4:5 we see the Father as being on the throne,
        and the Spirit as being in front of it.
    \I  Of course, the One on the throne has the supreme authority. \EI
\I  We also learn that He is under the authority of the Son.
\BI \I  Revelation 5:6
        \Q{ And I looked, and behold, in the midst of the throne and 
            of the four living creatues,
            and in the midst of the elders,
            stood a Lamb as though it had been slain,
            having seven horns and seven eyes,
            which are the seven Spirits of God sent out into all the earth.}
        The Spirit is pictured as belonging to the Lamb.
    \I  Revelation 3:1 is explicit.
        \Q{ These things says He who has the seven Spirits of God and 
            the seven stars: \ldots} \EI
\I  While equal in person, the Spirit joyfully submits in position/authority
    to both the Father and the Son.
\BI \I  No explicit mention in any worship scene in Revelation.
    \I  Does not appear on a throne, as the Father and Son do.
    \I  Instead, he directs worship to the Father and Son,
        as Galatians 4:6 says,
        \Q{ And because you are sons, God has sent forth the Spirit of His Son
            into your hearts, crying out, ``Abba, Father!''}
        The Holy Spirit does not direct us to worship Him, but the Father. \EI

\subsection{His ministry}
\I  Revelation not only reveals the submissiveness of the Spirit
    but also His working.
\I  He is, in a sense, God on the move, or God on the offensive.
\BI \I  First notice that Revelation 1:4 and 5:6 show 
        that He is in front of the throne.
    \I  And then Revelation 5:6 tells us that He is ``sent out into all the earth''.
    \EI
\I  And from John's gospel we know that,
\BI \I  John 3:8, He is the One whose working is compared to the wind which
        \Q{ \ldots blows where it wishes, and you hear the sound of it,
            but cannot tell where it comes from and where it goes.
            So is everyone who is born of the Spirit.}
    \I  and, John 16:8, who convicts
        \Q{\ldots the world of sin, and of righteousness, and of judgement: \ldots}
    \EI
\I  We could say that his work is a constant, global, invisible assault 
    to the glory of the Father and the Son.
\I  Specifically, His ministry on earth is the propagation and application
    of the word of God.
\BI \I  In other words, to bring the Word forth and 
        to bring it to bear on the hearts of men. \EI
\I  In Revelation, He brings forth the Word by ministering to John.
\BI \I  Revelation 1:10 \Q{I was in the Spirit on the Lord's Day, \dots}
    \I  Revelation 4:2 \Q{Immediately I was in the Spirit; and behold,
                        a throne set in heaven, and One sat on the throne.}
    \I  Revelation 17:3 \Q{So he [the angel] carried me away in the Spirit
                        in the wilderness.}
    \I  Revelation 21:10 \Q{And he [the angel] carried me away in the Spirit to 
                            a great and high mountain,} 
    \I  This is an extremely direct case of 2 Peter 1:21
        \Q{ \ldots for prophecy nevercame by the will of man,
            but holy men of God spoke as they were moved by the Holy Spirit.} \EI
\I  In Revelation, we also see that He ministers to the churches by bringing
    the Word to bear on their hearts.
\BI \I  Revelation 2:7, 2:11, 2:17, 2:29, 3:6, 3:13, 3:22
        \Q{He who has an ear, let him hear what the Spirit says to the churches.}
    \I  The words spoken by Jesus, written down by John, and
        ultimately communicated and delivered with power by the Spirit. 
    \I  John 16:13-14
        \Q{However, when He, the Spirit of truth, has come,
            He will guide you into all truth;
            for He will not speak on His own authority,
            but whatever He hears He will speak;
            and He will tell you things to come.
            He will glorify Me,
            for He will take of what is Mine and declare it to you.} \EI
        

        
            


\section{Application -- How it matters}
\subsection{The Spirit indwells the believer}
\subsection{Be filled with the Spirit}
\EI

\end{document}
