\documentclass[12pt]{article}
\newcommand{\BI}{\begin{itemize}}
\newcommand{\EI}{\end{itemize}}
\newcommand{\I}{\item}
\newcommand{\Q}[1]{\begin{quote} #1 \end{quote}}
\begin{document}
\part*{Revelation 1:1b-3 --- The Power and Promise of the Revelation of Jesus Christ}
\tableofcontents
\BI
\section*{Notes}
\I  The point of this sermon is:
        \begin{quote} 
        You need to know and obey the Revelation of Jesus Christ.
        \end{quote}
\I  Remember that with every subsection, your job is to answer three questions:
    \begin{enumerate}
    \I  What does it mean?
    \I  Why does it matter?
    \I  How does it change my life?
    \end{enumerate}


\section*{Introduction}
\I  The purpose of John's prologue is to prepare us to read, understand, and 
    obey the rest of the book.
\I  John does so by telling us (from last time)
\BI \I  the Person of Revelation, and
    \I  the purpose of Revelation. \EI

\section{The Power of Revelation}
\I  Lead: There is a reason why verse 2 doesn't simply say, ``who wrote it down.''
\I  John wants to emphasize the power of what follows, and he does so by describing it in three ways.

\subsection{Eternal: the word of God}
\I  Lead: Isn't it obvious that Revelation is the word of God?
\BI \I It should be to us, but not necessarily to the original recipients. \EI
\I  That Revelation is the word of God means that it is both eternal and living.
    Eternal in that 
\BI \I  it does not decay (Isaiah 40:8),
    \I  is complete, with no part missing or ommitted (22:18-19),
    \I  is perfect, and nothing can be added to it (Psalm 19:7);
\EI
    and living in that it
\BI \I  transforms, converting the soul (Psalm 19:7),
    \I  purifies, cleansing from sin (Psalm 119:9-10), and
    \I  strengthens, producing endurance for trials (Matthew 7:24-27).
\EI
\I  And so Revelation does not merely point to the truth, it \emph{is} the truth.

\subsection{Essential: the testimony of Jesus Christ}
\I  Lead: The next issue, is Revelation essential?
\BI \I Foundational or secondary?
    \I Part of the Christian's core curriculum, or an optional specialty?
\EI
\I  Holding to the testimony of Jesus Christ is the essence of what a Christian is. It was 
\BI \I  the reason for John's exile on Patmos (1:9), and
    \I  the reason why the saints were killed (6:9) and beheaded (20:4).
    \I  Believers are described as those who have the testimony of Jesus Christ (12:17),
    \I  and they overcome Satan by clinging to it, even to death (12:11).
\EI
\I  Testimony means a legally-binding witness as submitted to a judge.
\BI \I  As if Jesus is on the witness stand of a courtroom, and outlines 
        who He is, what He has done, and what He will do. \EI
\I  That Revelation is the testimony of Jesus Christ means that it is the essential truth of Jesus 
    (who He is, what He has done, and what He will do)
    that every Christian must cling to.

\subsection{Eyewitness: all things that he saw}
\I  John emphasizes that Revelation is a personal eyewitness account,
\BI \I  Not something he thought up or figured out. \EI
\I  Highlights the divine origin of the letter.
\BI \I  Revelation does not depend at all on John's ability to pastor, preach, interpret, or organize.
    \I  John only needs to see/hear and copy it down.
    \I  He is effectively Jesus' scribe/typist.
\EI
\I  Allows us to step into John's shoes and see what he saw.

\section{The Promise of Revelation}
\subsection{What is the blessing?}
\I  New life in Christ, or to have Christ wholly:
\BI \I  Jesus confesses you before the Father,
    \I  not hurt by the second death,
    \I  clothed in white garments, 
    \I  to eat from the tree of life,
    \I  to be given hidden manna to eat, and
    \I  given the morning star. \EI
\I  New identity in Christ, or to be Christ's wholly:
\BI \I  given a white stone with a new name no one else knows,
    \I  inscribed with the names of the Father, the Son, and the new Jerusalem,
    \I  to be a pillar in the temple of God and never leave,
    \I  to have power over the nations like Jesus, and
    \I  to sit with Christ on His throne. \EI

\subsection{Who is the blessing for?}
\I  Lead: Is it for select believers, or for the Church universal?
\I  The blessing is for those who overcome.
\BI \I All true believers overcome the world (1 John 5:5). \EI
\I  The blessing, essentially salvation in Jesus Christ, shows that it is for
    all Christians.

\subsection{How does the blessing matter?}
\I  Lead: Casts the Christian life in an entirely new light; it shows us 
\BI \I  what we are fighting for,
    \I  who we are fighting against,
    \I  how we must be victorious. \EI
\I  Christ's commands are not optional.
\I  The blessing gices us he power and motivation to overcome.

\EI
\end{document}
